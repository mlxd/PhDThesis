\documentclass[paper=a4, fontsize=12pt]{scrartcl}
\usepackage{blindtext}
\usepackage{enumitem}
\usepackage[a4paper, includehead, headheight=0.6cm, inner=3cm ,outer=2.5cm, top=2.5 cm, bottom=2.5cm]{geometry}  % Changing size of document
\usepackage[english]{babel} % The document is in English
%\usepackage[utf8]{inputenc} % UTF8 encoding
%\usepackage[T1]{fontenc} % Font encoding
\usepackage[no-math]{fontspec}
\usepackage{xltxtra,xunicode}%Above 2 lines redundant in xelatex

\usepackage{amsmath,amsfonts,amsthm} % Math packages
\usepackage{graphicx}
\usepackage{url}

%%% Custom sectioning
\usepackage{sectsty}
\allsectionsfont{\centering \normalfont\scshape}
%%% Custom headers/footers (fancyhdr package)
\usepackage{fancyhdr}
\pagestyle{fancyplain}
\fancyhead{}		% No page header
%\lhead[\thepage]{\thesection}
\rhead[]{}
\fancyfoot[L]{}											% Empty
\fancyfoot[C]{\thepage}											% Empty
\fancyfoot[R]{}									% Pagenumbering
\renewcommand{\headrulewidth}{0pt}			% Remove header underlines
\renewcommand{\footrulewidth}{0pt}				% Remove footer underlines
\setlength{\headheight}{10pt}
\newcommand{\horrule}[1]{\rule{\linewidth}{#1}} 	% Horizontal rule
\usepackage{eso-pic} % For the background picture on the title page
\newcommand\BackgroundPic{%
\put(-250,-140){%
\parbox[b][\paperheight]{\paperwidth}{%
\vfill
\centering
\includegraphics[width=\paperwidth]{../Images/symbol.jpg}%
\vfill
}}}


\begin{document}


\begin{titlepage}
%\AddToShipoutPicture*{\BackgroundPic}
    \begin{center}
        \vfill
        {\large \scshape Okinawa Institute of Science and Technology\\Graduate University}\\[0.7cm]
        \rule{\textwidth}{1.5pt}\\[0.75cm]
        {\LARGE \bfseries Thesis corrections report}\\[0.5cm]
        \rule{\textwidth}{1.5pt}\\[2.5cm]
        {\Large \bfseries Title:}\\[0.5cm]
        {\large  Non-equilibrium vortex dynamics in \\rapidly rotating Bose--Einstein condensate }\\[0.7cm]
        \vfill
        {\Large \bfseries Author:}\\[0.5cm]
        {\large  Lee James O'Riordan}\\
        \vfill
        {\Large \bfseries Examiners:}\\[0pt]
        \begin{align*}
            &\text{\large  Professor Simon Gardiner, Durham University}\\
            &\text{\large  Professor Gentaro Watanabe, Zhejiang University}\\
        \end{align*}
        \vfill

        \vspace{1cm}
        \rule{\textwidth}{1.5pt}\\[0.5cm]
         February, 2017
    \end{center}
\end{titlepage}

\section{Prof. Gardiner's comments}
\rule{\textwidth}{1.5pt}\\[0.5cm]
All points are addressed in the following changes or comments.


\begin{description}[align=left]
    \item [Throughout] $k_B$ has been modified to $k_\textrm{B}$.
    \item [Throughout] $\lambda_{dB}$ has been modified to $\lambda_\textrm{dB}$.
    \item [Throughout] Inline use of
\textbackslash frac\{a\}\{b\} has been modified to a/b.
    \item [Throughout] ``groundstate'' has been converted to ``ground state''.
    \item [Throughout] $g_{2D}$ has been modified to $g_\textrm{2D}$.
.
    \item [P iv] License for publication in Chapter 6 updated.
    \item [Throughout] ``i.e.'' has been modified to ``i.e.\textbackslash \ ''.
    \item [P6] Modified definition of laser cooling to specifically refer to doppler cooling.
    \item [P7] Modified introduction to Chapter 2 by correctly stating that Einstein translated Bose' paper for publication.
    \item [P10] $\mathbf{r}$ and $\mathbf{r}^{\prime}$ arguments are added to integral equation.
    \item [P10] $H_0$ notation better described.
    \item [P11,12] Eq. 2.13, 2.21 modified to $d/dt$ instead of $\partial / \partial t$. Query this one...
    \item [P12] Modification of line to read ``...the number of uncondensed atoms will be \textbf{essentially} zero...''
    \item [P17] Modification of line to read ``...the atoms will undergo \textbf{classical} rotation with the container...''
    \item [P20] Equation 2.45 modified to remove $C$.
    \item [P21] Eqs 2.49, 2.50 for energy functional modified to correct form.
    \item [P28] Time evolution derivation in Sec 3.1 modified to remove integral (Eq 3.3). Text clarification added to avoid confusion regarding an assumed approximation with Eq 3.6. Additionally, text modified to ``...assuming $\tau$ is a short time increment such that the formalism given in Sec.~\ref{sec:timeev} is valid'' to modify reference to removed equation.
    \item [P34] $2^8, 2^{11}$ expanded to $256,2048$ respectively.
    \item [P45] ``three distinctive eigenstates'' has been changed to ``three distinct eigenstates''
    \item [P49] ``waveg-uide'' hyphenation has been prevented.
    \item [P54] Modification of line to read ``...for Gross--Pitaevskii solutions with angular momentum...''
    \item [P59] Eq 3.28 and subsequent analysis modified to correct form.
    \item [P81] Grammatical dash in \LaTeX  code has been modified to ``-{}-{}-''.
    \item [P89] Section 5.1 ``perfect'' modified to ``near perfect'', and ``density'' modified to ``density profile''.
    \item [P89] Modification of line to read ``...Abrikosov lattice having 6-fold rotational symmetry i.e. a triangular lattice.''
    \item [P92] Modification of line to read ``For state-of-the-art theoretical analysis of condensates...''
    \item [P94] Modification of formula to $V_{\textrm{opt}} = (1/(2m))(\omega^2 - \Omega^2)r^2$
    \item [P101] Removed lines ``As the condensate system is finite-sized a refocussing of the interference structures will eventually take place. However, the time-scales necessary to examine this were beyond our numerical capabilities.''
    \item [P131] Comments and references add to ``hexatic phase'' description in Sec 6.4.
\end{description}
\noindent\rule{\textwidth}{1.5pt}\\[0.5cm]
\section{Prof. Watanabe's comments}
\rule{\textwidth}{1.5pt}\\[0.5cm]
All points are addressed in the following changes or comments.


\begin{description}
    \item [P iii] Additional comments added to the abstract describing the results and outcomes of the works carried out.
    \item [P1] ``...quantum computing application...'' modified to ``...computing applications...''.
    \item [P6] Added text ``...and the limit imposed by the resonance width of the atomic levels...''.
    \item [P7] Added text ``...for a fixed particle number, $N$...''.
    \item [P9] Added text ``...the diluteness requirements and higher masses of most atoms...''.
    \item [P11] Additional $\dagger$ removed from Eq 2.12c
    \item [P12] Modified text to read ``...the gas is sufficiently dilute that only two-body interactions occur''.
    \item [P13] Modified text to read ``$\Psi = \langle \hat{\Psi} \rangle$ is known as the condensate wavefunction''.
    \item [P13] ``equivalent'' changed to ``equal''
    \item [P14] $\Omega$ changed to $\mathbf{\Omega}$
    \item [P15] Eq 2.35a $L$ changed to $L_z$
    \item [P16] Modified text to read ``...determine the energetic stability of the system. The norm of these functions is given by $N_{\textrm{BdG}}=\int d\mathbf{r}(|u|^2 - |v|^2)$. If the norm is positive, with positive eigenvalues, the system is energetically stable. If the norm is positive with negative eigenvalues...''
    \item [P16] BdG matrix given by Eq 2.37 is non-Hermitian, and so the additional $\sigma_i$ term is not necessary. While to solve for complex eigenfunctions may require application of $\sigma_i$ to LHS [PHYSICAL REVIEW A 77, 043601 (2008)], the formalism presented should still hold for the described theory.
    \item [P16] Changed ``integrating over all space...'' to ``...integrating over $z$...''.
    \item [P17] ``...without any loss of generality.'' modified to ``...and reproduce much of the same vortex dynamics as a tightly confined three dimensional model.''.
    \item [P19] ``Following the formalism...'' has been modified to ``Multiplying by...'', and an additional reference added.
    \item [P21] Circulation defined as $h/m$.
    \item [P21] $d\mathbf{r}$ added to Eq 2.49
    \item [P21] Text modified to ``For a superfluid condensate in the rotating frame $\Omega_c = {E_v/L} = {E_v/(N\hbar)}$, where $E_v$ is the energy of the vortex, and $L$ is the angular momentum component of the superfluid along $z$''.
    \item [P24] Brackets added to Eq 2.51
    \item [P24] $\Omega$ changed to ${\Omega}_z$
    \item [P24] Additional citations added.
    \item [P24] $\Omega$ changed to $\Omega_\perp$.
    \item [P24] Changed text to read ``...sees the vortex cores expanding and forming a densely packed lattice...''
    \item [P26] Text modified to ``Monte Carlo methods generally do not allow for real-time dynamics, or reproducing the underlying wavefunction''.
    \item [P33] Added text ``...and $c_{\textrm{x}}$ are linear coefficients for the states''.
    \item [P45] Modified text to read ``For three equivalent potentials, $L,M,R$, with degenerate states $\{|L \rangle,|M\rangle,|R\rangle\}...$.
    \item [P48] Text modified to read ``...an additional harmonic oscillator potential, $V_z = m\omega_z^2 (z-z_0)^2/2$, is applied in the same direction to impart motion to the atom, which is initially at the $z=0$ position of the atom-chip. We set $z_0 = (\max z)/2$ to ensure the oscillator potential sits at the atomchip centre. ''
    \item [P49] Modified text to read ``...the assumption of all traps being on resonance (degenerate) ...''
    \item [P56] FSO replaced with ``Fourier split-operator''
    \item [P66] Additional references added to ``...bias field potential...''
    \item [P69] $\Omega_z$ changed to $\Omega$
    \item [P69] Eq 4.5 $(\omega - \Omega)^2$ corrected to $(\omega^2 - \Omega^2)$
    \item [P70] Modified text to read ``The system examined here is well within this regime for a frequency of $\Omega = 0.995\omega_\perp$.''
    \item [P71] Modified text to read ``In this rapidly rotating regime ''
    \item [P71,72] While I understand the examiner's preference for $\mu$m and nm, meter is a more general unit and so I prefer to keep to it in this work.
    \item [P76] Sign corrections to eq 4.11 4.13 and 4.15
    \item [P78] Additional citations added.
    \item [P86] Modified Figures 4.13, 4.14
    \item [P90] Text modified to read ``...and assuming the chosen rotation rate of $\Omega = 0.995\omega_\perp$, after rotating for
    $ \approx 1/6$\ s the system has returned to its initial orientation...''
    \item [P90] Modified text to read ``...the vortices near the edges are separated by a slightly larger distance than those at the centre...''.
    \item [P91] Modified text to read ``...shorter than both the rotation period of the vortex lattice, as well as the speed of sound in the condensate...''.
    \item [P92] Modified text to read ``...where the frequency of the oscillations...''.
    \item [P93] I have added additional commentary about the kinetic energy spectra, including some references linking the power-spectra and autocorrelations of signals. An additional discussion is also added for applying the methods to the condensate analysis by considering the classical continuity equation. I think the added text is adequate to justify the use of the classical methods in this situation.
    \item [P94] Modified text to read ``...the harmonic oscillator frequency  as $V_{\text{opt}} = (1/(2m))(\omega^2_\perp - \Omega^2)
    {r}^2$...''.
    \item [P95] Due to the resolution of the kinetic energy analysis, I do not expect the $10^2 \to 10^4$ difference in compressible to incompressible energies to be anything more than due to residual background values that are cross-talk between the two terms, and I expect these to vanish with increased kinetic energy sampling. I do not expect these to be numerical noise, and so having some structure is fine, given the difference in magnitudes. However, due to the difficulty in further increasing the resolution it is not currently possible to better resolve these. As such, I will not change this.
    \item [P100] Figure 5.7 $g_l$ changed to $g_{ll^\prime}$
    \item [P103] Text in caption of Fig 5.11 ``larger structures'' changed to ``structures with higher amplitudes''.
    \item [P111] $\mathbf{r}_{j,k}$ from Eq 6.2 defined.
    \item [P111] Text ``($n_j=6$ for a perfect triangular lattice)'' added.
    \item [P125] Text modified to ``We can use''
    \item [P128] Text ammended to read ``topological lattice defects due to the local rearrangement of the vortices, which persist for long times''. Given the prevalence of similar defects in solid-state systems, as well as in real vortex lattice systems, I do not expect angular momentum conservation to have much effect on this.
    \item [P128] Text modified to read ``This lowered the correlation value, which indicated a...''
    \item [P131] KTHNY abbreviated, hexatic phase explained, and relevant references given. Although angular momentum is conserved here, the appearance of defects and their mobility is not affected, so one may observe this phase, though as stated in the text, further research into this is required.
    \item [P137] Modified text and abbreviation of ``ZNG'' to ``Zaremba--Nikuni--Griffin (ZNG) ''

    \item [Point 3] While I think this can be a nice study, I do not currently see the applicability of the XY model consideration with regard to this system, as I think it does not benefit the work carried out. This is something that can be potentially examined at a future date.

    \item [Throughout] References on pages 19, 24, 66, 78 and 131 have all been updated to include additional publications. Reference to Tkachenko's work also added.
\end{description}

\end{document}
