In this thesis I have presented my work on the dynamical behaviour of out-of-equilibrium Bose--Einstein condensate dynamics in the presence of vortices. I began by introducing the necessary theory to model the condensate, using the mean-field Gross--Pitaevskii equation, and following this by discussing techniques for numerical simulation. The field of GPU computing is discussed, with the body of work in this thesis relying heavily of these cutting-edge computational acceleration methods. The development of a toolset, titled ``GPUE'', is mentioned, and forms the backbone of all the numerical simulations performed.

Vortex lattice...

From this tool, I next discussed the behaviour of few vortex condensates, and introduce two distinctive perturbation techniques through which I use to disturb the condensate. The first method is that of a flashed, or ``kicked'' optical potential, which modifies the condensate phase with the effect observed during dynamical evolution. With the use of an optical lattice matching the structure and lattice constant of the vortex lattice, the kick allows for the generation of transient, time-varying moir\'e interference structures in the density. This interference arises from the interference of the reciprocal lattice vectors of both the optical and vortex lattices. Through varying the alignment angle of the optical lattice relative to the vortices, the wavelength of the moir\'e structure can be changed. The kicking method shows how significantly robust the vortex lattice is to density variations, with the resulting phonons following a kick having little to no effect on their respective positions.


To investigate methods to disorder the vortex lattice, the second perturbation I introduce is the phase imprinting of topological defects. From the well ordered vortex lattice groundstate, I annihilate vortices at predefined positions in the lattice. While this is known to create phonons as a result of the annihilation, that was shown to have little to no effect on the vortex lattice. As a vacancy is created in the vortex lattice following the annihilation, the remaining vortices will attempt to redistribute and reorder to the most favourable position. Through extensive simulations, this is shown to create localised topological defects in the lattice, with the global lattice maintaining a large degree of order. The use of Delaunay triangulation allowed for identification of the defect types, largely showing the appearance of (5,7) topological lattice defects. Orientational correlations of the lattice showed that different degrees of imprints created varying degrees of lattice disordering, and is additionally demonstrated for a varying number of cases. Voronoi tessellation allowed for the local variations in lattice area and correlations to be identified following an annihilation.

The discussed methods represent, in this author's opinion, two very useful techniques for quantum state control and engineering. For the optical lattice kick, the creation of interference patterns with wavelengths much greater than the lattice spacing opens the possibility to detect vortices without time-of-flight expansion in a lattice. This technique is a unique method to examine the periodicity of a lattice system.

The vortex annihilation through phase imprinting appears to be a very good candidate to create varying degrees of disorder in a vortex lattice system. A potential use for this is to create controllable routes towards quantum turbulence from a well-ordered system. An additional potential use is for the investigation of crystal defects in condensed matter systems. Given the the large difference in scales between solid-systems and condensates, cold-atomic gases can analogously be used.
