
\section{Conclusions}
In this thesis we have presented our work on the dynamical behaviour of non-equilibrium Bose--Einstein condensates, where we have examined the behaviour of vortex lattices subjected to two distinct perturbations. We began by modelling the condensate using the mean-field Gross--Pitaevskii equation. Using this formalism we then discussed the superfluid properties of the condensate, and concentrated primarily on states with vorticity in two-dimensions. We primarily discussed high rotation rates of the condensate, where the condensate attains a large number of singly charged vortices arranged in a triangular Abrikosov vortex lattice. We restricted ourself to a rotation rate of $\Omega = 0.995\omega_\perp$m, and sought a numerical solution of the system with $N\approx 10^{6}$ atoms of $^{87}$Rb.

We next introduced the algorithmic framework to numerically solve this system. For this we used the Fourier split-operator method, where we made use of imaginary time evolution to determine the vortex lattice groundstate, and real time evolution for all subsequent dynamics. The simulation of this system was computationally challenging due to the finely sampled numerical grid size required to resolve all features of the condensate in position and momentum space. To overcome this challenge, we introduced GPU computing methods. We considered the problem of coherent atomic transport, where we investigated a system using SAP methods with magnetic waveguides on atomchips. The performance gained from these advanced computational techniques through a fully three-dimensional simulation of the Schr\"odinger equation showed significant increase over a standard CPU implementation. The resulting GPU code was compared with a traditional MPI-enabled code, and showed equivalent performance to an 8-core 8-node cluster for the same parameters. This led to the development of a software suite for the numerical solution of the Gross--Pitaevskii equation titled ``GPUE'', which was found to outperform other suites for the same class of problems.

Using the developed suite, we performed a series of numerical simulations for stationary and rotating condensates with a low number of vortices. We showed the velocity fields of the condensates for low vortex numbers, as well as for the vortex lattice solution which demonstrated solid body rotation. We mentioned the necessary criteria for ensuring a well-ordered vortex lattice, and examined a condition were these criteria are unfulfilled. Following this, we next introduced two distinctive perturbation techniques. Through use of these we aimed to disturb the condensate and allow non-equilibrium dynamics to be observed. The first method was a kicked optical potential, which modified the condensate phase. By matching the structure and lattice constants of both the optical and vortex lattices, the kick allowed for the generation of transient, time-varying superlattice structures in the density. These superlattice structures were observed during the subsequent dynamical evolution following the kick. By varying the alignment angle of the optical lattice relative to the vortices, we showed that the wavelength of the structures could be changed. The change in the structures were explained using moir\'e interference theory, and arose from the interference between the reciprocal lattice vectors of both the optical and vortex lattices. The kicking perturbation showed how robust the vortex lattice was to density variations, with the phonons generated by the kick having little to no effect on the vortex positions.

As the robustness of the vortex lattice was demonstrated following the kicked potential, we next investigated methods to create controllable disorder in the vortex lattice. Through the use of direct phase imprinting of topological excitations (phase singularities), we demonstrated that this was possible. From the well ordered vortex lattice groundstate we annihilated or flipped the rotation direction of vortices at predefined positions in the lattice. As a vacancy was created in the vortex lattice following an annihilation, the remaining vortices attempted to redistribute and reorder to the most favourable position. Through extensive simulations, this was shown to create localised topological defects in the lattice, with the overall lattice still maintaining a large degree of order. Varying degrees of disorder were then created by removing additional vortices, or by flipping a vortex rotation profile. The use of Delaunay triangulation allowed us to easily identify the defect types, and largely showed the appearance of (5,7) topological lattice defects. By examining the orientational correlations of the lattice we observed that different imprints created varying degrees of lattice disordering. We then made use of Voronoi tessellations to allow local variations in lattice area and orientational correlations respectively to be identified following an annihilation, and demonstrated the effect the phase imprinting had on the vortex lattice on different timescales.

\section{Outlook}
Given the current state-of-the-art experimental control of condensate systems through use of SLMs, the perturbation methods discussed within this thesis are expected to be realisable. These perturbations represent two very useful techniques for quantum state control and engineering. For the kicked optical lattice, the creation of moir\'e interference patterns with wavelengths much greater than the lattice spacing opens the possibility for detecting vortices without time-of-flight expansion in a lattice. We consider this technique to be a unique method for examining the periodicity of a lattice system, where the evolving pattern can also potentially be observed through the \textit{in-situ} imaging techniques, as discussed in~\ref{sec:intro_super}. Further extensions of this work can involve investigating the periodicity of large-scale soliton trains in quasi-1D condensates.

Some preliminary work in small-scale zig-zag and linear vortex crystals was carried out in conjunction with A.~Barahmi and Th.~Busch. This showed that with little periodicity in the system there was negligible observation of any peaks in the compressible energy spectrum. As a result, there were no discernible moir\'e superlattice patterns in the condensate density. It is expected that for these structures to be observed that highly periodic systems with a well defined reciprocal lattice are required. For highly periodic systems which are difficult to visualise, the moir\'e interference technique can allow the underlying structure to be determined. However, if the system is disordered, the well defined peaks will become less visible, and possibly disappear entirely dependent upon the degree of disorder. As a result, this method could potentially allow for an examination of lattice disorder, and can form the basis of a future investigation.

The vortex annihilation/flipping through phase imprinting appears to be a very good candidate to create varying degrees of disorder in a vortex lattice system. The analysis methods discussed and used for this work could easily be applied to real experimental data. A potential use for this is for creating controllable routes towards quantum turbulence from a well-ordered system. While the examination presented focussed primarily on the use of phase profiles opposite to that of the lattice, the imprinting of like-signed vortices also remains an interesting choice. By forcing vortices into different locations in the lattice is potentially an additional method to create lattice dislocations and hence topological lattice defects. Potentially, one might consider erasing and adding vortices at different locations to both create and remove topological lattice defects. Potentially, one can imagine using this as a form of memory storage technique in quantum computing. The applicability of this method can potentially be examined in a future work.

Additionally, one can also create multi-charged vortices in the condensate. The effect of the surrounding lattice on the resulting multi-charged vortex would be an interesting problem. One might expect the $l$-charge vortex decay to be suppressed if the energy to move the surrounding lattice vortices is greater than the energy to maintain the $l$-charged vortex. This was briefly investigated by examining the Bogoliubov-de Gennes solutions of the imprinted vortex lattice system, with the aim of observing if the resulting modes are complex. The excitation modes were, however, not found due to the numerical complexity of the problem, and it remains an open question if this suppression exists. This will be investigated in a future work.


While we briefly mentioned the search for a KTHNY hexatic phase transition in this system, this will require further examination. Future work can include an investigation for the existence of this transition, and examine whether dislocation mediated melting can occur as a resulting of the phase imprinting techniques. Though we consider the framework developed and examined for all the above methods to be valid, the inclusion of a coupling to a thermal cloud and higher level condensate states would ensure that the investigated methods are truly physically realistic. For such finite temperature condensates, one might consider the use of the ZNG formalism~\cite{BK:Proukakis_finitetemp_2013}, or the number conserving approach for a non-equilibrium driven condensate~\cite{BEC:Billam_pra_2013}. An extension of the above works can examine this.

The use of GPU computing for simulating quantum dynamics is currently an under-utilised paradigm. The potential for a significant performance gain exists, given an effective mapping of a numerical algorithm to the GPU hardware. While the code developed and utilised for all the above simulations offers a clear performance advantage, it should be noted that further development and maintenance of such code can be challenging. Rapid changes to the CUDA programming models have introduced many new features to the standard which could potentially be used for solving more complex problems of both linear and nonlinear Schr\"odinger-type problems. An extension of the GPUE codebase to cover one and three dimensional Gross--Pitaevskii systems will allow for this suite to be as feature rich as the currently most capable suites available~\cite{NUM:Wittek_cpc_2013,NUM:GPElab_1}, whilst still holding the current edge in performance. Solutions using arbitrary gauge fields for these problems will also offer a distinctive advantage by allowing this software to become a very general solver suite for BEC problems.

The methods and works examined in this thesis offer interesting answers, questions and possibilities for the future of controllable quantum systems and technologies.
