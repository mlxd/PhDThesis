
\section{Conclusions}
In this thesis we have presented work on the dynamical behaviour of out-of-equilibrium Bose--Einstein condensate dynamics in the presence of vortex lattices. We began by introducing the necessary theory to model the condensate, using the mean-field Gross--Pitaevskii equation, and showed the effect of reduced dimensionality on the model. From this Gross--Pitaevskii formalism, we then discussed the superfluid properties of the condensate, with an emphasis on vortex solutions. We primarily discussed high rotation rates of the condensate ($\Omega = 0.995\omega_\perp$), where the condensate attains a large number of singly charged vortices arranged in an Abrikosov vortex lattice. By examining the literature we showed that the mean-field limit holds for high angular momentum states holds true for this rotation rate, and so we sought a numerical solution of the system at this rotation rate with $N\approx 10^{6}$ atoms of $^{87}Rb$.

For a numerical solution of this system, we introduced the necessary algorithmic framework. This consisted of the Fourier split-operator method for imaginary time evolution to determine the vortex lattice groundstate, and real time evolution for all further dynamics. The simulation of this system was computationally difficult due to the required numerical grid size to resolve all features. To overcome this difficulty, we introduced GPU computing methods. We next demonstrated the performance of these advanced simulation techniques through a fully three-dimensional simulation of the Schr\"odinger equation, which was the first time this was used as far as we were aware. For this we investigated a system for the coherent transport of atoms using SAP methods, where we demonstrated an effective and realistic system to realise this. The resulting GPU performance for this problem was compared with a traditional MPI-enabled code, and showed equivalent performance to an 8-core 8-node cluster for the same parameters. Following this, we developed a code suite for the numerical solution of the Gross--Pitaevskii equation titled ``GPUE'', which was found to outperform in simulation time other suites for the same class of problems.

Using the developed suite, we performed a series of numerical simulations for condensates with a low number of vortices to demonstrate the effectivity of the code. We have shown the velocity fields of the condensates for a low vortex numbers, as well as for the vortex lattice solution. We mention the necessary criteria for ensuring a well-ordered vortex lattice, and demonstrate a condition were these criteria are unfulfilled. Following this, we next introduced two distinctive perturbation techniques through which we aimed to disturb the condensate. The first method was that of a flashed, or ``kicked'' optical potential, which modified the condensate phase, and was observable during dynamical evolution. With the use of an optical lattice matching the structure and lattice constant of the vortex lattice, the kick allowed for the generation of transient, time-varying moir\'e interference structures in the density. These structures arose from the interference of the reciprocal lattice vectors of both the optical and vortex lattices. Through varying the alignment angle of the optical lattice relative to the vortices, the wavelength of the moir\'e structure could be changed. The kicking method showed how significantly robust the vortex lattice was to density variations, with the resulting phonons following a kick having little to no effect on their respective positions.

While the robustness of the vortex lattice was demonstrated following the kicked potential, methods to create controllable disorder in the vortex lattice were investigated next. Through the use of direct phase imprinting of topological defects, we demonstrated that this was possible. From the well ordered vortex lattice groundstate, we annihilated or replaced vortices at predefined positions in the lattice. As a vacancy was created in the vortex lattice following an annihilation, the remaining vortices attempted to redistribute and reorder to the most favourable position. Through extensive simulations, this was shown to create localised topological defects in the lattice, with the global lattice maintaining a large degree of order. Varying degrees of disorder were thus created by removing additional vortices, or by flipping a vortex rotation profile. The use of Delaunay triangulation allowed for identification of the defect types, largely showing the appearance of (5,7) topological lattice defects. Orientational correlations of the lattice showed that different degrees of imprints created varying degrees of lattice disordering, and is additionally demonstrated for a varying number of cases. Voronoi tessellation allowed for the local variations in lattice area and correlations to be identified following an annihilation, and demonstrated the effect the phase imprinting had on the vortex lattice on different timescales.

\section{Outlook}


With the experimental availability of the aforementioned techniques for generating the perturbances, a realisattion of the simulated systems in this thesis should be a possibility. These methods represent, in this author's opinion, two very useful techniques for quantum state control and engineering. For the optical lattice kick, the creation of interference patterns with wavelengths much greater than the lattice spacing opens the possibility to detect vortices without time-of-flight expansion in a lattice. This technique is a unique method to examine the periodicity of a lattice system. The evolving pattern can also potentially be observed through the \textit{in-situ} imaging techniques, as discussed in~\ref{sec:intro_super}. Further extensions of this work could involve investigating the periodicity of large-scale soliton trains in quasi-1D condensates. Some preliminary work in zig-zag and linear vortex crystals, in conjunction with A.~Barahmi and Th.~Busch, showed that without well defined periodic behaviour there is negligible observation of any peaks in the compressible energy spectrum. As such, there were no discernible moir\'e patterns in the density. It is expected that for such an effect to be useful highly periodic systems are required, with well defined wave-vectors. It remains an open question as to whether this can be used in a realistic experimental setting for such a purpose.

The vortex annihilation/flipping through phase imprinting appears to be a very good candidate to create varying degrees of disorder in a vortex lattice system. A potential use for this is to create controllable routes towards quantum turbulence from a well-ordered system. While we have demonstrated the use of this method to simulated data, the proposed methods could easily extend to experimental works. While the investigation of KTHNY style phase transitions will require further examination, the appearance of such a phase would potentially allow a much easier access to such a phase in quantum systems than in other solid-state materials. Future work will include an investigation for the appearance of these phase transitions.

While the examination presented focussed primarily on the use of phase profiles opposite to that of the lattice, the imprinting of like-signed vortices also remains an interesting choice. By forcing vortices into different locations in the lattice allows for the creation of additional methods to dislocate the lattice. Potentially, one might consider erasing and adding vortices at different locations to both create and remove topological lattice defects. Potentially, one can imagine using this as a form of memory storage technique in quantum computing. The applicability of this method can potentially be examined in a future work.

Additionally, one can also create multi-charged vortices in the condensate. The effect of the surrounding lattice on the resulting multi-charged vortex would be an interesting problem. One might expect the decay to be suppressed if the energy to move the vortices to accommodate this is greater than the energy to maintain the $l$-charged vortex. This was briefly investigated by examining the BdG solutions of the problem, with the aim of observing if the resulting modes are complex. The BdG modes were, however, not found due to the numerical complexity of the problem, and it remains an open question if this suppression exists.

Though we do consider the framework developed and examined for all the aforementioned methods to be valid, the inclusion of a coupling to a thermal cloud and higher level condensate states would ensure that the investigated methods are truly physically realistic. For this, one might consider the use of the ZNG formalism~\cite{}. A future extension of the above works could examine this.

The use of GPU computing for simulating quantum dynamics is currently an under-utilised paradigm. The potential for significant performance increases exists, given an effective mapping to the GPU hardware. While the code developed and utilised for all the above simulations has a clear performance win, it should be noted that further development and maintenance of such code can be difficult. Rapid changes to the CUDA programming models have introduced many new features to the standard, of which could potentially be used for solving more complex problems of both linear and non-linear Schr\"odinger-type problems. An extension of the GPUE codebase to cover one and three dimensional Gross--Pitaevskii systems will allow for this suite to be as feature rich as the currently most capable suites available~\cite{NUM:Wittek_cpc_2013,NUM:GPElab_1}, whilst still holding the current edge in performance. Solutions using arbitrary gauge fields for these problems would also offer a distinctive advantage, and would allow this software to become a very general solver suite for BEC problems.

The methods examined in this thesis offer, in our opinion, some interesting questions for the future of controllable quantum systems and technologies.
