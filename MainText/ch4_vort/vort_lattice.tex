%\chapter{Bose--Einstein condensate dynamics}
\section{Condensate dynamics}
With the background theory outlined in Sections~\ref{sec:superfluid,sec:coldatoms}, I will now discuss manipulations and dynamics of the condensate. Due to phase coherence over all atoms, we can say that any perturbation to the condensate phase will (given sufficient time), have some affect on all condensed atoms. Direct manipulation of the phase is interesting as the condensate phase encapsulates all quantum behaviour of the system. Secondly, given the phase term is, following Eq.~\ref{eqn:madelung}, modifiable to any arbitrary values (within the range $\pm \pi$).

Also, with the condensate velocity determined by Eq.~\ref{eqn:velocity}, it is possible, upon engineering of the condensate phase, to control the atomic velocity. This opens an interesting set of possibilities as control of the phase, and hence velocity, allows for development of a wide range of quantum states and dynamics.




\section{Phase imprinting and manipulation}
Writing the wavefunction in the standard Madelung transform form, alongside a phase imprint term $\phi$ is given as
\begin{equation}
    \Psi(\mathbf{r},t) = |\Psi(\mathbf{r},t)|e^{\text{i}(\theta(\mathbf{r},t) + \phi(\mathbf{r}))}.
\end{equation}


%%% TO BE MOVED IN WITH PHASE ENGINEERING
Following \cite{BK:Pitaevskii_Stringari_2003} and taking the Madelung transform of the wavefunction given by Eq. \eqref{eqn:madelung}, the phase of the condensate may be specified as
\begin{equation}
\theta = \theta_c + \theta_i,
\end{equation}
where $\theta_c$ is the unperturbed condensate phase, and $\theta_i$ is the phase pattern to be imprinted. Thus, upon solving for the initial condensate phase, an additional phase pattern can be imprinted at any time by multiplying the wavefunction by the intended phase pattern. This is in line with the phase imprinting method, as previously introduced by Dobrek \textit{et al}. \cite{Vtx:Dobrek_pra_1999}. 
%%% TO BE MOVED IN WITH PHASE ENGINEERING



\subsection{Gaussian phase}

\subsubsection{One dimensional}

\subsubsection{Two dimensional}



\section{Few vortex states}
As discussed in Section~\ref{sec:superfluid}, the discovery and manipulation of quantum vortices remains an active area of research.







Given that the energy of a vortex-carrying condensate scales as $E\propto l^2$, as $L_z \propto l$, any increase in the angular momentum causes a squared increase in the energy. Thus, for energetic favorability, the system prefers to maintain singly-charged vortices. To generate a vortex in the condensate costs energy,

Wi

\section{Quantum vortex dynamics}



\section{Vortex lattice states}
    \begin{equation}
        E(\Psi) = \int \Psi^{*} H_{\text GP} -\Omega L_z \Psi
    \end{equation}
