In this document I review the area of cold atomic gases, and in particular the superfluid behaviour of Bose--Einstein condensate systems. The history of this area will be discussed, highlighting both theoretical and experimental observations. With the fundamentals of Bose--Einstein condensation explained, I will then examine the dynamical behaviour of condensates from the perspective of superfluidity. Vortex behaviour will be discussed, along with recent results, concentrating on the dynamics of vortices in response to external potentials.

 Following this, quantum chaos will be introduced, along with the many of the interesting dynamics that can be observed in superfluid flow. The creation and observation of quantum chaotic dynamics in a Bose--Einstein condensate shall be proposed. The goal of the proposed project will be to generate chaotic dynamics using a vortex lattice subjected to a periodically pulsed optical lattice. Beginning with a well defined state, generation of chaotic behaviour shall be examined. The Hamiltonian required to generate a vortex lattice in a condensate shall be mapped to the delta-kicked harmonic oscillator Hamiltonian, which will provide a model system to understand the observed system dynamics. To characterise the resulting dynamics, the trajectories of the vortices shall be plotted over the course of the evolution of the system. Further information may be obtained by examining the system using phase--space methods and Floquet theory to characterise the observed behaviours. Experiments to observe chaotic behaviour in the quantum regime are rare, and the realisation of this type of system should allow for the first of its kind to observe chaos in a system of well-ordered topological excitations. Given recent experimental progress in the area of trapping, cooling, rotating and controlling Bose--Einstein condensates, the proposed system should be realisable with currently available experimental techniques.
