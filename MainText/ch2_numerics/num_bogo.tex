\section{Bogoliubov equations}
\label{sec:bogo}


Bogo tbc
\begin{equation}\label{eqn:bogo_psi}
\Psi_0(\mathbf{r},t) = \exp\left(-i\frac{\mu t}{\hbar}\right)[\phi_0(\mathbf{r}) + u(\mathbf{r})\exp\left(-i\omega t\right) + v^{*}(\mathbf{r})\exp\left(i\omega t\right) ]
\end{equation}



\begin{equation}\label{eqn:bogo_h0}
H_0 = -\frac{\hbar^2}{2m}\nabla^2 + V(\mathbf{r}) + 2g\vert \phi_0 \vert^2
\end{equation}

\begin{equation}\label{eqn:bogo_lhs}
    \partial_t \Psi(\mathbf{r},t) = \exp\left(\frac{-i\mu t}{\hbar}\right)\left[\mu\phi_0 + (\mu+\hbar\omega)u\exp\left(-it\omega\right) + (\mu+\hbar\omega)v^{*}\exp\left(it\omega\right) \right]
\end{equation}

\begin{align}\label{eqn:bogo_nonlin}
    g|\Psi|^2\Psi &= g \phi_0^{*}\phi_0\phi_0 \\
                &= g\exp\left(\frac{-i\mu t}{\hbar}\right)\left(\phi_0^{*} + u^{*}\exp(i\omega t) + v\exp(-i\omega t)\right)\left(\phi_0 + u\exp(-i\omega t) + v^{*}\exp(i\omega t)\right)^2 \\
                & \approx g\exp\left(\frac{-i\mu t}{\hbar}\right)\left[
                |\phi_0|^2\left(
                 \phi_0 + 2(u\exp\left(-i\omega t\right) + v^{*}\exp\left(i\omega t\right) )\right) + \phi_0^2\left( u^{*}\exp\left(i\omega t\right) + v\exp\left(-i\omega t\right)
                \right)
                \right]
\end{align}
Equating terms gives
\begin{subequations}\label{eqn:bogo_lhsrhs}
Bogoliubov-de Gennes equations:
\begin{align}
    \mu \phi_0 &= (H_0 - \Omega L + g |\phi_0|^2)\phi_0,\\
    (\mu +\hbar\omega)u &= (H_0 - \Omega L + 2g|\phi_0|^2)u + g\phi_0^2 v,\\
    (\mu +\hbar\omega)v^{*} &= (H_0 - \Omega L + 2g|\phi_0|^2)v^{*} + g\phi_0^2 u^{*}
\end{align}
\end{subequations}

Concentrating on the excitations, we can form a matrix as
\begin{equation}
    \begin{pmatrix}
        H_0 - \mu -\Omega L_z & g\phi_0^2 \\
        -g\phi_0^{*2} & -H_0 + \mu +\Omega L_z
    \end{pmatrix}
    \begin{pmatrix}
        u \\
        v
    \end{pmatrix}
    = \hbar\omega
    \begin{pmatrix}
        u \\
        v
    \end{pmatrix}
\end{equation}
where the eigenvalues of which can be used to determine the stability of the system. For energy eigenvalues with imaginary components, the system will be dynamically unstable.

\subsection{Numerical implementation}
To solve the Bogoliubov equations numerically requires searching for the solution to a generalised non-Hermitian eigenvalue problem. Thus, the system must be formally specified in matrix form. The derivative operators of $H_0$ are specified using second-order central differences, represented by
\begin{equation}
    \partial_i = \frac{U_{i+1,j} + U_{i-1,j} - 2U_{i,j}}{h^2}
\end{equation}
where $i$ is the respective dimension for the derivative, and $h$ is the step size. In matrix form, this becomes
\begin{equation}
    B =
    \begin{bmatrix}
            -2      &   1    &    0   &  \cdots   &  \cdots   &  \cdots   & 0 \\
            1       &   -2   &    1   &           &           &     &  \vdots \\
            0       &    1   &   -2   & \ddots    &           &     &  \vdots \\
            \vdots  &        & \ddots & \ddots    & {\ddots}  &     &  \vdots \\
            \vdots  &        &        & \ddots    &    -2     &  1  &       0 \\
            \vdots  &        &        &           &     1     & -2  &       1 \\
            0       & \cdots & \cdots & \cdots    &     0     &  1  &      -2
        \end{bmatrix}.
\end{equation}
where the number of rows and columns equal the number of elements along the dimension to be differentiated. To construct a multidimensional version, we take a Kronecker sum of the matrix $B$ along each respective dimension of the system. This can be represented as
\begin{equation}
    \mathbf{B}_{i,j} = \mathbf{B}_i \oplus \mathbf{B}_j = \mathbf{B}_i \otimes \mathbf{I}_j + \mathbf{I}_i \otimes \mathbf{B}_j
\end{equation}
were $i,j$ are the indices of the respective dimensions, and $\mathbf{I}$ is the identity equal in size to dimensions $i,j$. With this operation, we can obtain a block diagonal matrix of the form
