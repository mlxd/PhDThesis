\section{Fourier split-operator method}\label{sec:numerics}
For many of the works described previously, it is necessary to apply numerical techniques to obtain solutions. As the Gross--Pitaevskii
equation, given in Eq. \eqref{eqn:gpe}, is used in the majority of the literature cited, we will consider it as the basis for the following discussion. The Gross--Pitaevskii equation is a second order non-linear partial differential equation, and so very few exact solutions exist; the problem must often be tackled by a numerical approach. Though there are many ways to solve such a system numerically, (such as Crank-Nicholson, Trotter-Suzuki), the method I have chosen is the pseudospectral Fourier split-operator \cite{Num:Bauke_cpc_2011}.

If we consider a unitary evolution operator of the form
\begin{equation}\label{eqn:1}
\Psi(\mathbf{x},t+\tau) = \exp\left[ -\frac{iH\tau}{\hbar}\right]\Psi(\mathbf{x},t),
\end{equation}
where $H$ is the Hamiltonian, composed of momentum, potential, non-linear interaction, and rotation terms defined in Eq. \eqref{eqn:gpe}, we can solve for the wavefunction, and its resulting dynamics over a specified time-scale. However, due to error propagation resulting from numerical integration, it is instructive to employ methods that allow for the highest precision while providing results in useful timescales. To allow for this, care must be taken udring the implementation of such integration methods.  If we take the Hamiltonian, $H$, in terms of its components as a combination of position and momentum space functions we obtain
\begin{equation}\label{eqn:2}
{H} = {H}_{\textbf{r}} + {H}_{\textbf{k}} + {H}_{\textbf{L}},
\end{equation}
where we first neglect the angular momentum operator, ${H}_{\textbf{L}}$, and consider only the two other non-commuting parts ${H}_{\textbf{r}}$, containing the position operator, and ${H}_{\textbf{k}}$, containing the momentum operator. This way we can reduce the error in the numerical integration scheme by using 2nd order Strang-splitting as
\begin{equation}\label{eqn:3}
\exp\left[ -\frac{ i\left(\hat{H}_{\textbf{r}} + \hat{H}_{\textbf{k}}\right)\tau}{\hbar} \right] = \exp\left[- \frac{i\hat{H}_{\textbf{r}}\tau}{2\hbar} \right]\exp\left[-\frac{i\hat{H}_{\textbf{k}}\tau}{\hbar}\right]\exp\left[ -\frac{i\hat{H}_{\textbf{r}}\tau}{2\hbar}\right] + \mathcal{O}\left(\tau^2\right).
\end{equation}
The respective functions can be mapped to the Gross--Pitaevskii equations position, and momentum terms as
\begin{equation}
\hat{H}_{\textbf{r}} = V(\bar{x}) + g\vert\Psi(\bar{x},t)\vert^2\; \hspace{5em} \hat{H}_{\textbf{k}} = \frac{-\hbar^2}{2m}\nabla^2.
\end{equation}
%\hat{H}_{\textbf{L}} = \Omega L,
Following Bauke \textit{et al}. \cite{Num:Bauke_cpc_2011}, we can numerically solve this differential equation as
\begin{equation}
\Psi\left(\textbf{r},t+\tau\right) = \left[\hat{U}_{\mathbf{r}}\left(t+\frac{\tau}{2}\right) \mathscr{F}^{-1} \left[ \hat{U}_{\mathbf{k}}(t+\tau) \mathscr{F} \left[ \hat{U}_{\mathbf{r}}\left(t+\frac{\tau}{2}\right) \Psi\left(\mathbf{r},t\right) \right] \right] \right]  \\ + \mathcal{O}\left(\tau^3\right),
\end{equation}
where $\hat{U}_{r}=e^{-i\hat{H}_{r}(t)\tau/\hbar}$ is the time evolution operator in real space, $\hat{U}_{k}=e^{-i\hat{H}_{p}(t)\tau/\hbar}$ in momentum space,  $\mathscr{F}$ and $\mathscr{F}^{-1}$ are the forward and inverse Fourier transform respectively.

\begin{figure}
    \centering
    \includegraphics[]{./ch2_numerics/splitop}
    \caption{A single pass through the Fourier split-operator method.}
    \label{fig:num_splitop}
\end{figure}

%%% TO BE MOVED IN WITH PHASE ENGINEERING
Following \cite{BK:Pitaevskii_Stringari_2003} and taking the Madelung transform of the wavefunction given by Eq. \eqref{eqn:madelung}, the phase of the condensate may be specified as
\begin{equation}
\theta = \theta_c + \theta_i,
\end{equation}
where $\theta_c$ is the unperturbed condensate phase, and $\theta_i$ is the phase pattern to be imprinted. Thus, upon solving for the initial condensate phase, an additional phase pattern can be imprinted at any time by multiplying the wavefunction by the intended phase pattern. This is in line with the phase imprinting method, as previously introduced by Dobrek \textit{et al}. \cite{Vtx:Dobrek_pra_1999}. The underlying theory of the Fourier split-operator method for the Gross--Pitaevskii equation is given by Javanainen \textit{et al}. \cite{BEC:Javanainen_jphysa_2006}, showing how the choice of non-linearity and operator splitting affects the outcome of the method. The authors arrive at the conclusion that treating the non-linearity and potential terms together with the most current wavefunction definition yields results with an error magnitude that matches those obtained in the Schr\"{o}dinger Fourier split-operator case, indicating its applicability to this type of problem.
%%% TO BE MOVED IN WITH PHASE ENGINEERING

 \subsection{Time evolution}

We can write the wavefunction of a quantum system as a linear superposition of states as
\begin{equation}
     |\Psi \rangle = \displaystyle\sum\limits_{n} C_n |\Psi_n \rangle,
\end{equation}
where $| \Psi_n \rangle$ are a set of basis states for the system, with complex coefficients $C_n$. We next assume a unitary evolution operator of the form
\begin{equation}
    \mathscr{U}(t,t_0) = \exp\left(\frac{-i\mathcal{H}(t-t_0)}{\hbar}\right),
\end{equation}
where $\mathcal{H}$ is the Hamiltonian of the system. To time evolve our system from some time $t_0$ to a final time $t$ we apply the evolution operator to the wavefunction, giving
\begin{equation}
    \mathscr{U}(t,t_0)|\Psi \rangle = \displaystyle\sum\limits_{n} C_n \exp\left(\frac{-i{E_n}(t-t_0)}{\hbar}\right)|\Psi_n \rangle,
\end{equation}
where I have replaced the Hamiltonian operator with the energy eigenvalue of the $n$-th state. It follows from here that each state evolves at a different rate, proportional to its given eigenenergy. Higher energy states will oscillate faster than those of lower energy states. However, to make accurate predictions it is often necessary to deal with a single quantum state, such as the lowest lying state.

To create the initial state for the desired evolution the ground-state of the Hamiltonian must be determined as a first step. This can be achieved by evolving the system in imaginary time, causing all higher energy terms in an initial guess for the condensate wavefunction to decay to zero. This leaves the lowest energy state, which is the ground-state of the system. Taking the evolution operator, we apply a Wick rotation, rotating the time component through $\pi/2$ into the imaginary plane, as $t \rightarrow -it$. Applying this new evolution operator to the wavefunction gives
\begin{equation}
        \mathscr{U^{'}}(t,t_0)|\Psi \rangle = \displaystyle\sum\limits_{n} C_n \exp\left(\frac{-{E_n}(t-t_0)}{\hbar}\right)|\Psi_n \rangle.
\end{equation}
I have removed the complex term in the operator, which now takes the form of an exponentially decaying operator. When applied to the wavefunction all higher energy terms will decay at a rate faster than lower energy components. This process also causes a loss of probability density, and so the wavefunction must be renormalised after application. Through repeated application of this operator, and a renormalisation afterwards, the quantum system can converge to the groundstate. To begin, however, we must assume an intial guess for the wavefunction, which has some finite overlap with the lowest lying state.

As effective as this approach may be, the convergence to the lowest lying energy state becomes less effective as the computation approaches the expected value \cite{Vtx:Danaila_pra_2005}. Although many such methods exist, one that is well suited for this task is the previously discussed Fourier split-operator method. Due to the way the algorithm operates, it is essential to have a large and finely sampled grid in order to resolve both position and momentum of the wavefunction. A minimum grid-size on the order of $2^8$ in 2D for both $X$ and $Y$ dimensions is required, provided we are interested in trivial condensate dynamics.

Given that in the presence of large values of angular momentum, the condensate wavefunction will accomodate many vortices. To ensure a well ordered lattice, it is insufficient to numerically solve the GPE at the required rotation rate. Assuming an initial Gaussian guess, the large number of vortices will enter the condensate from the edge and compete for lattice sites to form the expected Abrikosov pattern. Thus, to overcome this issue, following the groundstate of the condensate with a ramp of the rotation rate is often necessary. This is essentially adiabatic evolution during imaginary time, and for all required rotation rates of the condensate we get a vortex lattice ground-state.

An implementation of this method is a straight-forward process using MATLAB, and has been performed for the purpose of this study. However, due to the large computational overhead required to time-evolve such a system, the procedure takes a long time to simulate the system at the required degree of precision. Therefore, it is necessary to further develop the methods used, and to improve the implementation of this algorithm to leverage the recent advances in computational acceleration.

\section{Angular momentum operators using FSO method}
The Fourier split-operator method described earlier works well in handing cases where the operators live in position or momentum space respectively. However, the angular momentum operators are essentially a combination of both spaces as we deal with each basis respectively. Take, for example, the angular momentum operator along the $z$-axis, given by $L_z = xp_y - yp_x$. To apply $L_z$ it is essential that the wavefunction along each direction be in the correct space, given the mixed dependencies. Thus, to apply this operator we must Fourier transform along a single dimension, multiply by the $k$-space operator, take the inverse, multiply by the $r$-space operator, and then perform this operation along the other dimension, summing the results.

 This mixed phase approach accrues an error not encountered using methods solely in position or momentum space. The error can be determined by checking the commutativity of the respective components of the angular momentum operator as

 \begin{subequations}
 \begin{align}
 	L_1 = [x p_y,-y p_x] &= [x p_y,-y] p_x  -  y[x p_y,p_x], \\
 				   &= -[-y,x p_y] p_x + y [p_x, x p_y], \\
 				   &= -\left( {\cancelto{0}{[-y,x]}} p_y + x [-y,p_y] \right) p_x + y \left( [p_x,x] p_y + x {\cancelto{0}{[p_x,p_y]}} \right), \\
 				   &= -x {\cancelto{i\hbar}{[-y, p_y]}} p_x + y {\cancelto{-i\hbar}{[p_x,x]}} p_y, \\
 				   &= -i\hbar \left(x p_x + y p_y \right).
 \end{align}
\end{subequations}

 The complex error term can be seen as, in the case of the implemented evolution, allowing the angular momentum operator to change from imaginary time to real-time, and vice-versa in each respective case. To overcome this, we simply swap the application order of the operator components, between even and odd steps during the evolution. Starting with the alternate order we obtain a value of $L_2 = [-y p_x, x p_y] = i\hbar \left(x p_x + y p_y \right)$. Since we are applying this phase to the condensate we can overcome the error of one term by the application of the other, as
 \begin{equation}
 \exp{i L_1}\exp{i L_2} = 1.
 \end{equation}

 Although alternating will provide a cancellation of this error, it can be assumed that for large timesteps the error will have a non-insignificant contribution to the overall dynamics, as the wavefunction evolves from timestep to timestep. Thus, for this method to remain accurate we can perform the previous decomposition for third-order, or use as is for a second-order accurate scheme.

\subsection{Resolution considerations}
As the Fourier split-operator method does not take into account resolution in strictly position space, but momentum space, care must be taken to chose numerical grids. The reciprocal relationship between position and momentum space is known as \begin{equation}
    k_{\text{max}} = \frac{2\pi}{x_{\text{min}}},
\end{equation}
which follows an uncertainty relation; better resolution in one space, yields worse in the other. Thus, to allow for a condensate to be simulated effectively in both spaces, it must fit within the grid on which it is defined, and resolve to half the size of the smallest structure. It is easy to estimate a radius for the position space wavefunction, following the Thomas-Fermi limit. It is also rather easy to know that for a non-rotating condensate the wavefunction should occupy the lowest lying mode ($\mathbf{k}=0$), and those close to it, assuming a harmonic trap. Rotating the condensate, however, has the effect of expanding the wavefunction in position space due to centrifugal forces. Additionally, the momentum space wavefunction expands and occupies a larger space. With the addition of vortices to the system, there are now small scale structures to resolve, including the resolution of the wavefunction phase. Complex plane branch-cuts must be resolvable to ensure the correct behaviour of vortices. This leads to a difficult-to-simulate system; we have a growing position space, as well as momentum space, wavefunction simultaneously.

One such way of ensuring we can accurately resolve the system is to define a sufficient smallest length-scale on one such grid (say position). With this, we can maintain a constant resolution in position space, whilst furthering momentum space resolution by increasing the number of samples of our grid. By ensuring the position grid remains defined with the same lowest increment, it is possible to increase resolution in the reciprocal space. Computationally, this is costly, but is quite effective when using compute accelerators (GPUs), which will be discussed later. %For the purpose of the work carried out herein, unless otherwise specified the simulations were resolved on a grid of $2^{10}\times 2^{10}$ elements, with spatial extent of the condensate $R\approx 700~\mu$m, and reciprocal space extent $K \approx 5\times10^{8}$ m$^{-1}$.

 \section{Vortex tracking}
 To efficiently follow the vortex dynamics, some robust algorithm is needed to track their positions. One could track regions where the density drops to zero. However, this gives very little information on the topological excitation, and may mistake density dips such as phonos for the presence of such excitations. One of the most effective ways is to locate the $\pm 2\pi$ charge in the wavefunction phase, which is a signature of quantum vortices. We can assume that around a $2\times 2$ subgrid, the phase rotates from $-\pi$ to $+\pi$ in the presence of a vortex located on the subgrid. After an initial pass to determine the vortex locations closest the nearest grid element, a least-squares fit is performed to more accurately determine the vortex core position. With this, we can accurately the determine the motion of the vortices with high precision.

 To track the vortices during the evolution, the creation of an initial list of vortices is performed, with each given a unique identifier. Assuming the vortex cores can travel a limited distance (some multiple of the grid resolution) between time steps, we can say at subsequent times which vortex has moved to the newly found positions.

 This is performed through representing vortices as a graph, each with an assigned unique identifier, associated location, phase winding and on/off flag. Edges are created between vortices that are separated by at most double the healing length (?lattice constant). A finite boundary is chosen to examine only vortices at the center, which can cause vortices to appear and disappear on the boundary. Thus, any vortex which appears without association to an initial vortex, or any vortex that leaves the boundary, is switched off and remains so for all analysis.
