\section{Ultracold atoms}\label{sec:coldatoms}
\subsection{Cooling of atomic gases}\label{sub:cooling}
One of the major advances in experimental physics towards creating matter in extreme situations has been the laser cooling of trapped atoms to temperatures near absolute zero. This feat resulted from the pioneering work of C. Cohen-Tannoudji, S. Chu and W.
Phillips, and earned them the Nobel Prize in Physics in 1997 \cite{AO:Chu_revmod_1998,AO:Cohen_revmod_1998,AO:Phillips_revmod_1998}.
It relies on the use of counter-propagating detuned laser fields which are shown at a trapped cloud of atoms. Due to Doppler shifting of the frequencies, atoms moving towards the respective beams see resonant photons, absorb them and slow down due to the momentum absorbed, with this technique being termed ``Doppler cooling''. This is followed by a spontaneous emission in a random direction, for which the recoil kicks average out to zero; hence, the atoms become cooler. As a result, the atoms eventually reach a velocity below that which can be absorbed from the lasers due to the change in resonance frequencies, leaving a narrower velocity distribution. Although laser cooling allowed temperatures to reach micro-Kelvin regimes, additional techniques must be used to obtain atoms deep in the nano-Kelvin temperature range. A further discussion of these methods is presented in [Atomic physics, Foot; Laser cooling and trapping, Metcalf].
I will assume the effectiveness of these techniques to allow the creation of Bose--Einstein condensates in dilute atomic gases.

\subsection{Introduction to Bose--Einstein condensation}\label{sub:becintro}
Bose--Einstein condensation was predicted by N. Bose and A. Einstein \cite{BEC:Ketterle_revmod_2002}. This began as a work on the statistical behaviour of photons, eventually leading to the derivation of the ``Bose--Einstein distribution'' for an ideal bosonic gas.
In the framework of the grand canonical ensemble and following the description given by Pitaevskii and Stringari \cite[chap. 2]{BK:Pitaevskii_Stringari_2003}, the Bose--Einstein distribution is given by
\begin{equation}
\bar{n}_i = \frac{1}{e^{\beta(\epsilon_i - \mu)} -1},
\end{equation}
where $\bar{n}_i$ is the average occupation number of the $i$-th energy state, $\beta=(k_BT)^{-1}$ with $k_B$ being the Boltzmann constant and $T$ as the temperature, $\epsilon_i$ is the $i$-th energy eigenvalue, and $\mu$ is the chemical potential giving the energy required to add an atom to the system.
The total number of particles in the system, $N$, can be evaluated by summing over the individual occupation numbers, as
\begin{equation}
N=\displaystyle\sum_i \bar{n}_i.
\end{equation}
This work predicted that non-interacting, indistinguishable (bosonic) particles would, if their energy was sufficiently low, undergo a phase transition below a critical temperature into a new phase in which the particles would occupy the same lowest lying energy state of the system. The number of atoms occupying this lowest lying state, $N_0$, is given by
\begin{equation}
N_0 \equiv \bar{n}_0 = \frac{1}{e^{\beta(\epsilon_0 - \mu)} - 1},
\end{equation}
with $\epsilon_0$ representing the lowest energy eigenvalue. Since negative occupation numbers would be a nonphysical result, the chemical potential is limited to values of $\mu < \epsilon_0$. As $\mu$ tends to $\epsilon_0$ the occupation of the lowest energy state grows large. Separating the total number of atoms into the lowest lying (condensed), $N_0$, and higher lying (thermal), $N_T$, states as
\begin{equation}
N = N_0 + N_T = N_0 + \displaystyle\sum_{i\neq 0}\bar{n}_i(T,\mu),
\end{equation}
a relation for the the onset of Bose--Einstein condensation can be given. For a finite temperature system ($T>0$) this will happen when the temperature drops below a critical value, $T_c$, where $\mu$ will approach $\epsilon_0$, resulting in the macroscopic occupation of the lowest lying state, yielding a Bose--Einstein condensate (BEC).

BECs of ultracold atomic gases are a prime example of coherent matter waves, and demonstrate quantum mechanical effects on length-scales which can be considered mesoscopic. Given a cloud of identical bosons at higher temperature, the atoms exhibit billiard-ball type behaviour. As they are cooled, their thermal de Broglie wavelength increases as
\begin{equation}
\lambda_{dB} = \sqrt{\frac{\hbar^2}{2\pi mk_{B}T}},
\end{equation}
where $m$ is the atom mass, because the wave-nature of the atoms becomes more prominent, and the de Broglie waves start to overlap when $\lambda_{dB} \approx$ inter-particle separation. The quantity $n\lambda_{dB}^3$, where $n$ is the density of the gas, is known as the \emph{phase-space density}. Physically this denotes the number of particles present in a box with sides of $\lambda_{dB}$ length, and the phase transition to the condensate sets in at $\zeta\left(\frac{3}{2}\right)$, where $\zeta$ is the Riemann-Zeta function \cite{BK:Ueda_2010}. The value of $T_c$ for this to occur in a uniform gas is given by
\begin{equation}
k_BT_c = \frac{2\pi}{\zeta\left(\frac{3}{2}\right)^{2/3}}\frac{\hbar^2n^{2/3}}{m}.
\end{equation}
For a finite number of atoms in a realistic experimental scenario one can consider the case of a BEC trapped in a harmonic oscillator potential. For this finite-sized sample, and inhomogeneous density profile, the transition temperature is given by
\begin{equation}
k_BT_c = \frac{\hbar\bar{\omega}N^{1/3}}{\zeta(3)^{1/3}},
\end{equation}
where $\bar{\omega}=(\omega_x\omega_y\omega_z)^{1/3}$ is the geometric mean of the oscillator frequencies. Critical temperatures for harmonically trapped gases are on the order of nano-Kelvin assuming realisable trapping frequencies, which were experimentally inaccessible during the time of Bose and Einstein.

Fritz London, in 1938, following on from the body of work derived by Einstein and Bose drew the connection between superfluidity in liquid $^4$He and Bose--Einstein condensation \cite[Chap.~1]{BK:Pitaevskii_Stringari_2003}. However, due to the liquid nature of $^4$He at low temperatures only approximately $10\%$ of the atoms condense into a BEC \cite{BEC:Penrose_pr_1956}. In order to achieve the large occupation of the lowest energy state the system must be prepared so that the interparticle interaction strength does not destroy the coherence, as is the case for liquid $^4$He.

Due to their weak interactions, dilute atomic gases are closer to the ideal case discussed by Bose and Einstein. One negative impact, though, were the higher masses of the atoms used in such systems, which required reaching much lower transition temperatures. As such, the use of lighter elements were considered. Spin-polarised hydrogen
was one of the first systems to be investigated to create BEC \cite{BEC:Kleppner_enfe_1998,BK:Pitaevskii_Stringari_2003} in the 1970's. Using trapping and cooling techniques
present at the time, this type of system came close, but did not quite reach the required temperatures and phase-space densities for
Bose--Einstein condensation to occur until over twenty years later \cite{BEC:Fried_prl_1998}. Given the advent of laser cooling in the 1980's, the use of alkali atoms was considered partly due to the ease of accessibility of their optical transition frequencies. It was not until 1995 that the first BECs would be experimentally realised \cite{BEC:Cornell_science_1995,BEC:Ketterle_prl_1995}.

\subsection{Theoretical description of BECs: Gross--Pitaevskii equation}\label{sub:gpederiv}
I will, in the following section, outline the derivation of the mean-field Gross--Pitaevskii equation, which is widely used to study the behaviour of the condensate in many works cited in this review. Following the \textit{Les Houches 2013 Lecture Course} by J. Walraven \cite{LEC:Walraven_lh_2013} the second quantization form of the many-body Hamiltonian for interacting particles in an external potential is given by
\begin{equation}\label{eqn:ham2ndq}
\hat{\mathcal{H}} = \hat{H_1} + \hat{H_2} = \int \hat{\Psi}^{\dagger} H_0\left(\textbf{r},\textbf{p} \right)  \hat{\Psi} \; d\textbf{r}  + \frac{1}{2} \int\hat{\Psi}^{\dagger}\hat{\Psi}^{\dagger}V_{\textrm{int}}(\textbf{r}^\prime-\textbf{r})\hat{\Psi}\hat{\Psi} \; d\textbf{r}^\prime d\textbf{r},
\end{equation}
with $H_0\left(\textbf{r}, \textbf{p} \right) = -\frac{\hbar}{2m}\nabla^2 + V_{\textrm{ext}}\left(\textbf{r}\right)$. The external potential, $V_{\text{ext}}(r)$ is taken as harmonic, of the form
\begin{equation}
V_{\text{ext}}(\textbf{r}) = \frac{m}{2}\displaystyle\sum_{i}{\omega_i r_i},
\end{equation}
and with $\omega_i$ representing the trapping frequency in the $i$-th spatial dimension. The interaction potential, $V_{\text{int}}$ is assumed to be point-like as
\begin{equation}\label{eqn:v_int}
	V_{\text{int}}\left(\textbf{r}_i,\textbf{r}_j \right) = g\delta\left(\textbf{r}_i - \textbf{r}_j\right),
\end{equation} where $\delta$ is the Dirac delta function and the mean-field interaction, $g$ is given by
\begin{equation}
	g = \frac{4\pi\hbar^2 a_s}{m},
\end{equation}
with $a_s$ being the s-wave scattering length. Inserting the contact potential Eq. \eqref{eqn:v_int} into the second quantised interaction Hamiltonian $\hat{H}_2$ from Eq. \eqref{eqn:ham2ndq} above yields the following relations
\begin{subequations}
\begin{align}
\hat{H}_2 &= \frac{g}{2} \int \hat{\Psi}^{\dagger}(\textbf{r})\hat{\Psi}^{\dagger}(\textbf{r\textprime}) \delta\left(\textbf{r} - \textbf{r\textprime}\right)\hat{\Psi}(\textbf{r\textprime})\hat{\Psi}(\textbf{r})d\textbf{r}d\textbf{r\textprime} \\
 & = \frac{g}{2}\int \hat{\Psi}^{\dagger}(\textbf{r})\hat{\Psi}^{\dagger}(\textbf{r}) \hat{\Psi}(\textbf{r})\hat{\Psi}(\textbf{r})d\textbf{r}d\textbf{r} \\
 & = \frac{g}{2}\int \hat{\Psi}^{\dagger}\left(\textbf{r}\right)\hat{n}\left(\textbf{\textbf{r}}\right)\hat{\Psi}^{\dagger}d\textbf{r}.
\end{align}
\end{subequations}

In the Heisenberg picture, the evolution of the system is governed by the equation
\begin{equation}\label{eqn:heisenberg}
i\hbar \frac{\partial}{\partial t}\hat{\Psi}_H\left(\textbf{r}, t\right) = \left[\hat{\Psi}_{H}\left(\textbf{r}, t\right), \hat{\mathcal{H}}  \right],
\end{equation}
where the Heisenberg field annihilation operator, $\hat{\Psi}_H\left(\textbf{r}, t\right)$, is given by
\begin{equation}\label{eqn:psi_heisenberg}
\hat{\Psi}_H\left(\textbf{r}, {t} \right) = e^{\frac{i\hat{\mathcal{H}}t}{\hbar}}\hat{\Psi}\left(\textbf{r}\right) e^{\frac{-i\hat{\mathcal{H}}t}{\hbar}}.
\end{equation}
The operator, $\hat{\Psi}_H$ can be interpreted as the one removing an atom from a given state of the system. Therefore, if all $N$ atoms in the system are in the ground-state, $\vert 0_N\rangle$, as would be the case in an ideal condensate, the following relationship holds
\begin{subequations}
\begin{align}
\hat{\Psi}_{H}(\textbf{r},t)\vert 0_N \rangle &= e^{\frac{iE_0(N-1)t}{\hbar}}\hat{\Psi}(\textbf{r})e^{\frac{-iE_0(N)t}{\hbar}}\vert 0_N \rangle \\
&= \hat{\Psi}(\textbf{r})e^{\frac{i[E_0(N-1) - E_0(N)]t}{\hbar}} \vert 0_N \rangle \\
&= \hat{\Psi}(\textbf{r})e^{\frac{-i\mu t}{\hbar}} \vert 0_N \rangle,\label{eqn:psi_dagger_time} \label{eqn:stationary_soln}
\end{align}
\end{subequations}

where $\mu=E_0(N) - E_0(N-1)$ is the chemical potential. Using the bosonic commutation relations
\begin{eqnarray}
\left[\hat{\Psi}(\textbf{r}'), \hat{\Psi}^{\dagger}(\textbf{r})\right] &=& \delta(\textbf{r}' - \textbf{r}), \\
\left[\hat{\Psi}(\textbf{r}'), \hat{\Psi}(\textbf{r})\right] &=& \left[\hat{\Psi}^{\dagger}(\textbf{r}'), \hat{\Psi}^{\dagger}(\textbf{r})\right] = 0,
\end{eqnarray}
and noting the following relations
\begin{align}
\left[\hat{\Psi}(\textbf{r}),\hat{H}_1 \right] & = \hat{H}_0(\textbf{r},\textbf{p})\hat{\Psi}(\textbf{r}), \\
\left[\hat{\Psi}(\textbf{r}),\hat{H}_2 \right] & = g\hat{n}(\textbf{r})\hat{\Psi}(\textbf{r}), \\
\left[\hat{\Psi}(\textbf{r}),\hat{N} \right] & = \hat{\Psi}(\textbf{r}) ,
\end{align}
upon substitution of Eq.~\ref{eqn:psi_dagger_time} into Eq.~\ref{eqn:heisenberg}, it be rewritten as
\begin{equation}
    i \hbar \partial_t \left( \hat{\Psi}(\textbf{r}) e^{-\frac{i\mu t}{\hbar}} \right) = H \hat{\Psi}(\textbf{r}) e^{-\frac{i\mu t}{\hbar}}
\end{equation}
with
%\begin{equation}\label{eqn:h_many}
%\mathcal{H} = \displaystyle\sum\limits_{i=1}^N \left( -\frac{\hbar^2}{2m}\nabla^2  + V(\textbf{r}_i)\right) + \frac{1}{2}g\displaystyle\sum\limits_{i\neq j}\delta(\textbf{r}_i - \textbf{r}_j).
%\end{equation}
\begin{equation}\label{eqn:h_many}
H =  -\frac{\hbar^2}{2m}\nabla^2  + V(\textbf{r}) + g\hat{n}(\textbf{r})\right).
\end{equation}
Due to the macroscopic occupation of the lowest lying single-particle state, $\hat{\Psi}$ can be treated as the sum of a classical field and a quantum fluctuation term, as
\begin{equation}\label{eqn:gpe_fluc}
    \hat{\Psi} = \Psi + \delta\hat{\Psi}.
\end{equation}
As the quantum fluctuation component is small in the zero-temperature limit, we can safely neglect these, and use the classical field as the wavefunction describing the system. The time dependent mean-field Gross--Pitaevskii equation (GPE) can then be written as
\begin{equation}\label{eqn:gpe}
i\hbar\frac{\partial}{\partial t}\Psi(\textbf{r},t) = \left[-\frac{\hbar^2}{2m}\nabla^2 + V(\textbf{r}) + g\vert\Psi(\textbf{r},t)\vert^2 \right]\Psi(\textbf{r},t).
\end{equation}
The time independent form can be found by substituting Eq. \eqref{eqn:stationary_soln} into \eqref{eqn:gpe}, yielding
\begin{equation}
\mu\Psi(\textbf{r}) = \left[-\frac{\hbar^2}{2m}\nabla^2 + V(\textbf{r}) + g\vert\Psi(\textbf{r},t)\vert^2 \right]\Psi(\textbf{r}),
\end{equation}
where the wavefunction is normalised to the particle number, $N$, as follows
\begin{equation}\label{eqn:norm}
\displaystyle\int\limits_{-\infty}^{\infty}d\textbf{r} \left\vert \Psi\left(\textbf{r}\right) \right\vert^2 = N.
\end{equation}
In the case of a rotating condensate an additional term appears in the GPE, $-\Omega L$, where $\Omega$ is the angular rotation frequency, and $L$ is the angular momentum. Assuming rotation about a single axis, the longitudinal direction, $z$, $L$ can be replaced with $L_z$, giving the final form of the GPE in the co-rotating frame as
\begin{equation}\label{eqn:gpe_rotation}
i\hbar\frac{\partial}{\partial t}\Psi(\textbf{r},t) = \left[-\frac{\hbar^2}{2m}\nabla^2 + V(\textbf{r}) + g\vert\Psi(\textbf{r},t)\vert^2 - \Omega L_z  \right]\Psi(\textbf{r},t).
\end{equation}
This mean-field equation can be used to describe the behaviour of the condensate, provided the number of atoms is sufficiently large ($>10^3$ for most experimental set-ups) and the correlations are not too strong. Assuming a large number of bosons in the condensate, $N$, the interaction term dominates over the kinetic term of the Hamiltonian which can therefore be neglected. This turns finding the ground-state of the system into a solvable, algebraic problem, and is known as the Thomas-Fermi approximation \cite[~p. 84]{BK:Ueda_2010}. The Hamiltonian can thus be reduced to a combination of the trapping potential and the mean-field interaction, and the wavefunction, $\psi_{\textrm{TF}}$, can be determined from the time independent GPE as
\begin{equation}
\psi_{TF}(\textbf{r}) = \sqrt{ g^{-1}[\mu - V(\textbf{r})] \Theta(\mu - V(\textbf{r}))},
\end{equation}
where $\mu$ is the chemical potential, and $\Theta$ is the Heaviside step function, which ensures that the condensate density does not become negative. Within this regime the condensate can be described by hydrodynamical equations \cite[Pg.~180]{BK:Pitaevskii_Stringari_2003}, and from this, an approximate value for the cloud radius can be determined, called the Thomas-Fermi radius. The boundary of the cloud is determined by the surface at which the density becomes zero, and corresponds to the point where the trapping potential and chemical potential are equivalent. This gives the geometric mean spatial extent of the cloud \cite[~p. 169]{BK:Pethick_Smith_2008} as
\begin{equation}
\bar{R} = \left(\frac{15Na}{\bar{a}}\right)^{1/5}\bar{a},
\end{equation}
where the characteristic length of the harmonic oscillator is given by
$\bar{a} = \sqrt{{\hbar}/{m\bar{\omega}}}$. Thus, within the Thomas-Fermi radius, it is possible to explain the behaviour of the BEC analytically and compare with exact numerical results from integrating the full GPE.

\subsection{Bogoliubov-de Gennes equations}
\label{sec:bogo}
While the Gross--Pitaevskii equation captures the rich array of dynamics exhibited by a condensate system, it is often instructive to examine the stability of its solutions. Where earlier I neglected the small quantum fluctuations (see eq. \ref{eqn:gpe_fluc}), we can instead reformulate this term as a combination of counterpropagating waves with the stationary state solution as
\begin{equation}
    \delta\hat{\Psi} = \exp\left(\frac{-i\mu t}{\hbar}\right)[\Psi_0 + u\exp(-it\omega) + v^{*}\exp(it\omega)].
\end{equation}
Taking this expression where $\mu$ is the chemical potential of the condensate, the wavefunction in the groundstate can be defined as
\begin{equation}\label{eqn:bogo_psi}
\Psi_0(\mathbf{r},t) = \exp\left(-i\frac{\mu t}{\hbar}\right)[\phi_0(\mathbf{r}) + u(\mathbf{r})\exp\left(-i\omega t\right) + v^{*}(\mathbf{r})\exp\left(i\omega t\right) ].
\end{equation}

Firstly, we calculate the time-derivative of eq. \ref{eqn:bogo_psi}, which after simplificant becomes
\begin{equation}\label{eqn:bogo_lhs}
    \partial_t \Psi(\mathbf{r},t) = \exp\left(\frac{-i\mu t}{\hbar}\right)\left[\mu\phi_0 + (\mu+\hbar\omega)u\exp\left(-it\omega\right) + (\mu+\hbar\omega)v^{*}\exp\left(it\omega\right) \right].
\end{equation}

Next, the nonlinear interaction term is given as
\begin{subequations}
\begin{align}\label{eqn:bogo_nonlin}
    g|\Psi|^2\Psi &= g \phi_0^{*}\phi_0\phi_0 \\
                &= g\exp\left(\frac{-i\mu t}{\hbar}\right)\left(\phi_0^{*} + u^{*}\exp(i\omega t) + v\exp(-i\omega t)\right)\left(\phi_0 + u\exp(-i\omega t) + v^{*}\exp(i\omega t)\right)^2, \\
                & \approx g\exp\left(\frac{-i\mu t}{\hbar}\right)\left[
                |\phi_0|^2\left(
                 \phi_0 + 2(u\exp\left(-i\omega t\right) + v^{*}\exp\left(i\omega t\right) )\right) + \phi_0^2\left( u^{*}\exp\left(i\omega t\right) + v\exp\left(-i\omega t\right)
                \right)
                \right].
\end{align}
\end{subequations}

Plugging these terms into the GPE, and linearising in terms of $u$ and $v$ yields the Bogoliubov-de Gennes equations,
\begin{subequations}\label{eqn:bogo_lhsrhs}
\begin{align}
    \mu \phi_0 &= (H_0 - \Omega L + g |\phi_0|^2)\phi_0,\\
    (\mu +\hbar\omega)u &= (H_0 - \Omega L + 2g|\phi_0|^2)u + g\phi_0^2 v,\\
    (\mu +\hbar\omega)v^{*} &= (H_0 - \Omega L + 2g|\phi_0|^2)v^{*} + g\phi_0^2 u^{*},
\end{align}
\end{subequations}
where
\begin{equation}\label{eqn:bogo_h0}
H_0 = -\frac{\hbar^2}{2m}\nabla^2 + V(\mathbf{r}) + 2g\vert \phi_0 \vert^2
\end{equation}
This can be written as a matrix as
\begin{equation}
    \begin{pmatrix}
        H_0 - \mu -\Omega L_z & g\phi_0^2 \\
        -g\phi_0^{*2} & -H_0 + \mu +\Omega L_z
    \end{pmatrix}
    \begin{pmatrix}
        u \\
        v
    \end{pmatrix}
    = \hbar\omega
    \begin{pmatrix}
        u \\
        v
    \end{pmatrix}
\end{equation}
where the eigenvalues of these modes can be used to determine the stability of the system. The norm of these values is given by $\int d\mathbf{r}(|u|^2 - |v|^2)$. If the norm is positive, with positive eigenvalues, we can say the system is stable. If the norm is negative, the system will have an energetic instability. This will cause the system to move towards the lowest energy state only in the presence of dissipation. If, however, the norm is 0 with imaginary eigenvalues, the modes of fluctuations are dynamically unstable. Due to the complex eigenvalues over time the mode will dominate exponentially and destroy the initial state of the condensate. %In a later section I  will use this approach later to examine the behaviour of the Bose--Einstein condensate.

\subsection{Realisation of Bose--Einstein condensation in dilute gases}\label{sub:becdev}
Following the first experimental realisations of a BEC in dilute atomic gases by E. Cornell and C. Wiemann at University of Colorado, Boulder \cite{BEC:Cornell_science_1995}, and W. Ketterle at M.I.T. \cite{BEC:Ketterle_prl_1995}, work was carried out by Baym and Pethick \cite{BEC:Baym_prl_1996}, who investigated the mean-field ground-state solution in the Thomas-Fermi limit of a magnetically trapped rubidium BEC, as was created in the Boulder experiment.
%Learning from the techniques used and developed for condensation of alkali atoms, the group of D.~Kleppner and T.~Greytak succeeded in developing a condensate of spin-polarised hydrogen \cite{Fried_prl_1998}, following a twenty-year experimental search.

In a first comprehensive review of the area only a few years later by Dalfovo \textit{et al}. \cite{BEC:Dalfovo_revmod_1999}, the underlying theory of trapped Bose--Einstein condensates was examined in detail. The review discusses how the combination of laser cooling and evaporative cooling \cite{AO:Ketterle_amop_1996} allowed the experimentalists to reach the temperatures and required phase-space densities for Bose--Einstein condensation to occur. Ground-state properties of the system are examined assuming an effective interaction. Stationary properties of harmonically trapped condensates are discussed, in the framework of mean-field theory (Gross--Pitaevskii), which is shown to allow for the determination of system properties and dynamics. Attractive versus repulsive interactions are examined, followed by a discussion of ``Beyond mean-field theory'', where corrections to the theory are applied to enable more exact determination of ground-state energies.

The nature of vortices in a condensate are considered, and in particular critical angular velocities, core size, and imaging. Given the coherence of atoms in a condensate, matter-wave interference by simply expanding and overlapping two condensates is suggested, and the appearance of a dislocation in the interference pattern is a sign of the presence of a vortex. The authors further indicate the accurate agreement of mean-field theory with known experimental data, and differences between observation and mean-field predictions are mentioned. %Due to the level of complexity involved in obtaining a condensate, a significant degree of work is performed on developing a solid theoretical understanding of dynamical behaviour under many differing conditions.

Interpenetrating condensates are discussed in a paper by Sinatra \textit{et al}. \cite{BEC:Sinatra_prl_1999}, where the authors present a theoretical prediction of the mixing of a two-component, initially separated condensate. Predicted oscillations of the mean condensate separation were compared to the Boulder (JILA) experiment, and showed good agreement, supporting the use of mean-field theory in predicting condensate dynamics.

\subsection{Lower dimensional condensates}\label{sub:coldatom_recent}
Another area of interest in BEC systems is that of low-dimensional condensates, such as the work undertaken by G\"{o}rlitz \textit{et al}. \cite{BEC:Gorlitz_prl_2001}, wherein a strong axial confinement of the
condensate in the $z$-dimension is used to effectively create a ``pancake'' condensate in an optical trap. This trapping geometry restricts motion in the axial direction, creating effective two-dimensional dynamics. This may be extended further to one-dimension by using elongated ``cigar'' traps via magnetic potentials, such as those existing on atom-chip structures \cite{AO:Denschlag_prl_1999,AO:Folman_prl_2000,AO:Haase_pra_2001}, or in optical potentials. G\"{o}rlitz \textit{et al.} state that in lower dimensional systems phenomena such as solitons and vortices should have much greater stability, and should therefore be more suitable for investigating them. Realisation of a Tonks-Girardeau gas is also discussed, with such systems becoming accessible using optical potentials such as described by Moritz \textit{et al}. \cite{OL:Moritz_prl_2003}. Work in the effective 2D regime is of particular interest, and condensate behaviour in relation to excitations has been examined under such conditions. Kimura \cite{BEC:Kimura_pra_2002} investigates the dynamical behaviour of condensate breathing modes in both two and one dimensions. Tanatar \textit{et al}. \cite{BEC:Tanatar_arxiv_2002} consider the use of a tightly-confined pancake condensate, and discuss differences occurring to the inter-atomic scattering as a function of confinement strength along the $z$-dimension. Both mentioned works consider use of the Thomas--Fermi approximation, comparing results obtained analytically with that of numerical integration using the Gross--Pitaevskii equation.

%In the following section, I will discuss the condensate behaviour from the perspective of a superfluid, as superfluidity is an inherent property of a gaseous BEC. This will be approached using a GPE formalism.
