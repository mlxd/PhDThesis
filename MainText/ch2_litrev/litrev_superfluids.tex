\section{Superfluidity}\label{sec:superfluid}
\subsection{Introduction to superfluidity}\label{sec:intro_super}

Superfluidity is a macroscopic quantum effect that is closely related to Bose--Einstein condensation. Liquid helium has been known for many years to exhibit superfluid behaviour \cite{BEC:Penrose_pr_1956}. One interesting property of superfluids is that they have quantized circulation which can lead to the appearance of quantum vortices under rotation. Traditionally, such excitations have been created in liquid $^4$He by a ``rotating bucket'' type of experiment, where the container holding the superfluid is rotated about a single axis. When the fluid is initially above the $\lambda$-critical point, the temperature at which $^4$He moves from a classical fluid to a superfluid, the atoms will undergo solid-body rotation with the container. On cooling the atoms below the $\lambda$-critical point, liquid $^4$He will undergo a transition to the superfluid state. If the velocity is above a critical rotation frequency, $\Omega_c$, vortices are nucleated in the rotating superfluid helium system. Due to the strongly interacting nature of liquid helium, the nucleated vortices are difficult to visualise as the healing length, $\xi$, of liquid $^4$He is only on the order of {\r{A}}ngstr{\"o}ms \cite{BEC:Srinivasen_pramana_2006}.

In order to visualise these vortices experimentally it was necessary to use an indirect means of visualisation in the form of tracer particles \cite{BEC:Packard_physb_1982}. Limited success was had with this technique using solid hydrogen, and later plastic microspheres, as they tended to join together due to static charges. An improved technique, using charged particles, showed much greater success \cite{Vtx:Packard_prl_1969}. Ions, or electrons, were trapped inside the vortex lines, and an electric field along the direction of the lines allowed for acceleration of the charges towards a luminescent screen where they could be observed. Current experimental work in vortex visualisation with liquid $^4$He has advanced significantly \cite{Vtx:Tsubota_arxiv_2010,Vtx:Guo_pnas_2014}, yet fine control over the behaviour of the liquid and the vortex dynamics remains difficult.

%\todo[inline]{Derive hydrodynamic formalism for GPE}
Superfluid behaviour is observed in liquid helium due partly to a portion of the atoms condensing into the ground-state. Given that BECs show superfluidity, the use of a dilute gas where the majority of atoms are in the condensate state provides a more controlled means to investigate superfluidity and quantised vortices \cite{BK:Ueda_2010,BEC:Srinivasen_pramana_2006,Vtx:Tsubota_arxiv_2010,CT:Tsubota_jpsj_2008}. In contrast with liquid $^4$He, which has a healing length of the order of {\r{A}}ngstr{\"o}ms, the healing length of a dilute gas of alkali atoms is on the order of microns \cite{Vtx:Isoshima_pra_1999}. This places condensates in a much more accessible regime for visualising vortices compared with liquid $^4$He experiments. For many condensates visualisation of the vortices is provided by absorption imaging, following a time-of-flight expansion of the cloud~\cite{Vtx:Raman_prl_2001,VTX:Rankonjac_pra_2016} to allow the vortex cores to expand and become easily more visible. One drawback of this method is that it is destructive, and requires successive experimental realisations to acquire any dynamical information. To examine the dynamics of the vortices, successive imaging techniques are required. Single-shot experiments have been demonstrated, where a small percentage of the condensate is repeatedly transferred to an untrapped atomic state. This allows for the untrapped atoms to expand, and vortices have been imaged this way~\cite{VTX:Freilich_sci_2010}. With this method the precession of the vortices in a trapped condensate were directly observed, and did not require successive realisations. More recently, \textit{in-situ} imaging of a condensate with a vortex lattice was demonstrated and allowed resolution of the cores without time-of-flight expansion with a minimally destructive optical refraction technique~\cite{VTX:Wilson_pra_2015}.

To fully understand the behaviour of these systems, it is necessary to have a framework for modeling the condensate, in the absence and the presence of vortices. As described above, a dilute gas of condensed atoms close to absolute zero temperature can be readily modelled by the Gross--Pitaevskii equation \eqref{eqn:gpe_rotation}, offering a direct means to examine condensate behaviour. It is, however, useful to obtain a hydrodynamic description for the condensate, performed by treating its wave-function as
\begin{equation}\label{eqn:madelung}
\Psi(\textbf{r},t) = \sqrt{\rho(\mathbf{r},t)} e^{i\theta(\textbf{r},t)},
\end{equation}
with
\begin{equation}\label{eqn:density}
\rho(\textbf{r},t) = \Psi^*(\textbf{r},t)\Psi(\textbf{r},t) = \vert \Psi (\textbf{r},t) \vert ^2.
\end{equation}
and where $\theta(\textbf{r},t)$ is the condensate phase \cite[~chap. 1]{BK:Pitaevskii_Stringari_2003}. Following the formalism given by Pitaevskii and Stringari \cite{BK:Pitaevskii_Stringari_2003} and multiplying \eqref{eqn:gpe} by $\Psi^{*}(\mathbf{r},t)$ then subtracting the conjugate of that expression allows one to obtain the continuity equation as
\begin{equation}\label{eqn:continuity}
\frac{\partial}{\partial t}\rho(\textbf{r},t)  + \nabla\cdot \textbf{j}(\textbf{r},t) = 0,
\end{equation}
where the current density of the condensate is
\begin{equation}\label{eqn:current_density}
\textbf{j}(\textbf{r},t) = \frac{-i\hbar}{2m}\left[\Psi^*(\textbf{r},t)\nabla\Psi(\textbf{r},t) - \Psi(\textbf{r},t)\nabla\Psi^*(\textbf{r},t)\right].
\end{equation}
Using ansatz \eqref{eqn:madelung} and substituting it into Eq. \eqref{eqn:current_density} then gives the form for the current density,
\begin{equation}
\textbf{j}(\textbf{r},t) = \vert\Psi(\textbf{r},t)\vert ^2\frac{\hbar}{m}\nabla\theta(\textbf{r},t).
\end{equation}
The velocity of the superfluid, $\textbf{v}(\textbf{r},t)$, is defined as the ratio of the current density to the density, which is then given by
\begin{equation}\label{eqn:velocity}
\textbf{v}(\textbf{r},t)\equiv \frac{\textbf{j}(\textbf{r},t)}{\rho(\textbf{r},t)} = \frac{\hbar}{m}\nabla\theta(\textbf{r},t).
\end{equation}
The gradient of the phase therefore determines the velocity of the condensate atoms; this indicates that the superfluid behaviour in a condensate is irrotational ($\nabla\times(\nabla\theta) =0$). Assuming a closed loop integral about a central point in the condensate, and recalling the single-valued nature of the wavefunction, yields the relationship
\begin{equation}\label{eqn:circulation}
\oint_C \textbf{v}\cdot d\textbf{l} = \frac{\hbar}{m}2\pi l.
\end{equation}
This shows the quantised nature of circulation in a superfluid, with $l$ representing the integer charge of the circulation. The phase winding around the central region is given by multiples of $2\pi$, with the centre of the phase becoming ill-defined. To circumvent this problem the density at this point drops to zero, signalling the presence of a vortex in the condensate density. This drop happens over the scale of the healing length, which, for repulsive interactions, is given by
\begin{equation}
\xi = \frac{1}{\sqrt{8\pi \rho a_s}},
\end{equation}
where $\rho$ is the bulk density of the condensate, and $a_s$ is the s-wave scattering length. For a cylindrically symmetric rotating condensate carrying a single vortex the wavefunction can be described in Madelung form as $\Psi = |\Psi|e^{\textrm{i}l\theta}$. The velocity profile at a distance $r$ away from the core can then be given as
\begin{equation}\label{eqn:1_over_r}
    v =\frac{\hbar l}{m r}.
\end{equation}

To sustain a vortex the condensate must have sufficient angular momentum, which is imparted via the angular rotation frequency times the angular momentum operator $-\Omega L_z$ in the Hamiltonian given by Eq. \eqref{eqn:gpe_rotation}. The rotation frequency, $\Omega$, of the condensate has an upper-bound stability limit equivalent to the transverse trapping frequency, $\omega_{\perp}$, of the harmonically trapped condensate.
%\todo[inline]{Find a better home for speed of sound}

\subsection{Vortices in Bose--Einstein condensates}\label{ss:vorticesinbec}

Given that the circulation of a vortex in a superfluid is quantised, one may assume that for faster rotation rates the circulation increases in integer multiples of $\hbar/m$, beyond critical rotation thresholds to excite each higher multiple. To understand the effect of vortex charge on the condensate, it is necessary to examine the energy functional, as given by
    \begin{equation}\label{eqn:functional}
        E(\Psi) = \int \Psi^{*} (H_{\text{GP}}) \Psi.
    \end{equation}
By assuming a wavefunction of the form $\Psi = |\Psi|\exp(il\theta)$, where $l$ is the vortex charge, and substituting this into Eq.~\eqref{eqn:functional} the energy functional is given as
\begin{equation}\label{eqn:functional_full}
    E(\Psi) = \int \frac{\hbar^2}{2m} \left(|\nabla\Psi|^2  + \frac{|\Psi|^2 l^2 m \mathbf{v}^2}{2}  \right) + V|\Psi|^2 + \frac{g}{2}|\Psi|^4 + \Omega \Psi^{*} L_z \Psi
\end{equation}
where $\mathbf{v}$ is the superfluid velocity, as given by \eqref{eqn:velocity}. The $l^2$ term shows that the energy scales quadratically with an increase in charge. This energy growth dependence shows that it will be more favourable to allow for two singly charged vortices, than a single doubly charged vortex, of which will decay due to the presence of complex eigenmodes \cite{VTX:Kawaguchi_pra_2004}. For rates of rotation $\Omega_c < \Omega < \omega_\perp$, where $\Omega_c$, is the threshold rotation rate to create a vortex, the condensate will favour many singly quantised vortices, rather than one or several multiply charged ones. For a superfluid condensate $\Omega_c = \frac{E_v}{L}$, where $E_v$ is the energy of the vortex, and $L$ is the angular momentum component of the superfluid \cite{BK:Pitaevskii_Stringari_2003}.

The stability of vortex states in a condensate is also a widely discussed topic \cite{Vtx:Fedichev_pra_1999,Vtx:Feder_prl_1999}, as non-rotating traps show an instability for small displacements of the vortex from the trap centre. Since the ``rotating bucket'' technique used to generate vortices in $^4$He cannot be used for gaseous BECs, many other techniques have been developed \cite{Vtx:Anglin_prl_1999,Vtx:Davies_prl_1999,Vtx:Marshall_pra_1999,Vtx:Dobrek_pra_1999,VTX:Nakahara_physb_2000}. For the work presented in this thesis, the method proposed by Dobrek \textit{et al}. \cite{Vtx:Dobrek_pra_1999}, is of particular interest, as it allows to optically generate vortices by the use of what they term a ``phase-imprinting'' method. The authors describe a scheme where the phase of the condensate is directly controlled in such a way that the required topological charge to induce a vortex during evolution is provided by external lasers. Through use of an absorption plate whose absorption coefficient depends on the axis angle, the condensate can be imprinted with the required phase pattern. This method will form the basis of work carried out in this thesis, and will be discussed in detail later in Sec.~\ref{sec:phase}.


\subsection{Vortex lattices}\label{sec:sec2_vtxlatt}

Although a large number of works exist which investigate systems with low numbers of vortices \cite{THS:Davies_2000,Vtx:Chevy_prl_2000,Vtx:Cooper_prl_2001,Vtx:Rosenbusch_prl_2002,Vtx:Ogawa_pra_2002,Vtx:Bretin_joptb_2003,Vtx:Madison_prl_2000,Vtx:Madison_jmo_2000,Vtx:Chevy_aoi_2001,Vtx:Madison_prl_2001,Vtx:Mottonen_jpcm_2002}, such systems do not necessarily form periodic vortex lattices. I will therefore concentrate primarily on studies of systems containing many vortices, assuming large values of $\Omega \lesssim \omega_\perp$. For such systems, the vortices are arranged in a periodic triangular Abrikosov lattice, reminiscent of that which appears in type-II superconductors with magnetic flux lines \cite{Vtx:AboShaeer_sci_2001}. Experimental setups have demonstrated upwards of over 130 vortices in a well ordered triangular formation. The resulting lattices are shown to be highly stable and are ideal setups for investigations of periodic systems \cite{Vtx:AboShaeer_sci_2001,Vtx:Engels_prl_2002}.

The Thomas--Fermi limit discussed earlier describes the case where the kinetic energy term of the Hamiltonian may be neglected in comparison to the interaction energy, as it offers little contribution to the condensate behaviour. This remains true for low rotation rates, however kinetic energy becomes important in the limit of fast rotation. In this case $\Omega/\omega_{\perp}\approx 1$, and the centrifugal force term, $m\Omega^2r$, almost balances with the trapping force term, $-m\omega^2r$. The condensate behaviour then closely resembles that of the two-dimensional quantum Hall regime, where the system is residing in the lowest Landau level, (LLL) ($n=0$) \cite{Vtx:Ho_prl_2001}.  As the rotation of the cloud approaches the trapping frequency, the system tends to the LLL, wherein the nonlinear interaction term becomes relatively weak due to the centrifugal forces on the atomic cloud and in this regime the Thomas--Fermi approximation becomes invalid, as the interaction term no longer dominates over the kinetic energy. The interaction energy is then much smaller than the gap between energy levels for single particle Landau levels. While there have been studies using harmonic-plus-quartic potentials to allow condensates to remain trapped beyond the harmonic trapping frequency limit \cite{BEC:Bretin_prl_2004,Vtx:Ghosh_pra_2004}, I will in this thesis consider only systems of harmonic confinement.

A rotation rate very close to the trap frequency will require treatment beyond that of mean-field methods, as the system enters a fractional quantum Hall (FQH) state~\cite{Vtx:Regnault_prl_2003}. However, a mean-field approach can be used in the ``mean-field quantum Hall'' (MFQH) regime, which exists just below such rotation rates \cite{BEC:Fetter_revmodphys_2009}. The differences between these two regimes are characterised purely by the ratio of the kinetic energy to the interaction energy of the system, and in the MFQH regime we can thus assume the LLL has been achieved \cite{Vtx:Zhai_pra_2004,BEC:Stock_laserphyslett_2005}. At these rotation rates, the aspect ratio of the condensate, namely the ratio of the width in the $z$-dimension to the width in the $x$--$y$ plane, becomes very small yielding an effective two-dimensional system. This also validates the earlier assumption for modeling of a quasi-2D condensate system.

In such a system, the vortices about the central region will be uniformly ordered due to the balanced velocity fields. Vortices on the condensate edge, however, will have a distorted alignment. The number of atoms compared to the number vortices, denoted as the ``filling factor'', $\nu=N/N_v$ \cite{BK:Ueda_2010,Vtx:Ho_prl_2001}, becomes an important characteristic of the system at high rotation rates. In the case where $1000 > \nu > 10$, the system may be accurately described to be in the MFQH regime, which is attained close to the $\Omega / \omega_{\perp}\approx 1$ limit. For values $\nu \leq 10$ the system is said to be strongly correlated~\cite{BEC:Fetter_revmodphys_2009}, and will require the use of a discrete boson model, which becomes next to impossible to simulate for a realistic number of atoms in a condensate. To avoid dealing with the FQH regime and restricting the system to the MFQH regime where mean-field theory is applicable, choosing rotational frequencies that guarantee this is important. For an almost perfectly regular lattice, the rotation rate must be sufficiently large so that the condensate width extends to large distances and a large number of vortices are generated, without entering the FQH regime~\cite{Vtx:Aftalion_pra_2005}.

The distribution of the vortices in the MFQH regime forms a triangular lattice pattern that is almost regular~\cite{Vtx:Schweikhard_prl_2004}. As discussed earlier, the condensate has an irrotational flow profile due to velocity being defined as Eq.~\eqref{eqn:velocity}. This relation holds true provided that $\theta$ is well defined, which is not the case at the centre of a vortex. As the phase at the vortex core is ill-defined, this singular region creates a non-zero curl. A generalisation of the irrotational flow condition (i.e. $\nabla\times \mathbf{v}=0$) to account for this can be given as as~\cite{BK:Pitaevskii_Stringari_2003,BK:Pethick_Smith_2008}
\begin{equation}
    \nabla\times \mathbf{v}=\frac{2\pi l\hbar}{m}\delta^{2}\mathbf{r}_\perp \hat{z},
\end{equation}
where $\delta^2$ is a two-dimensional Dirac delta function, $z$ is the unit vector along $z$ and $\mathbf{r}_\perp=(x,y)$. For large rotation frequencies this value becomes very similar to that of a solid-body rotation, as given by $\nabla \times \mathbf{v}=2\Omega$. This result arises for a large regular vortex lattice, where the areal density of vortices can be specified by the Feynman relation~\cite{BK:Pitaevskii_Stringari_2003}
\begin{equation}\label{eqn:feynman}
n_{v} = \frac{m\Omega}{\pi\hbar}.
\end{equation}
In the case of realistic condensates systems, where the densities are not uniform due to harmonic trapping, deviations exist from this value which have been calculated theoretically \cite{Vtx:Sheey_pra_2004_2,Vtx:Sheey_pra_2004} and observed experimentally \cite{VTX:Coddington_pra_2004}. However, for the values chosen in all later discussed simulations these deviations tend to be small, and can in most cases be neglected.

Some key details for the classification of the regime of the condensate are \cite{BEC:Fetter_revmodphys_2009}:
\begin{itemize}
\item Mean-field theory and experiment agree well for large atom numbers ($N>10^3$) and rotational frequencies reaching approximately $0.995\omega_{\perp}$.
\item The density profile of slowly rotating condensates differs very little from that of non-rotating condensates, except for the inclusion of the vortex cores.
\item The Thomas--Fermi regime holds true for frequencies of $0.75 \leq \Omega \leq 0.99$, where kinetic energy remains negligible compared to flow velocity.
\item The MFQH regime description, holding for $0.99 \leq \Omega \leq 0.999$, sees the vortex cores expanding to fill all available condensate space. %Density variation gives rise to energy terms of the order of the flow velocity.
\end{itemize}

Given the above criteria, the rapidly rotating condensate system will be capably modeled using the mean-field Gross--Pitaevskii theory at a rotation rate of $\Omega \approx 0.995\omega_{\perp}$, assuming $N\approx 10^6$ atoms, as used in typical experimental setups. For the work described later I will assume the above values for the generation of a vortex lattice carrying condensate.

%\subsection{Vortices in largely asymmetric potentials}
%Most of the work reviewed previously involved symmetric trapping geometries when discussing the fast rotation regime. The use of asymmetric trapping potentials may prove to have interesting effects on condensate behaviour. An examination of vortex lattice structures in the fast rotation limit is offered by Oktel \cite{Vtx:Oktel_pra_2004} for the 2D trapping potential stretched along one of the dimensions. He showed that the lattice arrangement still favours an Abrikosov triangular structure, and states that this is within an experimentally accessible parameter range. If we consider equating the lowest trapping frequencies with the rotation frequency how will the system behave? Two papers investigating this problem are that of Sinha \textit{et al}. \cite{Vtx:Sinha_prl_2005}, and Sanchez-Lotero \textit{et al}. \cite{Vtx:Lotero_pra_2005}, and both demonstrate a similar finding: when the rotation rate approaches that of the lower trap frequency the condensate extends along the more loosely confined dimension, and becomes ``free'' when the frequencies are equal. A quantum channel forms as a result, with the vortices aligning in rows. However, with an increase in interaction strength between bosons in the condensate, the triangular vortex lattice can be recovered. Sinha \textit{et al}. mention that, similar to the symmetric case, the mean-field treatment fails as the vortex number approaches that of the atom number. The authors do discuss that, given a proper treatment accounting for quantum fluctuations, the obtained lattice is expected to melt in this limit. Further similarities between the symmetric and asymmetric trapping geometries are drawn by Fetter \cite{Vtx:Fetter_pra_2007}, who considers a 2D condensate at rotation rates approaching the lesser of the trap frequencies. Both systems can be modelled using a LLL wavefunction approach, however pointing out that it is not clear if ``this situation holds for arbitrary anisotropy and less extreme rotation speeds''. Fetter also mentions the transition to the quantum channel regime, suggesting that a correlated state may be involved in this transition, leaving it open for further investigation. Aftalion \textit{et al}. \cite{Vtx:Aftalion_pra_2009} show that for the case of fast rotation in asymmetric trapping geometries the condensate is expected to not display any vortices in the bulk density. With minor anisotropy the condensate will have a Thomas--Fermi profile in the LLL description similar to the isotropic case, and also showing lattice distortion on the condensate edges as previously reported. In the fast rotation regime, however, the condensate bulk remains free of visible vortices, with no lattice forming. The resulting density profile shows an inverted parabola in the unconfined ($\Omega = \omega_i$) dimension, with a Gaussian profile in the other dimension. This behaviour has similarities to the condensate in a QQ potential, wherein the bulk density shows no vortices at high rotation, a result shown by Fetter \textit{et al}. \cite{Vtx:Fetter_pra_2005}. They conclude by stating that beginning with a resting condensate and subjecting it to high rotation in this regime should not nucleate any vortices.

%Filling the void between slow \cite{Vtx:Oktel_pra_2004} and fast rotation \cite{Vtx:Aftalion_pra_2009}, the work of McEndoo \textit{et al}. \cite{Vtx:McEndoo_pra_2009} investigates the anisotropic regime for moderate rotation speeds, and hence low vortex numbers. The vortex lattice structure is shown to transition from the triangular lattice to that of a linear chain. The authors also discover that the critical stirring frequency of the condensate to nucleate a vortex increases for larger anisotropies in 2D. The role of symmetry during vortex nucleation at various rotation rates was examined by Dagnino \textit{et al}. \cite{Vtx:Dagnino_natphys_2009}, who investigated changes in the system symmetry over a sweeping of the rotation rate of the condensate, $\Omega$. The nucleation of a single vortex into the condensate for a given rotation rate is a symmetry--breaking process, and close to the critical symmetry breaking value, $\Omega_c$, the system fails to be described accurately by mean-field theory, which is appropriate at values above and below this transition point. The system is expected to exhibit strongly correlated behaviour at the critical values, and the paper discusses the limitations of the mean-field approach around this point. This indicates that to use a mean-field approach, the system parameters must be carefully chosen so that the system remains stable and away from the symmetry breaking points. The study concludes by discussing a formalism for describing quantum modes that may be investigated using Bogoliubov--de Gennes theory, stating that it agrees well with the full theory. The authors also restate the problem as a ``requantization'' of the mean-field theory into single particle states, originally described by Parke \textit{et al}. \cite{Vtx:Parke_prl_2008} as a means to describe effects outside the mean-field approach, even at low rotational frequencies.

%Given the wealth of literature available on rotating condensates, a review of all experimental and theoretical results then available was compiled and published by Fetter \cite{BEC:Fetter_revmodphys_2009}. The author draws some conclusions and defines a future outlook for the field.
%\begin{itemize}
%\item Theory and experiment agree well for rotational frequencies reaching approximately $0.995\omega_{\perp}$.
%\item The density profile of slowly rotating condensates differs very little from that of non-rotating condensates, except for the inclusion of the vortex cores.
%\item The Thomas--Fermi regime holds true for frequencies of $0.75 \leq \Omega \leq 0.99$, where kinetic energy remains negligible compared to flow velocity.
%\item The MFQH regime LLL description, holding for $0.99 \leq \Omega \leq 0.999$, sees the vortex cores expanding to fill all available condensate space. Density variation gives rise to energy terms of the order of flow velocity.
%\end{itemize}
%Following these statements, Fetter outlines some goals to observe effects and behaviours predicted to occur in rotating ultracold gas systems that, at the time of writing, still remain to be observed. Approaching and examining the behaviour of a condensate in the fractional quantum Hall regime, wherein $\nu$ is small, remains to be demonstrated. Rapidly rotating Fermi--gases are discussed also as an area requiring future investigation, and which may also provide insight into low $\nu$ states and behaviours.

%\subsection{Optical lattice potentials}\label{sub:optlatt}
%Trapping potentials for Bose--Einstein condensates are typically harmonic, and are formed using magnetic (such as Ioffe--Pritchard) or optical potentials (such as dipole traps). Although these traps provide a single trapping location for all the atoms, interesting behaviour can also be observed and investigated using of an array of traps. One method of achieving this is the use of counter-propagating laser fields, allowing for a periodic optical potential to be created. For two sets of counter-propagating laser fields in a plane at right angles, an array of one-dimensional cigar-shaped condensate traps can be created, wherein the atoms are tightly confined in 2D. Adding an additional set of fields perpendicular to the existing ones will allow trapping of the atoms to periodic locations corresponding to a cubic lattice \cite[~chap. 2]{BK:LH_2010}. %To model bosonic atoms sitting in an optical potential, the Bose--Hubbard Hamiltonian can be considered \cite{BEC:Bloch_revmodphys_2008}. This model accounts for atoms hopping between nearest neighbour sites on the lattice and also includes the presence of an interaction term,
%\begin{equation}\label{eqn:bosehubbard}
%\hat{H}_{BH} = -J \displaystyle\sum\limits_{<i,j>} \left(\hat{a}_i^\dagger \hat{a}_j + \hat{a}_j^\dagger \hat{a}_i \right) + \frac{U}{2}\displaystyle\sum\limits_i \hat{n}_i(\hat{n}_i - 1).
%\end{equation}
%One of the main features of atoms in optical lattices is that they can be tuned to move between the mean-field and the strongly correlated regime \cite{THS:McEndoo_2010}. %In this model the system must be assumed to be subjected to the optical lattice on timescales that allow for it to come to equilibrium within the lattice sites. For the work proposed later in Section \ref{sec:prelim}, this model may not be useful initially as the timescales of application are much shorter than those required to allow the system to reach equilibrium. It may become a necessary part of explaining some system behaviour and dynamics in both referenced and proposed future work, and it is thus valuable to discuss it here. Due to the vast amount of available literature I will concentrate on a much smaller subset of research in this area.

%In 2002, the first realisation of a condensate trapped in a three-dimensional optical lattice was achieved by Greiner \textit{et al}. \cite{OL:Greiner_nat_2002}. They successfully realised an optical lattice in which the depth of the individual lattice sites could be adjusted by the intensity of the laser fields. By increasing the intensity from zero allows each site to obtain a number of condensate atoms. The authors show that increasing the intensity of the lattice lasers allows the system to move from a long-range phase coherent (BEC) state, to one where the phases between individual sites are no longer coherent, the so-called Mott insulator state. Such a system can no longer be described by a mean-field theory, and requires the use of a discrete quantum treatment, offered by the Bose--Hubbard model. To ensure the system remains in the ground-state, the intensity of the lattice potential was ramped to the required value slowly compared to the timescales for the system to react. This ensured that no higher modes were excited, and allowed the condensate atoms to distribute amongst the trapping potential.

%A theoretical investigation into the effect of optical lattices on vortex lattice structures was undertaken by Reijnders and Duine \cite{OL:Reijnders_prl_2004}. The authors were able to show that for in the presence of an optical potential which is not matched to the vortex lattice the vortices would break from an Abrikosov pattern, and become pinned to optical lattice sites. This was experimentally verified by Tung \textit{et al}. \cite{Vtx:Tung_prl_2006}. They begin by comparing and contrasting pinned and Abrikosov vortex systems, and find that both allow ``correlated many-body states'' to be investigated. They consider competition between both lattice structures, and investigate the resulting behaviour of the condensate. Taking a 2D optical lattice superimposed upon the condensate in the fast rotation limit, the maxima of the optical potential create regions wherein it is energetically favourable for a vortex to sit. The authors show that a vortex may be pinned to a position given by the optical lattice, and through variation of the optical intensity a structural transition can occur from the triangular Abrikosov lattice to that of the optical lattice pattern. This is favoured when the two lattices rotate at the same rate, allowing the lattices to effectively ``lock''. Using this technique the authors demonstrated the transition to a square lattice structure.

%The use of optical lattices in combination with Bose--Einstein condensates in the manner described previously allows for many interesting effects unique to such systems. One such interesting body of theoretical work was that of S{\o}rensen \textit{et al}. \cite{OL:Sorensen_prl_2005}, which involved the creation of FQH states, in a manner different from the fast-rotation route. The authors describe a process starting with a condensate in a harmonic potential in the Mott insulator state, going to the FQH state via the application of a magnetic field and reduction of the optical lattice potential, where the resulting state is in the LLL. A related work by Palmer \textit{et al}. \cite{OL:Palmer_prl_2006} investigates cold bosons in an optical lattice in the presence of large magnetic fields. The authors show that it is possible to generate FQH states, with varying fractional values. However, the proposed method is rather complex, and is described as requiring near future experimental technology to generate and observe such structures.

%Another body of work investigating such behaviour was carried out by Vignolo \textit{et al}. \cite{Vtx:Vignolo_pra_2007}. They investigate vortex dynamics in a condensate upon excitation by a 2D optical lattice. This theoretical study restricts itself to a single vortex, and can therefore be considered realisable using current state of the art technology \cite{Vtx:Matthews_prl_1999,Vtx:Dobrek_pra_1999,Vtx:Tung_prl_2006}. Assuming a zero-temperature Bose gas, and a Bose--Hubbard Hamiltonian description, the authors derive a means of determining a vortex mass of the resulting system. This mass is stated as being easily tunable through variation of the confining potential along the rotation axis, assuming a pancake-shaped condensate. The use of Feshbach resonances \cite{BEC:Chin_revmod_2010} is also mentioned as a possible tool for altering the scattering length of the atoms, which provides a knob for the vortex mass to be adjusted. Through the use of Bragg scattering spectroscopic techniques, the authors state that in the presence of a vortex, the optical lattice structure exhibits a resonance allowing measurement of the vortex mass, which is absent when no vortex is present. %The paper concludes by mentioning possible further studies, such as the case wherein interaction and tunnelling strengths are of the same order, or when many vortices are present within the system. These are stated to give rise to more complex system dynamics, and are, in the authors opinions, experimentally realisable.

%\subsection{Recent advances and discoveries}
%There has been a significant amount of literature examining condensates with and without rotation and optical lattices, with both experimental and theoretical progress continuing in this area. The group of G. Campbell \cite{Vtx:Wright_pra_2013} examine the behaviour of a condensate in an annular trapping geometry, which is stirred by an optical potential barrier of varying intensity. Beginning with a stationary condensate, stirring was performed at a fixed rotational velocity using a repulsive potential barrier with diameter less than the trapping annulus width. Using the Thomas--Fermi approximation allowed for estimates of the chemical potential, $\mu$, which in turn determined a value for the trap width and barrier width. Starting from zero intensity, the barrier was ramped to a maximum intensity over 100 ms, remained there for 800 ms and decreased over 100 ms. Following a time-of-flight expansion, excitations of the condensate were observed. Release of a non-rotating condensate shows the central hole of the condensate to close, resulting in a large peak at the centre. In contrast, a circulating condensate will show the presence of one or more holes, which signify ``the presence of phase singularities (vortices)''. If a central hole is observed in the condensate density, the authors state that this is resulting from the presence of persistent currents in the condensate, which prior to release were flowing around the annular trap. The resulting diameter of the hole is dependent upon the winding number of the circulating current in the condensate. Holes that were reported as being off-centre show the presence of topological excitations (vortices) in the condensate.  The authors show the probability of these excitations for a range of differing barrier heights and rotational frequencies, and attempt to formulate a theoretical framework to describe the results. Recognising that a full 3D model may be required to accurately describe the behaviour, the authors restrict themselves to a 1D approach, stating that ring-shaped geometries can be effectively treated as such. Although differences exist between experimental results and the theoretical approach taken the results were shown to agree qualitatively, where Campbell \textit{et al}. suggest many possible reasons for such differences. This body of work represents some of the most recent experimental work on rotating condensates, and, as the authors have shown, requires further examination theoretically to accurately explain the results.

\section{Recent progress and outlook}
Recently there have been many advances in the control of atomic BEC systems. A notable example is the work of Gauthier \textit{et} al. \cite{BEC:Gauthier_arxiv_2016}, wherein they demonstrate arbitrary optical potential generation for Bose--Einstein condensates through use of digital micromirror devices (DMD). The authors demonstrate high resolution control and patterning of the condensate, and show near perfect control of the condensate atomic distribution. Given the current state of the art high performance imaging and control techniques available, these experimental systems can allow for high precision control and manipulations of the atoms, and therefore also vortices.

Another recent experimental work of note is that of Wigley \textit{et} al. \cite{BEC:Wigley_scirep_2016}, with a completely automated approach to BEC generation and control. Such automation can allow for much higher throughput of experimental data collection, and allow for a much wider breadth of physics to be explored in condensate systems.

The current state of the art experimental systems can offer a very high degree of control of condensates, and their ensuing dynamics. In fact, all further methods discussed can be built using currently available state of the art systems, and hence are experimentally realisable.
