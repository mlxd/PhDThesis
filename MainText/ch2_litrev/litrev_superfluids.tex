\section{Superfluidity}\label{sec:superfluid}
\subsection{Introduction to superfluidity}
%A classical fluid in a rotating vessel will eventually reach a steady-state wherein the fluid rotates as a whole with the container as a solid-body. This behaviour emerges as a result of the viscosity of the fluid and, assuming the no-slip condition \cite{FL:Day_2010}, that the fluid elements in direct contact with the boundary of the container adhere to its surface. Given sufficient time to reach equilibrium, the entire fluid will rotate at the same angular velocity. If the driving forces are removed, this flow will eventually cease to rotate, and once again become stationary. Such flow can be contrasted with that of an inviscid flow, and will not experience any effect from the boundary rotation. Once in motion, an inviscid fluid will continue to rotate indefinitely. This type of flow can be realised with superfluids \cite{BK:Prandtl_2010}.

Superfluidity is a macroscopic quantum effect that is closely related to Bose--Einstein condensation. Liquid helium has been known for many years to exhibit superfluid behaviour \cite{BEC:Penrose_pr_1956}. One interesting property of superfluids is that they have quantized circulation which can lead to the appearance of quantum vortices under rotation. Traditionally, such excitations have been created in liquid $^4$He by a ``rotating bucket'' type of experiment, where the container is rotated about a single axis. When the fluid is initially above the $\lambda$-critical point, the temperature at which $^4$He moves from a classical fluid to a superfluid, the atoms will undergo solid-body rotation with the container. On cooling the atoms below the $\lambda$-critical point liquid $^4$He will undergo a transition to the superfluid state. If the velocity is above a critical rotation value vortices are nucleated in the rotating superfluid helium system. Due to the strongly interacting nature of liquid helium the nucleated vortices are difficult to visualise as the healing length, $\xi$, of liquid $^4$He is on the order of {\r{A}}ngstr{\"o}ms \cite{BEC:Srinivasen_pramana_2006}.

In order to visualise such vortices experimentally it was necessary to use an indirect means of visualisation in the form of tracer particles \cite{BEC:Packard_physb_1982}. Limited success was had with this technique using solid hydrogen, and later plastic microspheres, as they tended to join together due to static charges. An improved technique, using charged particles, showed much greater success \cite{Vtx:Packard_prl_1969}. Ions, or electrons, were trapped inside the vortex lines, and an electric field along the direction of the lines allowed for acceleration of the charges towards a luminescent screen where they could be observed. Current experimental work in vortex visualisation with liquid $^4$He has advanced significantly \cite{Vtx:Tsubota_arxiv_2010,Vtx:Guo_pnas_2014}, yet fine control over the behaviour of the liquid and the vortex dynamics remains difficult.

Superfluid behaviour is observed in liquid helium due partly to a portion of the atoms condensing into the ground-state. Given that BECs show superfluidity, the use of a dilute gas where the majority of atoms are in the condensate state provides a more controlled means to investigate superfluidity and quantised vortices \cite{BK:Ueda_2010,BEC:Srinivasen_pramana_2006,Vtx:Tsubota_arxiv_2010,CT:Tsubota_jpsj_2008}. In contrast with liquid $^4$He, having a healing length of the order of {\r{A}}ngstr{\"o}ms, the healing length of a dilute gas of alkali atoms is on the order of microns \cite{Vtx:Isoshima_pra_1999}. This places condensates in a much more accessible regime for visualising vortices compared with liquid $^4$He experiments. To fully understand the behaviour of the system, it is necessary to have a framework for modelling the condensate, in the absence and the presence of vortices. As described above, a dilute gas of condensed atoms close to absolute zero temperature can be readily modelled by the Gross--Pitaevskii equation \eqref{eqn:gpe_rotation}, offering a direct means to simulate condensate behaviour. It is useful to obtain a hydrodynamic description for the condensate, performed by treating its wave-function as
\begin{equation}\label{eqn:madelung}
\Psi(\textbf{r},t) = \vert\Psi(\textbf{r},t)\vert e^{i\theta(\textbf{r},t)},
\end{equation}
where $\theta(\textbf{r},t)$ is the condensate phase, and $\vert\Psi(\textbf{r},t)\vert=\sqrt{\rho(\textbf{r},t)}$, the square root of the condensate density \cite[~chap. 1]{BK:Pitaevskii_Stringari_2003}.
The formalism used below follows that given by Ueda \cite{BK:Ueda_2010}, where the condensate density is given by
\begin{equation}\label{eqn:density}
\rho(\textbf{r},t) = \Psi^*(\textbf{r},t)\Psi(\textbf{r},t) = \vert \Psi (\textbf{r},t) \vert ^2.
\end{equation}
The continuity relation takes the form
\begin{equation}\label{eqn:continuity}
\frac{\partial}{\partial t}\rho(\textbf{r},t)  + \nabla \textbf{j}(\textbf{r},t) = 0,
\end{equation}
where the current density of the condensate is
\begin{equation}\label{eqn:current_density}
\textbf{j}(\textbf{r},t) = \frac{-i\hbar}{2m}\left[\Psi^*(\textbf{r},t)\nabla\Psi(\textbf{r},t) - \Psi(\textbf{r},t)\nabla\Psi^*(\textbf{r},t)\right].
\end{equation}
Using ansatz \eqref{eqn:madelung} and substituting it into Eq. \eqref{eqn:current_density} then gives the form for the current density,
\begin{equation}
\textbf{j}(\textbf{r},t) = \vert\Psi(\textbf{r},t)\vert ^2\frac{\hbar}{m}\nabla\theta(\textbf{r},t).
\end{equation}
The velocity of the superfluid, $\textbf{v}(\textbf{r},t)$ is defined as the ratio of the current density to the density, and then is given by
\begin{equation}\label{eqn:velocity}
\textbf{v}(\textbf{r},t)\equiv \frac{\textbf{j}(\textbf{r},t)}{\rho(\textbf{r},t)} = \frac{\hbar}{m}\nabla\theta(\textbf{r},t).
\end{equation}
The gradient of the phase therefore gives the velocity of the condensate atoms; this indicates that the superfluid behaviour in a condensate is irrotational ($\nabla\times(\nabla\theta) =0$). Assuming a closed loop integral about a central point in the condensate, and recalling the single-valued nature of the wavefunction, yields the relationship
\begin{equation}\label{eqn:circulation}
\oint_C \textbf{v}\cdot d\textbf{l} = \frac{\hbar}{m}2\pi l.
\end{equation}
This shows the quantised nature of circulation in a superfluid, with $l$ representing the integer charge of the circulation. The phase winding around the central region is given by multiples of $2\pi$, with the centre of the phase becoming ill-defined. To circumvent this problem the density at this point drops to zero, signalling the presence of a vortex in the condensate density. This drop happens over the scale of the healing length, which, for repulsive interactions, is given by
\begin{equation}
\xi = \frac{1}{\sqrt{8\pi \rho a_s}},
\end{equation}
where $\rho$ is the bulk density of the condensate, and $a_s$ is the s-wave scattering length. To sustain a vortex the condensate must have sufficient angular momentum, which is imparted via the angular momentum term $-\Omega L_z$ in the Hamiltonian given by Eq. \eqref{eqn:gpe_rotation}. This term accounts for rotation of the condensate, and puts us in the co-rotating frame. The angular velocity, $\Omega$, of the condensate has an upper-bound stability limit equivalent to the transverse trapping frequency, $\omega_{\perp}$, of the harmonically trapped condensate.


\subsection{Vortices in Bose--Einstein condensates}\label{ss:vorticesinbec}















%%%%%%
%% TO BE DISECTED!
%%%%%%
%\subsection{TO BE DISSECTED AFTER THIS POINT: Vortices in Bose--Einstein condensates}\label{ss:vorticesinbec}
%In \cite{BEC:Stringari_prl_1996} Stringari derives equations for the moment of inertia for the condensate, and shows that above $T_c$ the moment of inertia can be described classically. Below $T_c$ the moment of inertia is represented as a sum of two terms, one describing the classical solid-body rotation of the thermal cloud, the other describing the irrotational flow of the condensed cloud.  At zero Kelvin, the moment of inertia is given entirely by the irrotational component, indicating that all atoms are condensed.

%A paper by Dalfovo and Stringari \cite{Vtx:Dalfovo_pra_1996} suggested the possibility of exciting vortices using an anisotropy in the trapping potential. The authors discuss the applicability of mean-field Gross--Pitaevskii theory to a such problem. An examination into the behaviour of vortex flow is given and the solutions for the ground-state and vortex carrying states are found via numerical minimisation of the Gross--Pitaevskii energy functional. A decrease in the critical angular velocity for vortex formation with an increase in the number of condensate atoms, $N$, is shown. For attractive interactions, the critical angular velocity increases with larger $N$. Ground-state results are compared to experimental data from the JILA (Boulder) experiment \cite{BEC:Cornell_science_1995}, and were found to be in good agreement. A closely related work is one by Barenghi \cite{Vtx:Barenghi_pra_1996}, who discusses finite temperature corrections to the zero temperature regime to account for displacements of the vortex cores. It shows how small thermal excitations can, even at low temperatures, affect the vortex position. The work of Lundh \textit{et al}. \cite{Vtx:Lundh_pra_1997} further extends that of Dalfovo and Stringari. Here, the authors investigate the lowest required angular velocity that allows a vortex to enter a condensate, and consider corrections to the Thomas--Fermi approximation. They discuss the use of both isotropic and anisotropic trapping geometries, and find good agreement between the numerical integration of the GPE for experimentally realistic parameters, and analytical expressions they derive for the kinetic energy and the lower-bound on the critical angular velocity for a vortex to enter the atomic cloud.

%Marzlin \textit{et al}. \cite{Vtx:Marzlin_prl_1997} proposed a different method for generating vortex states. They consider a condensate subjected to two Laguerre--Gaussian beams of $\sigma^+$ and $\sigma^-$ polarisation respectively. The beams are chosen to be resonant with an internal transition, and transfer angular momentum through the coupling allowing the condensate to start rotating. Another method, investigated numerically by Jackson \textit{et al}. \cite{Vtx:Jackson_prl_1998} uses a blue-detuned light field to simulate an object moving through an effective two-dimensional BEC, which can be achieved by tightly confining it in the $z$-direction. They show the formation of vortices close to the centre of the laser potential, whose movement causes a ``phase-slip'', with the value of the phase going from $-\pi$ to $\pi$ around the vortex. They also note that the speed of vortex shedding from the object has an inverse relationship to the speed of sound in the condensate, given as
\begin{equation}
c(\textbf{r},t) = \sqrt{\frac{\rho (\textbf{r},t) U}{m}},
\end{equation}
where $U=4\pi\hbar^2 a_s/m$ represents the repulsive atom-atom interaction \cite{BEC:Andrews_prl_1997}. %A second paper by Marzlin \textit{et al}. \cite{Vtx:Zhang_pra_1998} suggests the use of an optical dipole potential resulting from four travelling-wave beams to excite a trapped condensate and induce circulation by transfer of angular momentum. Results of this study show that for harmonic trapping potentials a superposition of vortex states is created. Pure vortex states are shown to arise only from anharmonic traps combined with a rotating linear potential. This topic is treated by the same authors in a follow up work \cite{Vtx:Marzlin_pra_1998}.

%Building upon the earlier discussion by Marzlin \textit{et al}., Goldstein \textit{et al}. \cite{Vtx:Goldstein_pra_1998} consider schemes for accurate detection of the topological charge of a trapped condensate containing vortices. Their idea builds on the interference of the vortex carrying condensate with a second ground-state condensate, and they show that the vanishing of interference lines is related to the vortex charge. A work provided by Svidzinsky \textit{et al}. \cite{Vtx:Svidzinsky_pra_1998} discusses the hydrodynamic formalism within the Thomas-Fermi limit, and show it is an effective means of describing condensate behaviour in the presence of vortices. An alternative approach is presented by Lundh \textit{et al}. \cite{Vtx:Lundh_pra_1998}, who consider methods for vortex detection methods in large condensates. They discuss the difficulties that arise when only a single vortex is present, as the energy difference to the vortex-less ground state would be too small to measure. They consider the free expansion of both a two and three-dimensional condensate, which allows a singly charged vortex core to expand as well and become visible on an absorption picture. They note that in two-dimensions (2D) the vortex core and condensate expand at roughly the same rate, making detection somewhat more challenging than in three-dimensions (3D) where the core initially expands faster than the condensate. Lundh \textit{et al}. also note that in the limit of weak coupling a difference exists between the aspect ratios of condensates with and without vortices, and that this effect is enhanced by anisotropic trapping geometries.

The stability of vortex states in a condensate is also a widely discussed topic \cite{Vtx:Fedichev_pra_1999,Vtx:Feder_prl_1999}, as non-rotating traps show an instability for small displacements of the vortex from the trap centre. However, Feder \textit{et al}. \cite{Vtx:Feder_prl_1999} calculated the critical rotation frequencies for vortex stabilisation at the centre of an anharmonic rotating trap. Further proposed schemes to realise vortex carrying condensates are given by the following works \cite{Vtx:Anglin_prl_1999,Vtx:Davies_prl_1999,Vtx:Marshall_pra_1999,Vtx:Dobrek_pra_1999}. The method proposed by Dobrek \textit{et al}. \cite{Vtx:Dobrek_pra_1999}, shows a means to optically generate vortices by the use of what they term a ``phase-imprinting'' method. The authors describe a scheme where
the phase of the condensate is controlled directly, and given the required topological charge to induce a vortex during evolution. Through use of an absorption plate whose axis angle depends on the absorption coefficient, the condensate can be imprinted with the required phase pattern.

%The realisation of vortices in a two-component condensate was first achieved by Matthews \textit{et al}. \cite{Vtx:Matthews_prl_1999}. Using two internal states of $^{87}$Rb, the authors coherently coupled and induced transitions between both in such a way that one component acquired a $2\pi$ phase winding and rotated around the other component. A different route to creating vortices was successfully used by Madison \textit{et al}. \cite{Vtx:Madison_prl_2000}, who tightly focused a laser to stir the condensate beyond a critical frequency for nucleation. This way they were able to generate multiple vortices in the condensate.

\subsection{Vortex lattices}
%In early 1999, Castin and Dum \cite{Vtx:Castin_epjd_1999} examined the ground-state solutions for multi-vortex condensates, simulating up to 18 vortices. Their work is mostly performed in effective 2D, with data for a 3D configuration given following the main body of work. At the same time Feder \textit{et al}. \cite{Vtx:Feder_pra_2000} considered the case where multiple vortices are present in the condensate, and performed a numerical investigation of this on a 3D condensate of $^{23}$Na. The authors further mention that given a sufficient number of vortices in the condensate an almost regular triangular array would be observed. Evidence for this behaviour could be seen in the experimental data obtained by Madison \textit{et al}. \cite{Vtx:Madison_jmo_2000}, showing up to 11 vortices that seemingly arrange in a triangular array pattern. This resulted from laser stirring of the condensate, and was also covered in \cite{Vtx:Madison_prl_2000}. A follow-up study by Madison \textit{et al}. \cite{Vtx:Madison_prl_2001}, and Chevy \textit{et al}. \cite{Vtx:Chevy_aoi_2001} showed the relationship between vortex nucleation and the dynamical instabilities arising from the irrotational state, specifically, the excitation of the rotating quadrupolar mode of the condensate.

%Another method to realise vortices in a BEC is provided by Ogawa \textit{et al}. \cite{Vtx:Ogawa_pra_2002}. They proposed using spinor BEC in an Ioffe-Pritchard trap with a hyperfine $\vert F=1 \vert$ state. Through adiabatic variation of an axially applied magnetic field, $B_z$, a phase winding of $4\pi$ can be generated via manipulation of the spin degrees of freedom. An ``optical plug'', which is a blue-detuned laser field, was used along the vortex core to prevent atom losses due to Majorana spin flips when $\vert B \vert = 0$. The authors state that the process through which the phase accumulates a winding can be understood as a Berry phase during the evolution of the spins. The resulting vortex generated via this means is unstable due to the larger winding, and such will decay into singly charged vortices. Following the procedure given previously by Ogawa \textit{et al}., a theoretical analysis was performed by M\"ott\"onen \textit{et al}. \cite{Vtx:Mottonen_jpcm_2002} on vortex generation in a condensate with a hyperfine $\vert F=2 \vert$ state, showing the generation of a vortex with a phase winding of $8\pi$. This is reported to remain stable only in the presence of an optical plug, decaying into 4 singly charged vortices otherwise.

%Of all the work on multiple vortex generation and the underlying theory, one of the seminal works of the time was the experiment from Ketterle's group on vortex lattices \cite{Vtx:AboShaeer_sci_2001}. The authors begin with a comparison of vortices to the type of behaviour observed in type--II superconductors, where penetration of the magnetic field into the conductor is possible only via quantised flux lines. They then state that ``(v)orticity can enter rotating superfluids only in the form of discrete line defects with quantized circulation'', and therefore establish a close connection between the rotational properties of superfluids and superconductors. Furthermore, their experiment finds Abrikosov (triangular) style vortex lattices, which are also known from type--II superconductors. The group generated condensates with lattices containing upwards of 130 vortices, in a well ordered triangular formation. This was achieved through stirring of the condensate with a blue-detuned laser which was moved symmetrically around the condensate at a specified angular frequency, $\Omega$. The conclusions drawn by the authors are that BECs are a robust system for generating vortex lattice structures that naturally form into a regular triangular pattern, and show many similarities to type--II superconductors.

%Ho \cite{Vtx:Ho_prl_2001} showed that the behaviour observed in the fast rotation limit closely resembles that of the two-dimensional quantum Hall regime, where the system is residing in the lowest Landau level, $LLL$ ($n=0$). Ho states that the experimental data suggests the MIT group has not yet reached the LLL regime, but are very close. As the rotation of the cloud approaches the trapping frequency, the system tends to the LLL. However, in this regime the Thomas--Fermi approximation becomes invalid, requiring a more accurate consideration of the $z$-components. He concludes by stating that rotation close to the trap frequency will require treatment beyond that of the mean-field approach. An additional study from the MIT group \cite{Vtx:Raman_prl_2001} derived a result showing that the number of vortices in the condensate is proportional to the rotational frequency of the stirrer at a given frequency. They discuss the cause of vortex nucleation as a consequence of surfaces modes in the condensate and due to localised regions of turbulent behaviour.

%The group at JILA devised a method to systematically evaporatively cool and rotate a cloud of $^{87}$Rb confined in a harmonic potential \cite{Vtx:Haljan_prl_2001}. Beginning with a normal fluid component close to $T_c$, the harmonic trapping potential is initially deformed and rotated. This induces a rotation of the cloud, until it enters a solid-body rotational state with the trap. Upon reaching a steady-state, the rotating asymmetry is switched off, and the cloud is evaporatively cooled. This approach is similar to that of the rotating bucket experiment in liquid $^4$He. Applying an additional distortion to the trapping geometry allows for the removal of atoms with large axial displacements to take place, and as a result of this, the temperature of the cloud is reduced by a factor of four. Additionally, the rotation rate of the remaining cloud increases as the angular momentum per particle stays close to constant. At the end of the evaporation process, a condensate forms at the centre of the cloud, rotating with a frequency as high as 0.94 times the confining potential frequency. Results are compared with the theoretical calculations of Feder and Clark \cite{Vtx:Feder_prl_2001}, and were stated to agree.

%Although a large number of works exist which investigate systems with low numbers of vortices \cite{THS:Davies_2000,Vtx:Chevy_prl_2000,Vtx:Cooper_prl_2001,Vtx:Rosenbusch_prl_2002,Vtx:Ogawa_pra_2002,Vtx:Bretin_joptb_2003}, I will concentrate primarily on studies of systems containing a large number of vortices. Engels \textit{et al}. \cite{Vtx:Engels_prl_2002} were capable of creating condensates containing equivalent numbers of vortices to those obtained in the MIT experiments, and investigated the behaviour of these lattices with respect to excitation of higher lying modes. The authors show a ``sheetlike structure'' of the condensate vortices in the presence of an $m_z=-2$ surface mode. It is discussed that due to the difficulty in obtaining dense vortex lattices in pancake-like condensates that this may provide a means to investigate such behaviour.
%without the need to consider then experimentally inaccessible rotation rates and low atom numbers.

%Adhikari \textit{et al}. \cite{BEC:Adhikari_pra_2002} carried out numerical simulations of condensate behaviour subjected to a sudden perturbation of the trapping potential. The authors discuss changes in the interaction strength, the trapping potential, and the rotation of the condensate, and how a vortex will decay once rotation has ceased. Their simulations considered quasi (effective) two-dimensional condensates and solved the GPE using a split-operator technique \cite{BEC:Javanainen_jphysa_2006}. Subjected to a perturbation pulse the condensate boundary starts to oscillate, and the number of supported vortices changes dependent upon the sign of interaction change or trapping potential.

\subsection{Fast rotation limit}
%Mueller and Ho \cite{Vtx:Mueller_prl_2002} considered the case of a two-component condensate under fast-rotation, above the previously reported rotation rates that were experimentally accessible. Such systems are stated to be in the ``mean-field quantum Hall regime'', where each component can be described by an individual wave-function with an angular momentum large enough to ensure the LLL regime. These systems are a generalisation of the work by Ho \cite{Vtx:Ho_prl_2001}, and the authors show that the minimum energy state corresponds to a pair of interspersed vortex lattices, one for each respective component. They also show that variation of the interaction parameters allows the geometric arrangement of the lattices to be be modified, for example into a rhombic structure. An additional theoretical examination of vortices in the quantum Hall regime $(\Omega\approx\omega_{\perp})$ is offered by Subrahmanyam \cite{Vtx:Subrahmanyam_pra_2003}. At such rotation rates the aspect ratio of the condensate, namely the ratio of the width in the $z$-dimension to the width in the $x$--$y$ plane, becomes very small yielding an effective two-dimensional system.
%This is in line with the two-dimensional quantum Hall system, where the number of vortices will be on par with the number of atoms in the condensate, and where the vortex lattice unit cell equals the Larmor circle.
%It is also stated that in this limit the use of mean-field theory will fail due to perturbations becoming more prominent, and a full model for interacting bosons becomes necessary to fully describe the dynamics. For rotation rates less than this limit, however, the vortices arrange themselves into the expected triangular lattice. This work in the mean-field quantum Hall (MFQH) regime is further extended by Regnault and Jolicoer \cite{Vtx:Regnault_prl_2003}, who describe fractional quantum Hall (FQH) behaviour in condensate systems by using a non mean-field formalism. For the body of work I describe later, it will be assumed that the rotation rates will remain below FQH regime rotation rates, allowing for a MFQH description.

%In his thesis, Penckwitt \cite{THS:Penckwitt_2003} examines the theoretical formalism behind vortex lattice generation and behaviour in a condensate. Of particular interest is the settling of the lattice into a stable state using a rotating atom cloud. The basis of the investigation is formed from results given by Gardiner \textit{et al}. \cite{BEC:Gardiner_jphysb_2002}, with discussion of and investigation into some of the methods presented from the experimental results available at the time.

%Given the known upper limit of the rotational frequency for which a harmonically trapped condensate is stable, which is the harmonic trapping frequency, one might inquire as to whether other configurations exist in which higher rotation rates can be achieved. An investigation performed by Bretin \textit{et al}. \cite{BEC:Bretin_prl_2004} considers a condensate trapped within a quadratic plus quartic (QQ) potential, with an additional theoretical analysis for both the repulsive and attractive interaction cases given by Ghosh \cite{Vtx:Ghosh_pra_2004}.
%Beginning with a mention of all known work on the area, including but not limited to \cite{Vtx:Matthews_prl_1999,Vtx:Madison_prl_2000,Vtx:AboShaeer_sci_2001,Vtx:Haljan_prl_2001,Vtx:Ho_prl_2001,Vtx:Regnault_prl_2003}, Bretin \textit{et al}. examine the behaviour of the system at the limiting point $\Omega=\omega_{\perp}$, stating that the angular momentum of the system becomes singular. A comparison is made to a charged particle in a magnetic field, with an energy spectrum reminiscent of Landau levels with a $2\hbar\omega$ separation, as stated previously by Mueller and Ho \cite{Vtx:Mueller_prl_2002}. Through inclusion of the additional quartic term, the system can be examined at rotation frequencies at, and slightly beyond, the trapping frequency. Rotation frequencies of up to 0.95$\omega_{\perp}$ are shown to result in regular vortex lattices, with values beyond this inducing distortions to the structure of the lattice. The authors also mention that at rotation frequencies close to the trapping frequencies significant time is required ``to reach a well ordered vortex lattice'' when performing imaginary time evolution to find the system ground-state. The observation of giant vortices, which are possible within this regime, are mentioned but not discussed in much detail.

%The creation of such a MFQH system which can be described by the LLL was another milestone for the JILA group, and was reported in a paper by Schweikhard \textit{et al}. \cite{Vtx:Schweikhard_prl_2004}. The authors successfully demonstrate a condensate rotated upwards of 0.99 times $\omega_{\perp}$. Given that this regime displays behaviour different to the known results at the time, the importance of comparing the regime to the well known Thomas--Fermi limit was an important next step. Such a work was carried out by Watanabe \textit{et al}. \cite{Vtx:Watanabe_prl_2004}. They showed that the density profile of the rotating condensate in the MFQH limit was described well by the Thomas--Fermi approximation. This was shown to hold true provided that the number of vortices is much larger than unity, or that the condensate size is large in comparison with the harmonic oscillator length in the transverse direction, $\bar{a}_{\perp} = \sqrt{\hbar/m\omega_{\perp}}$. Comparing the MFQH regime to the Thomas--Fermi limit was also studied by Zhai \textit{et al}. \cite{Vtx:Zhai_pra_2004} who investigated the behaviour of two overlapping vortex lattices.

%As mentioned earlier, the Thomas--Fermi limit describes the case where the kinetic energy term of the Hamiltonian may be neglected in comparison to the interaction energy, as it offers little contribution to the condensate behaviour. This remains true for low rotation rates, however kinetic energy becomes important in the limit of fast rotation. In this case $\Omega/\omega_{\perp}\approx 1$, and the centrifugal force term, $m\Omega^2r$, almost balances with the trapping force term, $-m\omega^2r$. The kinetic energy term can no longer be neglected here, but one may now consider the condensate in the MFQH regime. The interaction energy is then much less than the gap between energy levels for single particle Landau levels \cite{Vtx:Ho_prl_2001}. Zhai \textit{et al}. state that the difference between these two regimes are characterised purely by the ratio of the kinetic energy to the interaction energy of the system, and in the MFQH regime we can thus assume the LLL has been achieved.

%Assuming the LLL description for a fast rotating Bose--Einstein condensate, the Dalibard group provide many theoretical works using this framework, beginning with the work offered by Aftalion \textit{et al}. \cite{Vtx:Aftalion_pra_2005}. The authors describe the case where the number of atoms to the number vortices, denoted as the ``filling factor'', $\nu=N/N_v$ \cite{BK:Ueda_2010,Vtx:Ho_prl_2001}, becomes an important characteristic of the system. In the case where the number of vortices is small compared to the number of atoms (still under the fast rotation limit), the system may be accurately described to be in the MFQH regime. However, as the rotational frequency of the condensate approaches the trapping frequency, the condensate width expands, and will tend to infinity. In this limit the number of vortices is comparable to the number of condensate atoms, and the system can no longer be described by a single macroscopic wave-function. Instead it can be described as being similar to ``an electron gas in the fractional quantum Hall regime''. To avoid dealing with the FQH regime, the authors restrict themselves to the MFQH regime, and choose rotational frequencies that guarantee this. The MFQH regime is entered when $1000 > \nu > 10$, which is attained close to the $\Omega / \omega_{\perp}\approx 1$ limit. One of the key conclusions from this paper is that the density distribution of the condensate in the MFQH regime is that of an inverted parabola, with the vortices forming a triangular lattice pattern that is almost regular. The area of each vortex cell differs from that of the solid-body rotating case, and is significantly distorted at the edge of the condensate. The results of Schweikhard \textit{et al}. \cite{Vtx:Schweikhard_prl_2004} are found to agree with what is proposed in this work, but they differ from the findings by Ho \cite{Vtx:Ho_prl_2001}, which predicted a Gaussian density distribution with an infinite regular lattice. For an almost perfectly regular lattice, the rotation rate must be sufficiently large that the condensate width extends to large distances and a large number of vortices are generated, without entering the FQH regime. For such a system, the vortices about the central region will be uniformly ordered due to the balanced velocity fields. Vortices on the condensate edge will have a distorted alignment, and can be neglected.

%A closely related work undertaken by Stock \textit{et al}. \cite{BEC:Stock_laserphyslett_2005} offers a brief review of many of the results relevant to vortex behaviour in condensates, up to and beyond the $\Omega=\omega_{\perp}$ regime. They discuss many of the known results in the MFQH regime, analogies to Landau levels, the use of QQ potentials, and briefly the FQH regime. Vibrational modes of vortex lattices, such as the Kelvin and Tkachenko modes, which are oscillations of the vortex line and vortex lattice respectively, are also reviewed. The use of QQ potentials is discussed as a way to allow observation of vortices with winding numbers larger than unity, where a purely quadratic potential would be unstable. Finally, they restate the fact that when the number of vortices approaches the atom number of the condensate, the system will no longer be capably described by a mean-field theory. Instead it enters a strongly correlated regime mirroring that of FQH regime, the observation of which is experimentally possible only for low atom numbers. A similarly comprehensive review of experimental results is offered by Srinivasan \cite{BEC:Srinivasen_pramana_2006}, and covers much of the same material, albeit with greater detail in parts.

%\subsection{Vortices in largely asymmetric potentials}
%Most of the work reviewed previously involved symmetric trapping geometries when discussing the fast rotation regime. The use of asymmetric trapping potentials may prove to have interesting effects on condensate behaviour. An examination of vortex lattice structures in the fast rotation limit is offered by Oktel \cite{Vtx:Oktel_pra_2004} for the 2D trapping potential stretched along one of the dimensions. He showed that the lattice arrangement still favours an Abrikosov triangular structure, and states that this is within an experimentally accessible parameter range. If we consider equating the lowest trapping frequencies with the rotation frequency how will the system behave? Two papers investigating this problem are that of Sinha \textit{et al}. \cite{Vtx:Sinha_prl_2005}, and Sanchez-Lotero \textit{et al}. \cite{Vtx:Lotero_pra_2005}, and both demonstrate a similar finding: when the rotation rate approaches that of the lower trap frequency the condensate extends along the more loosely confined dimension, and becomes ``free'' when the frequencies are equal. A quantum channel forms as a result, with the vortices aligning in rows. However, with an increase in interaction strength between bosons in the condensate, the triangular vortex lattice can be recovered. Sinha \textit{et al}. mention that, similar to the symmetric case, the mean-field treatment fails as the vortex number approaches that of the atom number. The authors do discuss that, given a proper treatment accounting for quantum fluctuations, the obtained lattice is expected to melt in this limit. Further similarities between the symmetric and asymmetric trapping geometries are drawn by Fetter \cite{Vtx:Fetter_pra_2007}, who considers a 2D condensate at rotation rates approaching the lesser of the trap frequencies. Both systems can be modelled using a LLL wavefunction approach, however pointing out that it is not clear if ``this situation holds for arbitrary anisotropy and less extreme rotation speeds''. Fetter also mentions the transition to the quantum channel regime, suggesting that a correlated state may be involved in this transition, leaving it open for further investigation. Aftalion \textit{et al}. \cite{Vtx:Aftalion_pra_2009} show that for the case of fast rotation in asymmetric trapping geometries the condensate is expected to not display any vortices in the bulk density. With minor anisotropy the condensate will have a Thomas--Fermi profile in the LLL description similar to the isotropic case, and also showing lattice distortion on the condensate edges as previously reported. In the fast rotation regime, however, the condensate bulk remains free of visible vortices, with no lattice forming. The resulting density profile shows an inverted parabola in the unconfined ($\Omega = \omega_i$) dimension, with a Gaussian profile in the other dimension. This behaviour has similarities to the condensate in a QQ potential, wherein the bulk density shows no vortices at high rotation, a result shown by Fetter \textit{et al}. \cite{Vtx:Fetter_pra_2005}. They conclude by stating that beginning with a resting condensate and subjecting it to high rotation in this regime should not nucleate any vortices.

%Filling the void between slow \cite{Vtx:Oktel_pra_2004} and fast rotation \cite{Vtx:Aftalion_pra_2009}, the work of McEndoo \textit{et al}. \cite{Vtx:McEndoo_pra_2009} investigates the anisotropic regime for moderate rotation speeds, and hence low vortex numbers. The vortex lattice structure is shown to transition from the triangular lattice to that of a linear chain. The authors also discover that the critical stirring frequency of the condensate to nucleate a vortex increases for larger anisotropies in 2D. The role of symmetry during vortex nucleation at various rotation rates was examined by Dagnino \textit{et al}. \cite{Vtx:Dagnino_natphys_2009}, who investigated changes in the system symmetry over a sweeping of the rotation rate of the condensate, $\Omega$. The nucleation of a single vortex into the condensate for a given rotation rate is a symmetry--breaking process, and close to the critical symmetry breaking value, $\Omega_c$, the system fails to be described accurately by mean-field theory, which is appropriate at values above and below this transition point. The system is expected to exhibit strongly correlated behaviour at the critical values, and the paper discusses the limitations of the mean-field approach around this point. This indicates that to use a mean-field approach, the system parameters must be carefully chosen so that the system remains stable and away from the symmetry breaking points. The study concludes by discussing a formalism for describing quantum modes that may be investigated using Bogoliubov--de Gennes theory, stating that it agrees well with the full theory. The authors also restate the problem as a ``requantization'' of the mean-field theory into single particle states, originally described by Parke \textit{et al}. \cite{Vtx:Parke_prl_2008} as a means to describe effects outside the mean-field approach, even at low rotational frequencies.

%Given the wealth of literature available on rotating condensates, a review of all experimental and theoretical results then available was compiled and published by Fetter \cite{BEC:Fetter_revmodphys_2009}. The author draws some conclusions and defines a future outlook for the field.
%\begin{itemize}
%\item Theory and experiment agree well for rotational frequencies reaching approximately $0.995\omega_{\perp}$.
%\item The density profile of slowly rotating condensates differs very little from that of non-rotating condensates, except for the inclusion of the vortex cores.
%\item The Thomas--Fermi regime holds true for frequencies of $0.75 \leq \Omega \leq 0.99$, where kinetic energy remains negligible compared to flow velocity.
%\item The MFQH regime LLL description, holding for $0.99 \leq \Omega \leq 0.999$, sees the vortex cores expanding to fill all available condensate space. Density variation gives rise to energy terms of the order of flow velocity.
%\end{itemize}
%Following these statements, Fetter outlines some goals to observe effects and behaviours predicted to occur in rotating ultracold gas systems that, at the time of writing, still remain to be observed. Approaching and examining the behaviour of a condensate in the fractional quantum Hall regime, wherein $\nu$ is small, remains to be demonstrated. Rapidly rotating Fermi--gases are discussed also as an area requiring future investigation, and which may also provide insight into low $\nu$ states and behaviours.

%\subsection{Optical lattice potentials}\label{sub:optlatt}
%Trapping potentials for Bose--Einstein condensates are typically harmonic, and are formed using magnetic (such as Ioffe--Pritchard) or optical potentials (such as dipole traps). Although these traps provide a single trapping location for all the atoms, interesting behaviour can also be observed and investigated using of an array of traps. One method of achieving this is the use of counter-propagating laser fields, allowing for a periodic optical potential to be created. For two sets of counter-propagating laser fields in a plane at right angles, an array of one-dimensional cigar-shaped condensate traps can be created, wherein the atoms are tightly confined in 2D. Adding an additional set of fields perpendicular to the existing ones will allow trapping of the atoms to periodic locations corresponding to a cubic lattice \cite[~chap. 2]{BK:LH_2010}. %To model bosonic atoms sitting in an optical potential, the Bose--Hubbard Hamiltonian can be considered \cite{BEC:Bloch_revmodphys_2008}. This model accounts for atoms hopping between nearest neighbour sites on the lattice and also includes the presence of an interaction term,
%\begin{equation}\label{eqn:bosehubbard}
%\hat{H}_{BH} = -J \displaystyle\sum\limits_{<i,j>} \left(\hat{a}_i^\dagger \hat{a}_j + \hat{a}_j^\dagger \hat{a}_i \right) + \frac{U}{2}\displaystyle\sum\limits_i \hat{n}_i(\hat{n}_i - 1).
%\end{equation}
%One of the main features of atoms in optical lattices is that they can be tuned to move between the mean field and the strongly correlated regime \cite{THS:McEndoo_2010}. %In this model the system must be assumed to be subjected to the optical lattice on timescales that allow for it to come to equilibrium within the lattice sites. For the work proposed later in Section \ref{sec:prelim}, this model may not be useful initially as the timescales of application are much shorter than those required to allow the system to reach equilibrium. It may become a necessary part of explaining some system behaviour and dynamics in both referenced and proposed future work, and it is thus valuable to discuss it here. Due to the vast amount of available literature I will concentrate on a much smaller subset of research in this area.

%In 2002, the first realisation of a condensate trapped in a three-dimensional optical lattice was achieved by Greiner \textit{et al}. \cite{OL:Greiner_nat_2002}. They successfully realised an optical lattice in which the depth of the individual lattice sites could be adjusted by the intensity of the laser fields. By increasing the intensity from zero allows each site to obtain a number of condensate atoms. The authors show that increasing the intensity of the lattice lasers allows the system to move from a long-range phase coherent (BEC) state, to one where the phases between individual sites are no longer coherent, the so-called Mott insulator state. Such a system can no longer be described by a mean-field theory, and requires the use of a discrete quantum treatment, offered by the Bose--Hubbard model. To ensure the system remains in the ground-state, the intensity of the lattice potential was ramped to the required value slowly compared to the timescales for the system to react. This ensured that no higher modes were excited, and allowed the condensate atoms to distribute amongst the trapping potential.

%A theoretical investigation into the effect of optical lattices on vortex lattice structures was undertaken by Reijnders and Duine \cite{OL:Reijnders_prl_2004}. The authors were able to show that for in the presence of an optical potential which is not matched to the vortex lattice the vortices would break from an Abrikosov pattern, and become pinned to optical lattice sites. This was experimentally verified by Tung \textit{et al}. \cite{Vtx:Tung_prl_2006}. They begin by comparing and contrasting pinned and Abrikosov vortex systems, and find that both allow ``correlated many-body states'' to be investigated. They consider competition between both lattice structures, and investigate the resulting behaviour of the condensate. Taking a 2D optical lattice superimposed upon the condensate in the fast rotation limit, the maxima of the optical potential create regions wherein it is energetically favourable for a vortex to sit. The authors show that a vortex may be pinned to a position given by the optical lattice, and through variation of the optical intensity a structural transition can occur from the triangular Abrikosov lattice to that of the optical lattice pattern. This is favoured when the two lattices rotate at the same rate, allowing the lattices to effectively ``lock''. Using this technique the authors demonstrated the transition to a square lattice structure.

%The use of optical lattices in combination with Bose--Einstein condensates in the manner described previously allows for many interesting effects unique to such systems. One such interesting body of theoretical work was that of S{\o}rensen \textit{et al}. \cite{OL:Sorensen_prl_2005}, which involved the creation of FQH states, in a manner different from the fast-rotation route. The authors describe a process starting with a condensate in a harmonic potential in the Mott insulator state, going to the FQH state via the application of a magnetic field and reduction of the optical lattice potential, where the resulting state is in the LLL. A related work by Palmer \textit{et al}. \cite{OL:Palmer_prl_2006} investigates cold bosons in an optical lattice in the presence of large magnetic fields. The authors show that it is possible to generate FQH states, with varying fractional values. However, the proposed method is rather complex, and is described as requiring near future experimental technology to generate and observe such structures.

%Another body of work investigating such behaviour was carried out by Vignolo \textit{et al}. \cite{Vtx:Vignolo_pra_2007}. They investigate vortex dynamics in a condensate upon excitation by a 2D optical lattice. This theoretical study restricts itself to a single vortex, and can therefore be considered realisable using current state of the art technology \cite{Vtx:Matthews_prl_1999,Vtx:Dobrek_pra_1999,Vtx:Tung_prl_2006}. Assuming a zero-temperature Bose gas, and a Bose--Hubbard Hamiltonian description, the authors derive a means of determining a vortex mass of the resulting system. This mass is stated as being easily tunable through variation of the confining potential along the rotation axis, assuming a pancake-shaped condensate. The use of Feshbach resonances \cite{BEC:Chin_revmod_2010} is also mentioned as a possible tool for altering the scattering length of the atoms, which provides a knob for the vortex mass to be adjusted. Through the use of Bragg scattering spectroscopic techniques, the authors state that in the presence of a vortex, the optical lattice structure exhibits a resonance allowing measurement of the vortex mass, which is absent when no vortex is present. %The paper concludes by mentioning possible further studies, such as the case wherein interaction and tunnelling strengths are of the same order, or when many vortices are present within the system. These are stated to give rise to more complex system dynamics, and are, in the authors opinions, experimentally realisable.

%\subsection{Recent advances and discoveries}
%There has been a significant amount of literature examining condensates with and without rotation and optical lattices, with both experimental and theoretical progress continuing in this area. The group of G. Campbell \cite{Vtx:Wright_pra_2013} examine the behaviour of a condensate in an annular trapping geometry, which is stirred by an optical potential barrier of varying intensity. Beginning with a stationary condensate, stirring was performed at a fixed rotational velocity using a repulsive potential barrier with diameter less than the trapping annulus width. Using the Thomas--Fermi approximation allowed for estimates of the chemical potential, $\mu$, which in turn determined a value for the trap width and barrier width. Starting from zero intensity, the barrier was ramped to a maximum intensity over 100 ms, remained there for 800 ms and decreased over 100 ms. Following a time-of-flight expansion, excitations of the condensate were observed. Release of a non-rotating condensate shows the central hole of the condensate to close, resulting in a large peak at the centre. In contrast, a circulating condensate will show the presence of one or more holes, which signify ``the presence of phase singularities (vortices)''. If a central hole is observed in the condensate density, the authors state that this is resulting from the presence of persistent currents in the condensate, which prior to release were flowing around the annular trap. The resulting diameter of the hole is dependent upon the winding number of the circulating current in the condensate. Holes that were reported as being off-centre show the presence of topological excitations (vortices) in the condensate.  The authors show the probability of these excitations for a range of differing barrier heights and rotational frequencies, and attempt to formulate a theoretical framework to describe the results. Recognising that a full 3D model may be required to accurately describe the behaviour, the authors restrict themselves to a 1D approach, stating that ring-shaped geometries can be effectively treated as such. Although differences exist between experimental results and the theoretical approach taken the results were shown to agree qualitatively, where Campbell \textit{et al}. suggest many possible reasons for such differences. This body of work represents some of the most recent experimental work on rotating condensates, and, as the authors have shown, requires further examination theoretically to accurately explain the results.

<!!!To be added -> results from NZ conference on arbitrary trapping potentials, automated BEC from C3QSl check over last 2 years for additionals!!!>

With this section I have given an overview of theoretical and experimental results dealing with the superfluid behaviour of condensates.
