\section{Superfluidity}\label{sec:superfluid}
\subsection{Introduction to superfluidity}\label{sec:intro_super}

Superfluidity is a macroscopic quantum effect that is closely related to Bose--Einstein condensation. Liquid helium has been known for many years to exhibit superfluid behaviour \cite{BEC:Penrose_pr_1956}. One interesting property of superfluids is that they have quantized circulation which can lead to the appearance of quantum vortices under rotation. Traditionally, such excitations have been created in liquid $^4$He by a ``rotating bucket'' type of experiment, where the container holding the superfluid is rotated about a single axis. When the fluid is initially above the $\lambda$-critical point, the temperature at which $^4$He moves from a classical fluid to a superfluid, the atoms will undergo rotation with the container due to viscous effects. On cooling the atoms below the $\lambda$-critical point, liquid $^4$He will undergo a transition to the superfluid state. If the velocity is above a critical rotation frequency, $\Omega_c$, vortices are nucleated in the rotating superfluid helium system. Due to the strongly interacting nature of liquid helium, the nucleated vortices are difficult to visualise as the healing length, $\xi$, of liquid $^4$He is only on the order of {\r{A}}ngstr{\"o}ms \cite{BEC:Srinivasen_pramana_2006}.

In order to visualise these vortices experimentally it was necessary to use an indirect means of visualisation in the form of tracer particles \cite{BEC:Packard_physb_1982}. Limited success was had with this technique using solid hydrogen, and later plastic microspheres, as they tended to join together due to static charges. An improved technique, using charged particles, showed much greater success \cite{Vtx:Packard_prl_1969}. Ions, or electrons, were trapped inside the vortex lines, and an electric field along the direction of the lines allowed for acceleration of the charges towards a luminescent screen where they could be observed. Current experimental work in vortex visualisation with liquid $^4$He has advanced significantly \cite{Vtx:Tsubota_arxiv_2010,Vtx:Guo_pnas_2014}, yet fine control over the behaviour of the liquid and the vortex dynamics remains difficult.

%\todo[inline]{Derive hydrodynamic formalism for GPE}
Superfluid behaviour is observed in liquid helium due partly to a portion of the atoms condensing into the ground state. Given that BECs show superfluidity, the use of a dilute gas where the majority of atoms are in the condensate state provides a more controlled means to investigate superfluidity and quantised vortices \cite{BK:Ueda_2010,BEC:Srinivasen_pramana_2006,Vtx:Tsubota_arxiv_2010,CT:Tsubota_jpsj_2008}. In contrast with liquid $^4$He, which has a healing length of the order of {\r{A}}ngstr{\"o}ms, the healing length of a dilute gas of alkali atoms is on the order of microns \cite{Vtx:Isoshima_pra_1999}. This places condensates in a much more accessible regime for visualising vortices compared with liquid $^4$He experiments. For many condensates visualisation of the vortices is provided by absorption imaging, following a time-of-flight expansion of the cloud to allow the vortex cores to expand and become more easily visible~\cite{Vtx:Raman_prl_2001,VTX:Rankonjac_pra_2016}. One drawback of this method is that it is destructive, and requires multiple experimental realisations to acquire any dynamical information. To examine the dynamics of the vortices, successive imaging techniques are required. Single-shot experiments have been demonstrated, where a small percentage of the condensate is repeatedly transferred to an untrapped atomic state. This allows for the untrapped atoms to expand, and vortices have been imaged this way~\cite{VTX:Freilich_sci_2010}. With this method the precession of the vortices in a trapped condensate can be directly observed, in a single-shot experiment. More recently, \textit{in-situ} imaging of a condensate with a vortex lattice was demonstrated and allowed resolution of the cores without time-of-flight expansion with a minimally destructive optical refraction technique~\cite{VTX:Wilson_pra_2015}.

To fully understand the behaviour of these systems, it is necessary to have a framework for modeling the condensate, in the absence and the presence of vortices. As described above, a dilute gas of condensed atoms close to absolute zero temperature can be readily modelled by the Gross--Pitaevskii equation \eqref{eqn:gpe_rotation}, offering a direct means to examine condensate behaviour. It is, however, useful to obtain a hydrodynamic description for the condensate, performed by treating its wave-function as
\begin{equation}\label{eqn:madelung}
\Psi(\textbf{r},t) = \sqrt{\rho(\mathbf{r},t)} e^{\textrm{i}\theta(\textbf{r},t)},
\end{equation}
with
\begin{equation}\label{eqn:density}
\rho(\textbf{r},t) = \Psi^*(\textbf{r},t)\Psi(\textbf{r},t) = \vert \Psi (\textbf{r},t) \vert ^2.
\end{equation}
where $\theta(\textbf{r},t)$ is the condensate phase \cite{BK:Pitaevskii_Stringari_2003}. Following the formalism given by Pitaevskii and Stringari \cite{BK:Pitaevskii_Stringari_2003} and multiplying Eq.~\eqref{eqn:gpe} by $\Psi^{*}(\mathbf{r},t)$ then subtracting the complex conjugate of that expression allows one to obtain the continuity equation as
\begin{equation}\label{eqn:continuity}
\frac{\partial}{\partial t}\rho(\textbf{r},t)  + \nabla\cdot \textbf{j}(\textbf{r},t) = 0,
\end{equation}
where the current density of the condensate is
\begin{equation}\label{eqn:current_density}
\textbf{j}(\textbf{r},t) = \frac{-\textrm{i}\hbar}{2m}\left[\Psi^*(\textbf{r},t)\nabla\Psi(\textbf{r},t) - \Psi(\textbf{r},t)\nabla\Psi^*(\textbf{r},t)\right].
\end{equation}
Using ansatz \eqref{eqn:madelung} and substituting it into Eq. \eqref{eqn:current_density} then gives the form for the current density,
\begin{equation}
\textbf{j}(\textbf{r},t) = \vert\Psi(\textbf{r},t)\vert ^2\frac{\hbar}{m}\nabla\theta(\textbf{r},t).
\end{equation}
The velocity of the superfluid, $\textbf{v}(\textbf{r},t)$, is defined as the ratio of the current density to the density, which is then given by
\begin{equation}\label{eqn:velocity}
\textbf{v}(\textbf{r},t)\equiv \frac{\textbf{j}(\textbf{r},t)}{\rho(\textbf{r},t)} = \frac{\hbar}{m}\nabla\theta(\textbf{r},t).
\end{equation}
The gradient of the phase therefore determines the velocity of the condensate atoms; this indicates that the superfluid behaviour in a condensate is irrotational ($\nabla\times(\nabla\theta) =0$). Assuming a closed loop integral about a central point in the condensate, and recalling the single-valued nature of the wavefunction, yields the relationship
\begin{equation}\label{eqn:circulation}
\oint \textbf{v}\cdot d\textbf{l} = \frac{\hbar}{m}2\pi l.
\end{equation}
This shows the quantised nature of circulation in a superfluid, with $l$ representing the integer charge of the circulation. The phase winding around the central region is given by multiples of $2\pi$, with the centre of the phase becoming ill-defined. To circumvent this problem the density at this point drops to zero, signalling the presence of a vortex in the condensate. This drop happens over the scale of the healing length, which, for repulsive interactions, is given by
\begin{equation}
\xi = \frac{1}{\sqrt{8\pi \rho_b a_s}},
\end{equation}
where $\rho_b$ is the bulk density of the condensate, and $a_s$ is the $s$-wave scattering length. For a cylindrically symmetric rotating condensate carrying a single vortex in its centre the wavefunction can be written as
\begin{equation}\label{eqn:madelung_l}
    \Psi = |\Psi|e^{\textrm{i}l\theta},
\end{equation}
where $l$ is the vortex charge. The velocity profile at a distance $r_{v}$ away from the core can then be given as
\begin{equation}\label{eqn:1_over_r}
    v =\frac{\hbar l}{m r_v}.
\end{equation}

To sustain a vortex the condensate must have sufficient angular momentum, which is imparted via the angular rotation frequency times the angular momentum operator $-\Omega L_z$ in the Hamiltonian given by Eq. \eqref{eqn:gpe_rotation}. The rotation frequency, $\Omega$, of the condensate has an upper-bound stability limit equivalent to the transverse trapping frequency, $\omega_{\perp}$, of the harmonically trapped condensate.
%\todo[inline]{Find a better home for speed of sound}

\subsection{Vortices in Bose--Einstein condensates}\label{ss:vorticesinbec}

Given that the circulation of a vortex in a superfluid is quantised, one may assume that for faster rotation rates the circulation increases in integer multiples of $\hbar/m$, beyond critical rotation thresholds to excite each higher multiple. To understand the effect of vortex charge on the condensate, it is necessary to examine the energy functional, as given by
    \begin{equation}\label{eqn:functional}
        E[\Psi] = \int \Psi^{*} (H_{\text{E}}) \Psi,
    \end{equation}
where $H_{\text{E}} = (-\hbar^2/2m)\nabla^2 + V + g|\Psi|^2/2 - \Omega L_z$. By substituting \eqref{eqn:madelung_l} into Eq.~\eqref{eqn:functional} the energy functional is given as
\begin{equation}\label{eqn:functional_full}
    E[\Psi] = \int \frac{\hbar^2}{2m} \left(|\nabla\Psi|^2  + \frac{|\Psi|^2 l^2 m \mathbf{v}^2}{2}  \right) + V|\Psi|^2 + \frac{g}{2}|\Psi|^4 + \Omega \Psi^{*} L_z \Psi,
\end{equation}
where $\mathbf{v}$ is the superfluid velocity, as given by Eq.~\eqref{eqn:velocity}. The $l^2$ term shows that the energy scales quadratically with an increase in charge. This energy growth dependence indicates that it will be more favourable to allow for two singly charged vortices, than a single doubly charged vortex, which will decay due to the presence of complex eigenmodes \cite{VTX:Kawaguchi_pra_2004}. For rates of rotation $\Omega_c < \Omega < \omega_\perp$, where $\Omega_c$, is the threshold rotation rate to create a vortex, the condensate will favour many singly quantised vortices, rather than one or several multiply charged ones. For a superfluid condensate $\Omega_c = {E_v/L_z}$, where $E_v$ is the energy of the vortex, and $L_z$ is the angular momentum component of the superfluid along $z$~\cite{BK:Pitaevskii_Stringari_2003}.

The stability of vortex states in a condensate is also a widely discussed topic \cite{Vtx:Fedichev_pra_1999,Vtx:Feder_prl_1999}, as non-rotating traps show an instability for small displacements of the vortex from the trap centre. Since the ``rotating bucket'' technique used to generate vortices in $^4$He cannot be used for gaseous BECs, many other techniques have been developed \cite{Vtx:Anglin_prl_1999,Vtx:Davies_prl_1999,Vtx:Marshall_pra_1999,Vtx:Dobrek_pra_1999,VTX:Nakahara_physb_2000}. For the work presented in this thesis, the method proposed by Dobrek \textit{et al}. \cite{Vtx:Dobrek_pra_1999}, is of particular interest, as it allows one to optically generate vortices by the use of what they term a ``phase-imprinting'' method. The authors describe a scheme where the phase of the condensate is directly controlled in such a way that the required topological charge to induce a vortex during evolution is provided by external lasers. Through use of an absorption plate whose absorption coefficient depends on the axis angle, the condensate can be imprinted with the required phase pattern. This method will form the basis of one work carried out in this thesis, and will be discussed in detail later in Sec.~\ref{sec:phase}.


\subsection{Vortex lattices}\label{sec:sec2_vtxlatt}

Although a large number of works exist which investigate systems with low numbers of vortices \cite{THS:Davies_2000,Vtx:Chevy_prl_2000,Vtx:Cooper_prl_2001,Vtx:Rosenbusch_prl_2002,Vtx:Ogawa_pra_2002,Vtx:Bretin_joptb_2003,Vtx:Madison_prl_2000,Vtx:Madison_jmo_2000,Vtx:Chevy_aoi_2001,Vtx:Madison_prl_2001,Vtx:Mottonen_jpcm_2002}, such systems do not necessarily form periodic vortex lattices. We will therefore concentrate primarily on studies of systems containing many vortices, assuming large values of $\Omega \lesssim \omega_\perp$. For such systems, the vortices are arranged in a periodic triangular Abrikosov lattice, reminiscent of that which appears in type-II superconductors with magnetic flux lines \cite{Vtx:AboShaeer_sci_2001}. Experimental setups have demonstrated upwards of over 130 vortices in a well ordered triangular formation. The resulting lattices are shown to be highly stable and are ideal setups for investigations of periodic systems \cite{Vtx:AboShaeer_sci_2001,Vtx:Engels_prl_2002}.

The Thomas--Fermi limit discussed earlier describes the case where the kinetic energy term of the Hamiltonian may be neglected in comparison to the interaction energy, as it offers little contribution to the condensate behaviour. This remains true for low rotation rates, however kinetic energy becomes important in the limit of fast rotation. In this case $\Omega/\omega_{\perp}\approx 1$, and the centrifugal force term, $m\Omega^2r$, almost balances with the trapping force term, $-m\omega^2r$. The condensate behaviour then closely resembles that of the two-dimensional quantum Hall regime, where the system is residing in the lowest Landau level, (LLL) ($n=0$) \cite{Vtx:Ho_prl_2001}.  As the rotation of the cloud approaches the trapping frequency, the system tends to the LLL, wherein the nonlinear interaction term becomes relatively weak due to the centrifugal forces on the atomic cloud. In this regime the Thomas--Fermi approximation becomes invalid, as the interaction term no longer dominates over the kinetic energy. The interaction energy is then much smaller than the gap between energy levels for single particle Landau levels. While there have been studies using harmonic-plus-quartic potentials to allow condensates to remain trapped beyond the harmonic trapping frequency limit \cite{BEC:Bretin_prl_2004,Vtx:Ghosh_pra_2004}, we will in this thesis consider only systems in a harmonic confinement.

At high rotation rates very close to the trap frequency ($\Omega/\omega_\perp \approx 1$) the system enters a fractional quantum Hall (FQH) state~\cite{Vtx:Regnault_prl_2003}, and will require treatment beyond that of mean-field methods. This state will require the use of a discrete boson model, which becomes next to impossible to simulate for a realistic number of atoms in a condensate. However, for rotation rates just below this limit a mean-field approach can be used in the so-called ``mean-field quantum Hall'' (MFQH) regime~\cite{BEC:Fetter_revmodphys_2009}. The differences between these two regimes are characterised purely by the ratio of the kinetic energy to the interaction energy of the system, and in the MFQH regime we can assume that the LLL has been achieved~\cite{Vtx:Zhai_pra_2004,BEC:Stock_laserphyslett_2005}. To identify these different regimes one may use the the ``filling factor'', $\nu=N/N_v$ \cite{BK:Ueda_2010,Vtx:Ho_prl_2001}, which examines the ratio of atoms to vortices in the system. This becomes an important characteristic at high rotation rates, and in the case where $1000 > \nu > 10$, the system may be accurately described to be in the MFQH regime. For values $\nu \leq 10$ the system is said to be strongly correlated~\cite{BEC:Fetter_revmodphys_2009}, and the system enters the FQH regime. For an almost perfectly regular vortex lattice, the rotation rate must be sufficiently large so that the condensate width extends to large distances and a large number of vortices are generated, without entering the FQH regime~\cite{Vtx:Aftalion_pra_2005}. To ensure the applicability of mean-field theory choosing rotational frequencies that guarantee this is important.

The distribution of the vortices in the MFQH regime forms a triangular lattice pattern that is almost regular~\cite{Vtx:Schweikhard_prl_2004}. As discussed earlier, the condensate has an irrotational flow profile due to velocity being defined as Eq.~\eqref{eqn:velocity}. This relation holds true provided that $\theta$ is well defined, which is not the case at the centre of a vortex. As the phase at the vortex core is ill-defined, this singular region creates a non-zero curl. A generalisation of the irrotational flow condition (i.e. $\nabla\times \mathbf{v}=0$) to account for this can be given as as~\cite{BK:Pitaevskii_Stringari_2003,BK:Pethick_Smith_2008}
\begin{equation}
    \nabla\times \mathbf{v}=\frac{2\pi l\hbar}{m}\delta^{2}\mathbf{r}_\perp \hat{z},
\end{equation}
where $\delta^2$ is a two-dimensional Dirac delta function, $\hat{z}$ is the unit vector along $z$ and $\mathbf{r}_\perp=(x,y)$. For large rotation frequencies this value becomes very similar to that of a solid-body rotation, as given by $\nabla \times \mathbf{v}=2\Omega$. This result arises for a large regular vortex lattice, where the areal density of vortices can be specified by the Feynman relation~\cite{BK:Pitaevskii_Stringari_2003}
\begin{equation}\label{eqn:feynman}
n_{v} = \frac{m\Omega}{\pi\hbar}.
\end{equation}
In the case of realistic condensates systems, where the densities are not uniform due to harmonic trapping, deviations exist from this value which have been calculated theoretically \cite{Vtx:Sheey_pra_2004_2,Vtx:Sheey_pra_2004} and observed experimentally \cite{VTX:Coddington_pra_2004}. However, for the values chosen in all later discussed simulations these deviations tend to be small, and can in most cases be neglected.

Some key details for the classification of the regime of the condensate are \cite{BEC:Fetter_revmodphys_2009}:
\begin{itemize}
\item Mean-field theory and experiment agree well for large atom numbers ($N>10^3$) and rotational frequencies reaching approximately $0.995\omega_{\perp}$.
\item The density profile of slowly rotating condensates differs very little from that of non-rotating condensates, except for the inclusion of the vortex cores.
\item The Thomas--Fermi regime holds true for frequencies of $0.75 \leq \Omega \leq 0.99$, where kinetic energy remains negligible compared to flow velocity.
\item The MFQH regime description, holding for $0.99 \leq \Omega \leq 0.999$, sees the vortex cores expanding to fill all available condensate space. %Density variation gives rise to energy terms of the order of the flow velocity.
\end{itemize}

Given the above criteria, the rapidly rotating condensate system in our work can be described using mean-field Gross--Pitaevskii theory at a rotation rate of $\Omega = 0.995\omega_{\perp}$, assuming $N\approx 10^6$ atoms, as used in typical experimental setups.

\section{Recent progress and outlook}
Recently there have been many advances in the control of atomic BEC systems. A notable example is the work of Gauthier \textit{et} al. \cite{BEC:Gauthier_arxiv_2016}, wherein they demonstrate arbitrary optical potential generation for Bose--Einstein condensates through use of digital micromirror devices (DMD). The authors demonstrate high resolution control and patterning of the condensate, and show near perfect control of the condensate atomic distribution. Given the current state of the art high performance imaging and control techniques available, these experimental systems can allow for high precision control and manipulations of the atoms, and therefore also vortices.

Another recent experimental work of note is that of Wigley \textit{et} al. \cite{BEC:Wigley_scirep_2016}, with a completely automated approach to BEC generation and control. Such automation can allow for much higher throughput of experimental data collection, and allow for a much wider breadth of physics to be explored in condensate systems.

The current state of the art experimental systems can offer a very high degree of control of condensates, and their ensuing dynamics. In fact, all further methods discussed can be built using currently available state of the art systems, and hence are experimentally realisable.
