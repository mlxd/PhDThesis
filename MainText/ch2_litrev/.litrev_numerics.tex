\section{Numerics}\label{sec:numerics}
For many of the works described previously, it is necessary to apply numerical techniques to obtain results. As the Gross--Pitaevskii
equation, given in Eq. \eqref{eqn:gpe}, is used in the majority of the literature cited, we will consider it as the basis for the following discussion. This equation will also form the basis of the research proposed for my PhD, which will be discussed further in Section \ref{sec:prelim}. Given that the Gross--Pitaevskii equation is a non-linear partial differential equation, analytic solutions are known only for a few specific cases and numerical methods for evaluation are often necessary. Although many methods exist, one widely used family are pseudospectral methods, such as the Fourier split operator method \cite{Num:Bauke_cpc_2011}.

If we consider a unitary evolution operator of the form
\begin{equation}\label{eqn:1}
\Psi(\bar{x},t+\tau) = \exp\left[ -\frac{iH\tau}{\hbar}\right]\Psi(\bar{x},t),
\end{equation}
where H is the Hamiltonian, composed of momentum, potential, non-linear, and rotation terms defined in Eq. \eqref{eqn:gpe}, we can solve for
the wavefunction over a specified timescale. However, due to error propagation resulting from numerical integration techniques, it is necessary to
employ methods that allow for the highest precision while providing results in useful timescales. To allow for this, careful choice of the numerical integration methods must be taken.  If we take the Hamiltonian, $H$, in terms of its components as a combination of position and momentum space functions we obtain
\begin{equation}\label{eqn:2}
\hat{H} = \hat{H}_{\textbf{r}} + \hat{H}_{\textbf{p}} + \hat{H}_{\textbf{L}},
\end{equation}
where we first neglect the angular momentum operator, $\hat{H}_{\textbf{L}}$, and consider only the two other non-commuting parts $\hat{H}_{\textbf{r}}$, containing the position operator, and $\hat{H}_{\textbf{p}}$, containing the momentum operator. This way we can reduce
the error in the numerical integration scheme by using 2nd order Strang-splitting as
\begin{equation}\label{eqn:3}
\exp\left[ -\frac{ i\left(\hat{H}_{\textbf{r}} + \hat{H}_{\textbf{p}}\right)\tau}{\hbar} \right] \approx \exp\left[- \frac{i\hat{H}_{\textbf{r}}\tau}{2\hbar} \right]\exp\left[-\frac{i\hat{H}_{\textbf{p}}\tau}{\hbar}\right]\exp\left[ -\frac{i\hat{H}_{\textbf{r}}\tau}{2\hbar}\right].
\end{equation}
The respective functions can be mapped to the Gross--Pitaevskii equation's position, and momentum terms as
\begin{equation}
\hat{H}_{\textbf{r}} = V(\bar{x}) + g\vert\Psi(\bar{x},t)\vert^2\; \hspace{5em} \hat{H}_{\textbf{p}} = \frac{-\hbar^2}{2m}\nabla^2.
\end{equation}
%\hat{H}_{\textbf{L}} = \Omega L,
Following Bauke \textit{et al}. \cite{Num:Bauke_cpc_2011}, we can numerically solve this differential equation as
\begin{equation}
\Psi\left(\textbf{r},t+\tau\right) = \mathscr{F}^{-1} \left[\frac{\hat{U}_{p}(t+\tau)}{2} \mathscr{F} \left[ \hat{U}_{r}(t+\tau) \mathscr{F}^{-1} \left[ \frac{\hat{U}_{p}(t+\tau)}{2} \mathscr{F} \left[ \Psi\left(\textbf{r},t\right) \right] \right] \right] \right]  \\ + O\left(\tau^3\right),
\end{equation}
where $\hat{U}_{r}=e^{-i\hat{H}_{r}(t)\tau/\hbar}$ is the time evolution operator in real space, $\hat{U}_{p}=e^{-i\hat{H}_{p}(t)\tau/\hbar}$ in momentum space,  $\mathscr{F}$ and $\mathscr{F}^{-1}$ are the forward and backwards Fourier transform respectively. Following \cite{BK:Pitaevskii_Stringari_2003} and taking the Madelung transform of the wavefunction given by Eq. \eqref{eqn:madelung}, the phase of the condensate may be given by
\begin{equation}
\theta = \theta_c + \theta_i,
\end{equation}
where $\theta_c$ is the unperturbed condensate phase, and $\theta_i$ is the phase pattern to be imprinted. Thus, upon solving for the initial condensate phase, an additional phase pattern can be imprinted at any time by multiplying the wavefunction by the intended phase pattern. This is in line with the phase imprinting method, as previously introduced by Dobrek \textit{et al}. \cite{Vtx:Dobrek_pra_1999}. The underlying theory of the Fourier split-operator method for the Gross--Pitaevskii equation is given by Javanainen \textit{et al}. \cite{BEC:Javanainen_jphysa_2006}, showing how the choice of non-linearity and operator splitting affects the outcome of the method. The authors arrive at the conclusion that treating the non-linearity and potential terms together with the most current wavefunction definition yields results with an error magnitude that matches those obtained in the Schr\"{o}dinger Fourier split-operator case, indicating its applicability to this type of problem.

\subsection{Time evolution}
To create the initial state for the desired evolution the ground-state of the Hamiltonian needs to be determined as a first step. This can be achieved by evolving the system in imaginary time, where $t\rightarrow it$. This causes all higher energy terms in an initial guess for the condensate wavefunction to decay to zero, leaving the lowest energy state, which corresponds to the ground-state. As effective as this approach may be, the convergence to the lowest lying energy state becomes less effective as the computation approaches the expected value \cite{Vtx:Danaila_pra_2005}. Although many such methods exist, the one that is best suited for this task is that of a Fourier
split-operator method. Due to the way the algorithm operates, it is essential to have a large and finely sampled grid in order to resolve both position and momentum of the
wavefunction. A minimum grid-size on the order of $2^8$ in 2D for both $X$ and $Y$ dimensions is required. An implementation
of such a method at the defined resolution is a straight-forward process using MATLAB, and has been performed for the purpose of this study.
However, due to the large computational overhead required to deal with such a calculation, the procedure takes a long time to evolve the system to the necessary degree of accuracy. Therefore, it is necessary to further develop the methods used, and to improve the implementation of this algorithm to leverage the recent advances in computational acceleration.
%The use of the phase imprinting technique allows for a system to be mapped with a predefined phase-pattern, corresponding to a required state. With this technique and the imaginary time evolution method described, it will be possible to consider condensate behaviour by imprinting vortices into position, reducing much of the time needed to find the ground-state in the presence of rotation. The ability to manipulate the phase of the condensate also opens the possibility of exploring phase-engineered states, but we shall only consider this as a tool for efficient numerical creation of condensate states.

An additional method for accelerating numerical solutions involves the use of multiple compute cores on a central processing unit (CPU) operating independently on different data elements in unison. This form of parallel computation can be achieved through the use of the OpenMP (Open Multi-Processing)  application programming interface (API), which defines how a program may parallelise certain elements of code. It allows the developer to fully utilise the power of a multicore processor. However, the limit on how much performance can be gained by this method is set by the number of compute cores available to the system. It should also be noted that MATLAB has inherent support for such programming paradigms, and fully abstracts the implementation from the developer. Therefore, in this instance using such a means of parallelisation would not be very beneficial. Another widely used programming paradigm is that of MPI (message passing interface). Where OpenMP allows a user to utilise all available processors on a single system, MPI allows the use of an (almost) unlimited number of computer systems operating in parallel together, each known as a node. This is the method generally preferred in programs written for compute clusters, where a large number of nodes are available to use. It is preferable for applications that have minimal dependence between data, as a compute bottleneck may occur if data spread over multiple nodes is required for an operation. This would require continual transmission of data  between individual nodes, which (at current data rates) would be limited to bus speeds of (assuming Infiniband optical connections) on the order of ten gigabytes per second. Compared to a local calculation requiring little to no transfers, the memory bandwidth can be (assuming current high performance 12 core processors) as high as 60 gigabytes per second \cite{DAT:Intel_xeon}. Therefore, it is important to note that transfers should be minimised to avoid bottlenecks, but transfers are often necessary to make use of the large number of processing cores required to obtain results within short timescales. Therefore, to give a significant performance benefit, a large number of cores, a high memory bandwidth, a high-speed interconnect between cores (nodes), as well as sufficient space to store the problem in memory are required.

One possible means of achieving this is through the use of graphics processing units (GPUs). GPUs are signal processing devices, and have been highly developed over the past 20 years to offload much of the computation required to display images from the central processing unit (CPU). As a result, GPUs have been given the task of performing operations necessary to update a large number of pixels in a short amount of time. This has been achieved through giving the GPUs a large number of specialised compute cores for floating-point arithmetic, effectively operating in parallel. With the advent of general purpose GPU (GPGPU) computing, the ability to exploit these cores for the purpose of numerical computing has become possible. A problem can be mapped to the hardware of a GPU, and all parallelisable operations can be accelerated, reducing the overall compute overhead required for evaluating results. For the latest generation of industry standard GPUs used in computational
acceleration the memory bandwidth for the device global memory (equivalent of RAM) is given as 288 gigabytes per second, with almost 3000 cores on demand, yielding a total of $1.41\times10^{12}$ floating-point operations per second (FLOPS). In comparison to this are Intel Xeon CPU throughput values, which yield approximately $1\times10^{11}$ FLOPS. As can be seen, performance of an order of magnitude can be gained by using a GPU for calculations, over high-performance (Xeon) CPUs. This has been shown to allow for effective implementation of the previously mentioned Fourier split-operator method \cite{Num:Bauke_cpc_2011}, and we have shown that it yields performance exceeding that of CPU's for a modest choice of GPU \cite{AO:Morgan_pra_2013}.

\subsection{Recent advances}
Although it has been shown to be adequate for evaluating the Gross--Pitaevskii equation, the use of the Fourier split-operator method becomes problematic when considering rotation, as seen by Zhang \textit{et al}. \cite{BEC:Zhang_anm_2007}. The authors state that the available methods dealing with rotation ``have low-order accuracy in space''. The rotational term in the Hamiltonian complicates the implementation of a general method for rotating condensates, and fails for high rotation rates when the angular momentum operator magnitude approximates that of the other operators. Therefore, Zhang \textit{et al}. develop a formalism for dealing with such rotations, which is more accurate and efficient than those currently available. In their following paper, Zhang and Rong \cite{Vtx:Zeng_cpc_2009} extend this approach, developing a formalism for a rotating condensate approaching the fast rotation limit, which they show to be effective in both 2D and 3D. Having examined the groundstate images from their respective works, it seems that the authors have failed to generate a perfectly ordered triangular Abrikosov lattice, as can be seen from the presence of dislocations. In line with this, the authors have described further numerical methods for rotating condensates, wherein the same method is utilised in two papers, one written by Bao \textit{et al}. \cite{Num:Bao_siam_2013}, and Ming \textit{et al}. \cite{Num:Ming_jcp_2014}. Employing a Lagrangian coordinate transform the authors in both papers describe a means of removing the troublesome angular momentum term from the Gross--Pitaevskii equation, whilst still considering rotation. These papers represent latest research on numerical methods appropriate for modelling of rotating condensates.
