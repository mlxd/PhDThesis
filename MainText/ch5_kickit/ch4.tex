\chapter{How to use the templates} \label{ch-2}

This is a practical guide into how to use this template, by explaining the role of the different folders, \texttt{Temporary\_Thesis.tex} and  \texttt{Final\_Thesis.tex}, the main file to compile the temporary version of the thesis and the final version espectively.

\section{Folders}

The main folder contains three folders detailed here:

\begin{itemize}

\item \textbf{Images.} This folder should contain all the images that you will use in your thesis. It can contain subfolders, for example one for each chapter. To include an image from the main text, use something like \texttt{\textbackslash includegraphics\{subfolder/image.jpg\} } without worrying about the \texttt{Images} path.

\item \textbf{MainText.} This folder contains a series of \LaTeX\ files that form the main text: introduction, chapters, conclusion and appendices. The introduction and conclusion as they are now are not numbered, which creates a few difficulties with the headers of the thesis. Those are solved by including the commands \texttt{\textbackslash unnumberedchapter\{\}} and \texttt{\textbackslash numberedchapter} before including the files in \texttt{xxx\_Thesis.tex}. If you want the introduction and conclusion to be numbered, re-write and treat them as regular chapters.

\item \textbf{Preamble.} This folder contains a series of \LaTeX\ files with the pages that will appear before the main text. Please write (or copy and paste) your own text in those files and delete the dummy text when appropriate. The files are:
\begin{itemize}
\item \texttt{abbreviations.tex} --- List of abbreviations. If the list goes over one page, create another table.
\item \texttt{abstract.tex} --- Abstract. Follow directions in the file.
\item \texttt{acknowledgments.tex} --- Acknowledgments. Follow directions in the file.
\item \texttt{declaration.tex} --- Declaration of Original and Sole Authorship. Only modify the last item. This page needs to be signed once printed.
\item \texttt{dedication.tex} --- Dedication (optional). Should only be a very few lines.
\item \texttt{glossary.tex} --- Glossary (optional). If the list goes over one page, create another table.
\item \texttt{mydefinitions.tex} --- \textbf{Important} --- This file should contain all the values relevant for the title page (name, thesis title, etc, which will be used automatically in the title and various preamble files), your bibliography style, all packages you need for your thesis and your custom definition and commands. Be careful of not importing a package that has already been imported in \texttt{xxx\_Thesis.tex}, and be aware that some packages might interfere with each other.
\item \texttt{nomenclature.tex} --- Nomenclature (optional). If the list goes over one page, create another table.
\item \texttt{physics\_bibstyle.bst} --- Bibliography style file modified by Jeremie Gillet in 2011 to suit his thesis. Might be suitable for physics. If you want to use another custom bibliography style, include the file in this folder.
\item \texttt{Thesis\_bibliography.bib} --- BibTeX file containing your bibliography.
\end{itemize}

\end{itemize}

\section{\texttt{Temporary\_Thesis.tex} and  \texttt{Final\_Thesis.tex}}

Those are the main files, the only ones that need to be compiled to build the thesie. Compile once with \LaTeX, once with BibTeX and finally twice with \LaTeX\ to get all the references right.

Let's go through each section and comment them briefly. The last section will emphasize the differences between the two files.

\subsection{PACKAGES AND OTHER DOCUMENT CONFIGURATIONS}

This section contains the minimum number of packages and definitions to compile the thesis. No line should be removed or modified.

\subsection{ADD YOUR CUSTOM VALUES, COMMANDS AND PACKAGES}

This section should not be modified directly. Instead, your packages and definitions should be included in  \texttt{Preamble/mydefinitions.tex}.

\subsection{TITLE PAGE}

Creates the title page. Do not modify.

\subsection{PREAMBLE PAGES}

Structures the style (header) for the preamble pages and builds them. Do not modify, except for deleting the optional preambles you might not want to include.

\subsection{LIST OF CONTENTS/FIGURES/TABLES}

Creates the different lists. Do not modify.

\subsection{THESIS MAIN TEXT}

Structures the style for the main text chapters and builds them. 

The command \texttt{\textbackslash numberedchapter} is only relevant for a transition between unnumbered sections and numbered sections, it does not need to be included between each chapter. 

\subsection{APPENDICES}

Structures the style for the appendices and builds them. The appendices are numbered with letters but are structured like regular chapters.

\subsection{BIBLIOGRAPHY}

Builds the bibliography. The style of the bibliography can be defined in \texttt{Preamble/mydefinitions.tex}.

\subsection{PUBLISHED ARTICLES}

This last section add the PDF files of your previously published articles (or about to be published) to the thesis. You should only include PDF files provided by the publishing journal. This is strictly for the examiners' convenience in the temporary bound thesis, as for copyright reasons these files may not be published in the final version of the thesis.

\subsection{Differences between \texttt{Temporary\_Thesis.tex} and  \texttt{Final\_Thesis.tex}}

There are two main differences between the two versions. 

The first difference is that the final version does not contain the published articles for copyright reasons. 

The second difference is in the document style: page size, header and line spacing are different This might create small issues, such as page breaking with large tables, images or captions, when compiling the same content with the two \texttt{xxx\_Thesis.tex} files.



