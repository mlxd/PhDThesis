
This body of work was carried out during my time as a Ph.D student at Okinawa Institute of Science and Technology Graduate University (OIST), and grew out of work and ideas I pursued at University College Cork (UCC), Ireland. The purpose of the work described herein is to understand the dynamics of rapidly rotating Bose--Einstein condensates subjected to perturbations, and to control and engineer specific states during the evolution. While it is possible derive some analytical solutions to rapidly rotating condensate behaviour (e.g. lowest Landau level), such solutions are rare. This thesis concentrates on the numerical solutions of the Gross--Pitaevskii equation, and the resulting dynamics of those solutions.

%This work is motivated by the use of Bose--Einstein condensate (BEC) systems as model tools to investigate turbulent, and leading to chaotic quantum behaviour.

This work is motivated by gaining an understanding of the dynamical behaviour of quantum vortices as they appear in an Abrikosov geometry subjected to perturbation. %A further application of this work would be the use of such tools to model and investigate turbulent and chaotic quantum behaviour.

 Understanding these can help with engineering quantum states for future technologies. Ideally, these systems can be used for long-term memory storage for quantum computing systems as they are topologically protected and are very robust against perturbations. These type of systems allow the study of quantum mechanical effects on mesoscopic scales, making them a great tool for simulating condensed matter physics. A further application of this work is the use of such tools to model and investigate turbulent and chaotic quantum behaviour. Given the difficulties in understanding turbulence in classical systems, quantum turbulence offers a more controllable route. Given these statements, being able to manipulate and engineer specific states, and thus behaviour, from BECs requires novel new tools. There are two types of perturbations one may examine with these systems: i) the modification of the phonon spectrum of the condensate which does not influence the angular momentum. I have investigated the use of kicked optical potentials for this; ii) direct control of the topological excitations, and hence angular momentum. This is performed with direct phase engineering of the wavefunction. I examine both in this order, and investigate their usefulness for generating the states described.

To do this I assume a trial system of a rapidly rotating BEC having a large number of vortices, arranged in a triangular Abrikosov lattice pattern pattern. The wavefunction is subjected to the different manipulations listed earlier, and I examine the resulting dynamics. This system requires the solution of a two-dimensional partial differential equation at high resolution subjected to a variety of different initial conditions and perturbation. This is a non-trivial numerical problem, and requires the use of advanced numerical computing techniques to allow for results in a reasonable time. For this I make use of graphics processing unit (GPU) computing, and I will discuss the development of such tools and compare them against conventional simulation techniques.

The thesis will be outlined as follows:

\section{Chapter 2}
I will introduce the field of cold-atomic gas systems, and discuss the theory of Bose--Einstein condensation. The field will be thoroughly examined and discussed via a literature review, examining theory, experimentation, origins and also future of the field. Emphasis will be placed on material and works relevant to the study performed herein. I outline the necessary topics of relevance to the work, and discuss implications of each.

\iffalse
Introduction (~5-10 pages)

    - Thesis Statement (one or two sentences)
        What is your thesis about and what have you done?
        If you have a hypothesis what is it?
        How will you test (prove/disprove) your hypothesis?
    - Motivation
        Why is this problem you've worked on important
    - Goals / Objectives
        What are you trying to do and why?
        How will you or the reader know if or when you've met your objectives?
    - **** Contributions *****
        What is new, different, better, significant?
        Why is the world a better place because of what you've done?
        What have you contributed to the field of research?
        What is now known/possible/better because of your thesis?
    Outline of the thesis (optional)
\fi

\section{Chapter 3}
    Here I examine the use of methods for numerically solving the Bose--Einstein condensate. I introduce the necessary algorithms and considerations to simulate such a system. General purpose GPU (GPGPU) computing will be introduced here, with the implementation of the Gross--Pitaevskii equation discussed. Calculation of the Bogoliubov spectrum, and its implementation is discussed here also.

\iffalse
Background / Related Work (~8-20 pages)
    More than a literature review
    Organize related work - impose structure
    Be clear as to how previous work being described relates to your own.
    The reader should not be left wondering why you've described something!!
    Critique the existing work - Where is it strong where is it weak? What are the unreasonable/undesirable assumptions?
    Identify opportunities for more research (i.e., your thesis) Are there unaddressed, or more important related topics?
    After reading this chapter, one should understand the motivation for and importance of your thesis
    You should clearly and precisely define all of the key concepts dealt with in the rest of the thesis, and teach the reader what s/he needs to know to understand the rest of the thesis.
\fi

\section{Chapter 4}
\iffalse
Theory / Solution / Program / Problem (~15-30 pages)

    continuing from Chapter 2 explain the issues
    outline your solution / extension / refutation
\fi

\section{Chapter 5}

\iffalse Implementation / Formalism (~15-30 pages)
    not every thesis has or needs an implementation
\fi

\section{Chapter 6}

\iffalse
Results and Evaluation (~15-30 pages)

    adequacy, efficiency, productiveness, effectiveness (choose your criteria, state them clearly and justify them)
    be careful that you are using a fair measure, and that you are actually measuring what you claim to be measuring
    if comparing with previous techniques those techniques must be described in Chapter 2
    be honest in evaluation
    admit weaknesses
\fi

\section{Chapter 7}
\iffalse
 Conclusions and Future Work (~5-10 pages)

    State what you've done and what you've found
    Summarize contributions (achievements and impact)
    Outline open issues/directions for future work
\fi
