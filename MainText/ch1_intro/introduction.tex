
This body of work was carried out during my time as a Ph.D student at Okinawa Institute of Science and Technology Graduate University (OIST), and grew out of work and ideas I pursued at University College Cork (UCC), Ireland. The purpose of the work described herein is to understand the dynamics of rapidly rotating Bose--Einstein condensates subjected to perturbations, and to control and engineer specific states during the evolution. While it is possible to derive some analytical solutions to rapidly rotating condensate behaviour (e.g. lowest Landau level), such solutions are rare. This thesis concentrates on the numerical solutions of the Gross--Pitaevskii equation, and the resulting dynamics of those solutions. This work is motivated by gaining an understanding of the dynamical behaviour of quantum vortices as they appear in an Abrikosov geometry subjected to perturbation. %A further application of this work would be the use of such tools to model and investigate turbulent and chaotic quantum behaviour.

Understanding these vortex systems can help with engineering quantum states for future technologies. Ideally, they can be used for long-term memory storage in quantum computing systems as they are topologically protected and are very robust. These types of systems allow the study of quantum mechanical effects on mesoscopic scales, making them a great tool for simulating condensed matter physics. A further application of this work is the use of such tools to model and investigate turbulent, and possibly chaotic, quantum behaviour. Due to the difficulties in understanding turbulence in classical systems, quantum turbulence offers a more controllable route. Given these statements, being able to manipulate and engineer specific states, and thus behaviour, from BECs requires novel new tools. There are two types of perturbations one may examine with these systems: i) the modification of the phonon spectrum of the condensate which does not influence the angular momentum. I have investigated the use of kicked optical potentials for this; ii) direct control of the topological excitations, and hence angular momentum. This is performed with direct phase engineering of the wavefunction. I examine both in this order, and investigate their usefulness for generating the states described.

To do this I assume a trial system of a rapidly rotating BEC having a large number of vortices, arranged in a triangular Abrikosov lattice pattern. The wavefunction is subjected to the different manipulations listed earlier, and I examine the resulting dynamics. This system requires the solution of a two-dimensional partial differential equation at high grid resolution subjected to a variety of different initial conditions and controllable perturbations. This is a non-trivial numerical problem, and requires the use of advanced numerical computing techniques to allow for results in a reasonable time. For this I make use of graphics processing unit (GPU) computing, and I will discuss the development of such tools and compare them against conventional simulation techniques.

The thesis will be outlined as follows:

\section{Background}
I will introduce the field of cold-atomic gas systems, and discuss the theory of Bose--Einstein condensation. The field will be thoroughly examined and discussed, looking at theory, experimentation, origins and also the future of the field. Emphasis will be placed on material and works relevant to the study performed herein. I outline the necessary topics of relevance to the work, and discuss implications of each. I will present a derivation of the Gross--Pitaevskii equation, used to model Bose--Einstein condensates, as well as a discussion on the Bogoliubov-de Gennes equations. I will next discuss the hydrodynamic form of the condensate, and give the hydrodynamic form of the Gross--Pitaevskii equation. Here I introduce superfluidity, and the nature of quantised vortices in these systems. I will conclude with an outlook on the cutting edge work in the field in the context of condensate trapping and control.


\section{Chapter 3}
Here I examine the use of methods for numerically solving the Gross--Pitaevskii equation for simulating the Bose--Einstein condensate. I introduce the necessary algorithms and considerations to simulate such a system. The Fourier split-operator method will be introduced, as well as the need for imaginary time evolution, and considerations required to effectively simulate the condensate. Graphics processing unit (GPU) computing will be introduced here, with the implementation of the Gross--Pitaevskii equation discussed. I will present a difficult numerical problem here, namely the solution of a full three-dimensional Schr\"odinger equation. The use of GPU computing makes this problem tractable in realistic times. The work focuses on the area of adiabatic control techniques, and demonstrates the use of GPU computing to discover the dynamics of a system for observing matter-wave STIRAP. Relevant works to this discussion are presented by [PRA STIRAP]\cite{}, as well as [GPUE]\cite{}.

\section{Chapter 4}
I examine the dynamics of the condensate here, and discuss the methods used for probing the condensate system. I begin by introducing some dynamical behaviour of the condensate in the presence of vortices, and introduce the model system used for further discussions. I present the velocity profiles and discuss some of the behaviour expected in a condensate with many vortices. I will follow this with an introduction to two main perturbation methods for the condensate of which I will later use: optical kicking, and phase imprinting. Both methods will be introduced, with a discussion on their uses in condensate systems. I will also discuss the analysis of the condensate dynamics, concentrating primarily on the kinetic energy spectrum.

\section{Chapter 5}
Here I investigate the use of optical kicking and the resulting effects on the vortex lattice carrying condensate. The robustness of the vortex lattice is demonstrated while being kicked by a matching optical potential, showing little to no deviation of ideal vortex positions. The resulting condensate density though shows the appearance of a superlattice pattern. I analyse this system, and demonstrate that the resulting superlattice pattern is from interference in reciprocal space between the kicking optical potential and the present vortex lattice. Moir\'e interference theory accurately predicts the observed behaviour, and is backed up by examining the kinetic energy spectrum of the condensate. Uses of this technique are discussed to conclude. Relevant works to this chapter are [PRA MOIRE]\cite{}.

\section{Chapter 6}
Phase imprinting of the condensate is here examined as a means to investigate the robustness of the lattice. Given the order observed during the optical kicking, the necessary requirements to cause a disordering of the lattice are investigated. Through phase imprinting, lattice vacancies and topological defects are created in the vortex lattice, and their behaviour investigated over long times. The vortex lattice demonstrates highly robust behaviour, even in the presence of such defects. I discuss the use of this method for creating varying degrees of disorder in the lattice, and propose it as a system for investigating order/disorder transitions. Relevant works to this are [PRA DEFECT]\cite{}.

\section{Chapter 7}
Here I conclude the work discussed in the thesis, and discuss extensions, and future ideas for the field.
