%----------------------------------------------------------------------------------------
%	VALUES FOR THE THESIS
%----------------------------------------------------------------------------------------

\newcommand{\name}{Lee James O'Riordan} % Author name
\newcommand{\thesistitle}{The life and death of vortex 161} % Title of the thesis
\newcommand{\submissiondate}{September, 2016} % Submission date "Month, year"
\newcommand{\supervisor}{Prof. Thomas Busch} % Supervisor name
%\newcommand{\cosupervisor}{C.~O'Supervisor} % Co-Supervisor name, comment this line if there is none


%----------------------------------------------------------------------------------------
%	BIBLIOGRAPHY STYLE (pick the style you want)
%----------------------------------------------------------------------------------------

\usepackage[square, numbers, sort&compress]{natbib} % for bibliography - Square brackets, citing references with numbers, citations sorted by appearance in the text and compressed (as in [4-7])
%\usepackage[longnamesfirst,round]{natbib} % Natural Sciences bibliography

%\bibliographystyle{Preamble/physics_bibstyle} % You may use a different style adapted to your field
\bibliographystyle{unsrtnat} % You may use a different style adapted to your field


%----------------------------------------------------------------------------------------
%	YOUR PACKAGES (be careful of package interaction)
%----------------------------------------------------------------------------------------

\usepackage{amsthm,amsmath,amssymb,amsfonts,bbm}% Math symbols
\usepackage{flexisym,mathrsfs,cancel}
\usepackage[parfill]{parskip}
\usepackage{hyperref}

\usepackage{chapterbib}
%\usepackage[utf8]{inputenc}
%----------------------------------------------------------------------------------------
%	YOUR DEFINITIONS AND COMMANDS
%----------------------------------------------------------------------------------------

% New Commands
\renewcommand{\baselinestretch}{1.2}


\newcommand{\bea}{\begin{eqnarray}} % Shortcut for equation arrays
\newcommand{\eea}{\end{eqnarray}}
\newcommand{\e}[1]{\times 10^{#1}}  % Powers of 10 notation

% Defining a theorem box for Criteria
\newtheorem{critere}{Criterion}
\newcommand{\crit}[2]{
\begin{center}
\fbox{ \begin{minipage}[c]{0.9 \textwidth}
\begin{critere}
\textbf{\textup{ #1}} --- #2
\end{critere}
\end{minipage}  } \end{center}
}
