%----------------------------------------------------------------------------------------
%	VALUES FOR THE THESIS
%----------------------------------------------------------------------------------------

\newcommand{\name}{Lee James O'Riordan} % Author name
%\newcommand{\thesistitle}{The life and death of vortex 161} % Title of the thesis
\newcommand{\thesistitle}{Non-equilibrium vortex dynamics in rapidly rotating Bose--Einstein condensates} % Title of the thesis
\newcommand{\submissiondate}{October, 2016} % Submission date "Month, year"
\newcommand{\supervisor}{Prof. Thomas Busch} % Supervisor name
%\newcommand{\cosupervisor}{C.~O'Supervisor} % Co-Supervisor name, comment this line if there is none


%----------------------------------------------------------------------------------------
%	BIBLIOGRAPHY STYLE (pick the style you want)
%----------------------------------------------------------------------------------------

\usepackage[square, numbers, sort&compress]{natbib} % for bibliography - Square brackets, citing references with numbers, citations sorted by appearance in the text and compressed (as in [4-7])
%\usepackage[longnamesfirst,round]{natbib} % Natural Sciences bibliography

%\bibliographystyle{Preamble/physics_bibstyle} % You may use a different style adapted to your field
%\bibliographystyle{unsrtnat} % You may use a different style adapted to your field
\bibliographystyle{unsrtnat} % You may use a different style adapted to your field


%----------------------------------------------------------------------------------------
%	YOUR PACKAGES (be careful of package interaction)
%----------------------------------------------------------------------------------------


\usepackage{amsthm,amsmath,amssymb,amsfonts,bbm}% Math symbols
\usepackage{flexisym,mathrsfs,cancel}
\usepackage[parfill]{parskip}
\PassOptionsToPackage{hyphens}{url}\usepackage{hyperref}
%\usepackage[hyphens]{url}

\usepackage{chapterbib}
\usepackage{float}


%\iffalse
%%%%%%% For stopping figures taking their own pages
\setcounter{topnumber}{2}
\setcounter{bottomnumber}{2}
\setcounter{totalnumber}{4}
\renewcommand{\topfraction}{0.85}
\renewcommand{\bottomfraction}{0.85}
\renewcommand{\textfraction}{0.15}
\renewcommand{\floatpagefraction}{0.8}
\renewcommand{\textfraction}{0.1}
\setlength{\floatsep}{5pt plus 2pt minus 2pt}
\setlength{\textfloatsep}{5pt plus 2pt minus 2pt}
\setlength{\intextsep}{5pt plus 2pt minus 2pt}
%%%%%%%
%\fi

%DOOM

\newfontfamily\doomfontL[Path=./fonts/]{AmazDooMLeft.ttf}
\newfontfamily\doomfontR[Path=./fonts/]{AmazDooMRight.ttf}
\newfontfamily\doomfontLO[Path=./fonts/]{AmazDooMLeftOutline.ttf}

%%%

\DeclareMathOperator*{\atantwo}{atan2}
\DeclareMathOperator*{\argmin}{arg\,min}

%\DeclareMathOperator*{\textprime}{'}

\newcommand{\lee}{\textcolor{red}}
\usepackage{todonotes}

%\usepackage{unicode-math}
%\setmathfont{xits-math.otf}


%\usepackage[utf8]{inputenc}
%----------------------------------------------------------------------------------------
%	YOUR DEFINITIONS AND COMMANDS
%----------------------------------------------------------------------------------------

% New Commands
    \renewcommand{\baselinestretch}{1.2}


\newcommand{\bea}{\begin{eqnarray}} % Shortcut for equation arrays
\newcommand{\eea}{\end{eqnarray}}
\newcommand{\e}[1]{\times 10^{#1}}  % Powers of 10 notation

% Defining a theorem box for Criteria
\newtheorem{critere}{Criterion}
\newcommand{\crit}[2]{
\begin{center}
\fbox{ \begin{minipage}[c]{0.9 \textwidth}
\begin{critere}
\textbf{\textup{ #1}} --- #2
\end{critere}
\end{minipage}  } \end{center}
}
