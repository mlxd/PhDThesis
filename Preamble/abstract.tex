\unnumberedchapter{Abstract}
\chapter*{Abstract}
\subsection*{\thesistitle}

This body of work examines the non-equilibrium dynamics of vortex lattice carrying Bose--Einstein condensates. We solve the mean-field Gross--Pitaevskii equation for a two-dimensional pancake geometry, in the co-rotating frame within the limit of high rotation frequencies. The condensate responds to this by creating a large periodic lattice of vortices with 6-fold triangular symmetry. By applying two distinct perturbations to this lattice, we examine the resulting effects on the vortices during time evolution. The first perturbation involves applying an optical potential with matching geometry to the vortex lattice. We observe the appearance of interference fringes, and we show that these can be described by moir\'e interference theory. This is backed up by a decomposition of the kinetic energy spectra of the condensate. The applied perturbation only modifies the condensate density, with the vortex positions largely unaffected. From this we conclude that the vortex lattice is very stable and robust against phononic disturbances.

Next, by removing vortices at predefined positions in the lattice using phase imprinting techniques, we examine the resulting order of the lattice. By performing this we generate stable topological defects in the crystal structure.  The resulting lattice remains highly ordered in the presence of low numbers of these defects, where crystal structure and order of the lattice shows to be highly robust. By varying the type of imprinted phases we can create controllable degrees of disorder in the lattice. This disorder is analysed using orientational correlations, Delaunay triangulation, and Voronoi diagrams of the vortex lattice, and demonstrates a method for examining order and generating disorder in vortex lattices in Bose--Einstein condensates.

All work described makes extensive use of GPU computing techniques, and allows for the simulation of these systems to be realised in short times. The implementation of the calculations using GPU computing are also discussed, where the software is shown to be the fastest of its kind out of the independently tested software suites. 
