\unnumberedchapter{Abstract}
\chapter*{Abstract}
\subsection*{\thesistitle}

This body of work examines the non-equilibrium dynamics of vortex lattice carrying Bose--Einstein condensates. We solve the mean-field Gross--Pitaevskii equation for a two-dimensional pancake geometry, in the co-rotating frame within the limit of high rotation frequencies. The condensate responds to this by creating a large periodic lattice of vortices with 6-fold triangular symmetry. By applying two distinct perturbations to this lattice, we examine the resulting effects on the vortices during time evolution. The first perturbation involves applying an optical potential with matching geometry to the vortex lattice. We observe the appearance of interference fringes, and we show that these can be described by moir\'e interference theory. This is backed up by a decomposition of the kinetic energy spectra of the condensate. Next, by removing vortices at predefined positions in the lattice using phase imprinting techniques, we generate stable topological defects in the crystal structure. By varying the type of imprinted phases a controllable degree of disorder is created in the lattice. This is analysed using orientational correlations, Delaunay triangulation, and Voronoi diagrams of the vortex lattice. All work described makes extensive use of GPU computing techniques, and allows for the simulation of these systems realisable in short times. The implementation of the calculations using GPU computing are also discussed.
