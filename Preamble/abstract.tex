\unnumberedchapter{Abstract}
\chapter*{Abstract}
\subsection*{\thesistitle}
\iffalse
Maximum 400 words, not to exceed one A4 page.

No figures or tables.  No references.  Just aims, brief methods, results, brief conclusions.

Avoid over-detailed technical method descriptions.

Should be readable to a literate science reader familiar with your general area, but not necessarily experts-only material

This will be published online within 3 months of award of the degree, as a minimum.  The entire thesis must be published within one year, unless restrictions apply (as above).
\fi
This body of work examines the non-equilibrium dynamics of vortex lattice carrying Bose--Einstein condensates. We solve mean-field Gross--Pitaevskii equation in a two-dimensional pancake geometry, in the corotating frame with a rotation frequency of $\Omega\omega_\perp=0.995$. The condensate responds to this by creating a large periodic lattice of vortices with 6-fold triangular symmetry. By applying two distinct perturbations to this lattice, we examine the resulting effect on the vortices during time evolution. The first perturbation involves applying an optical potential with matching geometry to the vortex lattice. We observed the appearance of interference fringes, and we show that these can be described by moir\'e interference theory. This is backed up by a decomposition of the kinetic energy spectra of the condensate. Next, by removing vortices at predefined positions in the lattice using phase imprinting techniques, we create stable topological defects in the crystal structure. By varying the type of imprinted phases a controllable degree of disorder was created in the lattice. This was analysed using orientational correlations, Delaunay triangulation, and Voronoi diagrams of the vortex lattice. All work described made heavy use of GPU computing techniques, and made the simulation of these situations realisable in short times.
